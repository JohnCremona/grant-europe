\newcommand{\CS}{computer science}
\newcommand{\MATH}{mathematics}

\section*{Germany}

\subsection*{Leibniz Universitat Hannover}
\begin{itemize}
\item Miguel Angel Marco Buzunariz (Postdoc in the Institut für Algebraische Geometrie)
  %trac: mmarco, email: mmarco at unizar.es or miguel at math.uni-hannover.de
\end{itemize}

\subsection{Friedrich--Schiller--Universität Jena}
\begin{itemize}
\item Simon King (Postdoc in department of \MATH and \CS)
  % trac: SimonKing
  % github: simon-king-jena
  % http://users.minet.uni-jena.de/~king/
  % email: simon dot king at uni-jena dot de
  % Core developer in Sage (Coercion, categories, Groebner bases, group cohomology)
  % Current research topics:
  % - Modular cohomology of finite groups
  % - Standard basis theory for path algebra quotients
  % Previous research topics:
  % - Non-modular invariant theory of finite groups
  % - Algebraic and geometric topology
  % - Oriented matroids
\end{itemize}

\section*{Austria}

\begin{itemize}
\item Manuel Kauers (RISC Linz)
\item Emil Widmann (Salzburg)
\item Martin Rubey (Technische Universität Wien)
\end{itemize}

\section*{Switzerland}
\subsection*{Universität Zurich}
\begin{itemize}
\item Paul-Olivier Dehaye (SNSF Ass. Professor)
  %pdehaye paul-olivier.dehaye at math.uzh.ch
\item Valentin Feray (Ass. Professor, \MATH)
  %http://user.math.uzh.ch/feray/ valentin.feray at math.uzh.ch
\end{itemize}
\subsection*{ETH Zurich}
\begin{itemize}
\item Martin Raum (ETH Postdoctoral Fellow, ETH Zürich)
  % martin@raum-brothers.eu, www.raum-brothers.eu/martin
  % trac: mraum, github: martinra
  % (supported by Marie Curie COFUND). 
  % At the moment is seems like I will start in September at MPIM in Bonn.
\end{itemize}

\section*{Great Britain}
\subsection*{University of Warwick, England}
\begin{itemize}
\item Bruce Westbury (Associate Fellow)
 %bruce bruce.westbury at gmail.com Bruce.Westbury at warwick.ac.uk
\item John Cremona (Professor)
\item Peter Bruin (Assistant Professor)
\end{itemize}

\subsection*{University of Oxford, England}
\begin{itemize} 
\item Dmitrii Pasechnik (Senior Research Fellow, \CS)
  % http://www.cs.ox.ac.uk/people/dmitrii.pasechnik/
  % dimpase, dimpase
\item Volker Braun (postdoc at the Mathematical Institute)
\end{itemize}

\subsection*{University of St. Andrews, Scotland}
GAP Group people (?)

\section*{Belgium}

\begin{itemize}
\item Jeroen Demeyer (postdoc, Dept. of Mathematics, Ghent University)
\item other people using Sage for teaching at Ghent University (?)
\end{itemize}

\section*{The Netherlands}

\begin{itemize}
\item Rudi Pendavingh (Assistant Professor, Dept. of Mathematics, TU Eindhoven)
% a major contributor to Sage matroids code
\end{itemize}

\section*{Québec}

After checking, members of the CNRS working at CRM could qualify to
participate to the proposal. But alas this is not the situation of our
friends at LACIM.

\section*{France}

\subsection*{Greater Paris region: LRI (UMR 8623),  LMO (UMR 8628), LIGM
  (UMR 8049), LIP6 (UMR 7606), LIPN (UMR 7030), LUTH (UMR 8102), PRiSM (UMR 8144)}

%\subsection*{Université Paris Sud: LRI (%Laboratoire de Recherche en Informatique,
%  UMR 8623), LMO (%Laboratoire de Mathématiques d'Orsay,
%  UMR 8628)}

\begin{itemize}
\item Nicolas M. Thiéry (Professor, \CS, 80\%): Combinatorial
  representation theory, core dev.
  % Lead developer of Sage-Combinat since 2000%:\\%, regular organizer of Sage events
  %Combinatorial representation theory (monoids, root systems,
  %crystals, ...), random walks, software design, coordination, core
  %development and category infrastructure.
\item Florent Hivert (Professor, \CS, 30\%): Hopf algebras, operads,
  trees, parallelism% Lead developer of Sage-Combinat since 2000%:\\
  %Hopf algebras, operads, parallelism, core development.
\item Nathann Cohen (CR CNRS, section 6, 30\%): Graph theory%Major Sage contributor since 2009%:\\
  %Graph theory.
\item Aladin Virmaux (PhD student, \MATH/\CS, 80\%)\\
  Topics: representation theory of towers of monoids
\item Jean-Baptiste Priez (PhD student, \CS, 80\%)
  % Contributor to Sage since 2011.\\
  % Hopf algebras, parallelism.
% \subsection*{Laboratoire de Mathématique d'Orsay, UMR 8628, Université Paris Sud}
\item Samuel Lelièvre (Assistant Professor, \MATH, 25\%): %, contributor since \TODO{...}:\\
  Translation surfaces%.
  % samuel.lelievre@u-psud.fr trac-sage: slelievre github: slel
  %Status, institute:
  %   Maitre de conferences
  %   Laboratoire de mathematique d'Orsay
  %   UMR 8628 Cnrs / Universite Paris-Sud


%\end{itemize}

%\subsection*{Université Paris Est: LIGM (%Laboratoire d'informatique Gaspard Monge,
%  UMR 8049)}

%\begin{itemize}
\item Nicolas Borie (Ass. professor, \CS, 50\%): Invariant theory,
  exhaustive generation%: Regular contributor since 2008%:\\
  %Algebraic combinatorics, computer algebra, invariant theory,
  %exhaustive generation up to isomorphism
\item Jean-Yves Thibon (Professor, \CS, 20\%):
  %Thème(s) de recherche potentiel: Combinatoire algébrique
  Hopf algebras
%\end{itemize}

%\subsection*{LIP6 (UMR 7606), Université Pierre et Marie Curie, Paris 6}

%\begin{itemize}
\item Genitrini Antoine (Ass. Professor, \CS, 15\%): %\\
  Analytic combinatorics, random generation
\item Marc Mezzarobba (CR CNRS, section 6, 10-20\%): %Regular contributor since 2009%:\\
  % LIP6
  %Numerical evaluation of special functions, (univariate) Ore algebras.
  Special functions, Ore algebras
\item Annick Valibouze (Professor, \CS, 60\%): % entre 50\% et 90\% (ça dépendra de ce que tu retiens):\\
  Invariant theory, %symmetric functions,
  constructive Galois theory
%\end{itemize}

% \subsection*{LIPN, Université Paris 13}

% \begin{itemize}
%\item Cyril Banderier (CR CNRS, section 7, 30\%): Analytic combinatorics, random walks%\\
%   analytic combinatorics, random walks, symbolic computation
%   (asymptotic,  limit laws of rational, algebraic or holonomic functions)
%   Language theory (context-free, \TODO{grammaires d'attributs})
% \end{itemize}

%\subsection*{Université Montpellier 2: LIRMM (UMR 5506)}

%\begin{itemize}
\item Thierry Monteil (CR CNRS, section 41, 30\%): %\\
  %Combinatorics on
  %words,
  Symbolic dynamics, discrete
  geometry%, ...%, tilings%, representation of dynamical systems by
  %Bratteli diagrams.
%\end{itemize}

%\subsection*{Observatoire de Paris, Université Paris Diderot, Meudon: LUTH (%Laboratoire Univers et Théories
%  UMR 8102)}

%\begin{itemize}
\item Eric Gourgoulhon (DR CNRS, section 17, 20\%): %\\
  Differential geometry for physics% general relativity and quantum field theory.
\item Luca De Feo (Ass. Professor, \CS, Université de Versailles - Saint-Quentin)
  % luca, luca, luca.defeo at polytechnique.edu
\end{itemize}


% \subsection*{Montréal: CRM/CNRS (UMI 3457), Québec, Canada}% }
% \begin{itemize}
% \item Franco Saliola (professor, 40\%): Combinatorial representation theory, Hopf algebras%,
%   %, major Sage contributor since 2008%:\\
%   %Posets, words, representation theory, Hopf algebras, Knutson-Tao
%   %puzzles, ...
%   %(posets, words, irreducible representations of the symmetric groups; combinatorial Hopf algebras (QSym/NSym); Knutson-Tao puzzles:\\
%   %Representation theory of finite dimensional algebras, finite monoids, and groups; combinatorial Hopf algebras.\\
%   %Topics in your research that would get impacted: essentially every aspect of my research program since computer exploration plays a central rule in my methodology.
% \item Srecko Brlek (professor, 20\%): Combinatorics on words
% \item Alain Goupil (professor, 20\%): Algebraic and enumerative combinatorics
% \item Alexandre Blondin-Massé (associate professor, 20\%):
%   Words, tilings, discrete geometry
% \end{itemize}

%\vspace{-1.2ex}
\subsection*{Université de Rouen: LITIS (EA 4108)}

%Université de Rouen Avenue de l’Université - BP 8
%76801 Saint-Étienne-du-Rouvray Cedex

\begin{itemize}
\item Jean-Gabriel Luque (professor, \CS, 25\%): Symmetric functions
  et al,
  %and generalizations, %classical
  invariant theory%, hyperdeterminants.

  % Thème(s) de recherche potentiel: Combinatoire des polynômes de Macdonald, Hyperdeterminants et invariants des hypermatrices
  % Thème(s) de développement possible: Développement d'un outil permettant d'engendrer des polynômes de Macdonald et leurs généralisations
  % à partir du graphe de Yang-Baxter.
  % Développer des outils pour la théorie des invariants. Combinatoire de l'hyperdéterminant.
  % Les fonctions génératrices de Dirichlet?
% \item Deneufchâtel Matthieu (ATER, section 27, 30\%)\\
%   Lie algebras, non commutative symmetric functions
% \item Ali Chouria (doctorant 2ème année, section 27, 20\%)\\
%   Combinatorial Hopf algebras, symmetric functions, operads
\item Eric Laugerotte (Ass. professor, \CS, 30\%): %, contributor to \textsc{MuPAD-Combinat}:\\
  Automatons, tree automatons%.
\item Olivier Mallet (Ass. professor, \CS, 30 \%): %\\
  %$k$-shapes, polycubes,
  Combinatorics, Dirichlet series, Hopf algebras
\end{itemize}
% Jean-Paul

\vspace{-1.2ex}
\subsection*{Université Bordeaux I: LaBRI (%Laboratoire Bordelais de Recherche en informatique,
  UMR 5800)}

\begin{itemize}
\item Jean-Christophe Aval (CR CNRS, section 27, 15\%): %\\
  Hopf algebras, enum. combinatorics%: permutations, trees, paths, tableaux
  %Algebraic combinatorics: (quasi-)symmetric functions, Hopf algebras\\
  %Enumerative combinatorics: permutations, trees, paths, tableaux)
\item Adrien Boussicault (Ass. professor, section 27, 30\%): %\\
  Symmetric functions, tree-like tableaux, %random walks on a graph,
  %hyperdeterminants.
\item Vincent Delecroix (CR CNRS, section 41, 20\%): Words, symbolic dynamics
  %random generation % Regular contributor since 2009%: \\
  %Combinatorics on words and symbolic dynamic, random generation,
  %scaling limits.
\end{itemize}

\vspace{-1.2ex}
\subsection*{Université Lyon I: ICJ (%Institut Camille Jordan,
  UMR 5208)}
\begin{itemize}
\item Frédéric Chapoton (CR CNRS, \MATH, 15\%): % Regular contributor since 2003%:\\
  Trees, Species, Operads, Hopf algebras
  %Those are important for my research on tree-indexed series and
  %operads.
\item Philippe Malbos (Ass. professor, \MATH, 15\%): %\\
  Representation theory, %of Monoids,
  rewriting systems%, syzygies.
\end{itemize}

% \subsection*{LIAFA, Université Paris-Diderot}

% \begin{itemize}
% \item Sébastien Labbé (Postdoctorant CRSNG du Canada, 30\%)\\
%   Combinatorics on words and symbolic dynamic, discrete
%   geometry, algorithms on multidimensional continuous fractions
% \end{itemize}

\vspace{-1.2ex}
\subsection*{Université Aix-Marseille 1: LATP (UMR 6632)}
\begin{itemize}
\item Thierry Coulbois (Ass. professor, \MATH, 20\%):
  Geometry, dynamics, and algos on free groups
  %\\
  %Géométrie des groupes, dynamique, problèmes algorithmiques des
  %groupes libres.
\end{itemize}

\subsection*{Other}

\begin{itemize}
\item Julien Puydt (Higher education teacher)
  %I'm willing to help, but that will only be voluntary work in my spare
  %time: I'm not only a teacher in CPGE, I'm also detached to the
  %ministry of defense, so there's no way I can be replaced, take an
  %actual official position.
\end{itemize}
