\section{Actions}


\subsection{Community animation and training}

\begin{itemize}
\item Sage Days (training workshops and developers meetings)
\item Exchange visits for students, researchers, ... in between
  partner institutions, and with developers worldwide
\item Strengthening of collaborations with sister projects: GAP,
  Linbox, Singular, Pari/GP, IPython, Scientific Python, R, to name but a
  few.
\item \TODO{please expand}
\end{itemize}

\subsection{Collaborative tools}

\begin{itemize}
\item Development of a database of notebooks \TODO{Be clearer: we want to get rid of notebooks over the filesystem and .sws tarball}
\item Workflow improvements: continuous integration, ...
\item \TODO{please expand!}
\item foster collaboration with upstream libraries by sharing the development
  and maintenance of the interface (as a standalone Python bindings)
\end{itemize}

%\TODO{lmonade has very similar objectives but uses the gentoo prefix whereas Linux distributions use very different packaging systems:
%\begin{itemize}
%\item gentoo prefix (gentoo)
%\item pacman (arch),
%\item yum (redhat),
%\item apt (debian),
%\item easy\_install
%\item Python index packaging (pip)
%\end{itemize}}

\subsection{Modularization of the Sage distribution}
Separation of the different components of Sage (communication with third-party softwares, build system, Sage native code code). This is a prerequisite for easier packaging and integration in standard Linux distributions and lmonade, native integration within the IPython notebook and other interfaces (larchenv, Spyder, ...) and collaboration with sister projects.

\subsection{Deployment}
\begin{itemize}
\item Better Windows support
\item Live USB keys
\item Creation, deployment, and distribution of preconfigured virtual
machines for Sage as a cloud service, in particular within the
StratusLab infrastructure.
\end{itemize}
 % http://sagedebianlive.metelu.net/

\subsection{Other technical}

\begin{itemize}
\item support for \href{http://www.symbolic-computing.org/}{SCSCP}
\item \TODO{expand}
\end{itemize}

\subsection{Dissemination and teaching}

\begin{itemize}
\item Documentation improvements: overview, cross links, overview of
  recent improvements
\item Thematic tutorials
\item Collections of pedagogical documents\\
  E.g. a complete collection of interactive class notes with computer
  lab projects for the ``Algèbre et Calcul formel'' option of the
  French math aggregation (starting from 2015, only open-source
  systems will be supported, and Sage is a major player).
  % See http://nicolas.thiery.name/Enseignement/Agregation/ as a starter
\item indexation of notebooks and tutorials in a database accessible through a
  web interface
\item \TODO{please expand!}
\end{itemize}

\subsection{Math features}

\begin{itemize}
\item Foundation work: categories, morphisms, coercion, ...
\item Generic parallelism tools for combinatorics
\item Representation theory of monoids and algebras
\item Ore algebras and applications to combinatorics
\item Further development of combinatorial species
\item Automatons and combinatorics on words
  % Interface with Vaucanson, ...
  % https://github.com/coulbois/sage-train-track
  % https://www.lrde.epita.fr/wiki/Vaucanson
  % More support for group theory (especially free group automorphisms, mapping class group of surfaces, braid groups)
\item Symbolic dynamical systems (substitutive systems, Bratelli diagrams,
  sofic shifts, tilings, interval exchange transformations)
\item Manifolds (symbolic computations in charts, differential equations, etc)
  % http://sagemanifolds.obspm.fr/
\item Intersection theory in algebraic geometry
  % http://www.math.sciences.univ-nantes.fr/~sorger/chow_en.html
\item \TODO{please expand!}
\end{itemize}

