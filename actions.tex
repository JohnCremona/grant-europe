\section{Actions}
\subsection{Collaborative tools}

\begin{itemize}
\item Development of an online repository for indexing and sharing
  notebooks and other Sage resources.
  %\TODO{Be clearer: we want to get rid of notebooks over the
  %  filesystem and .sws tarball}
\item Workflow improvements: continuous integration, ...
\item Collaborative interactive tools: e.g. embedding live shared
  \sage sessions (and more generally IPython) in voice-over-IP or
  teleconference calls.
  % Eugen Dedu:
  % I think such a module can be thought of as a screen-capturing
  % module, i.e. allow Ekiga to capture the screen of a Sage user (this
  % is currently not possible).  This is not a difficult task to do.
  % Julien Puydt: ekiga can do that since something like 2008 with my
  % experimental gstreamer plugin, and I shall be able to present
  % interesting sample code to the ekiga-devel mailing-list in something
  % like two-three weeks (after I'm done with my students), which will
  % hopefully be part of the next version.
  %
  % But as Nicolas noted in his answer, some kind of interative session
  % where people can share a sage session would be better.
  %
  % I think the feature decomposes in the following pieces:
  % - IPython should have a way to share sessions between several
  %   participants using an open and standard protocol ;
  % - ekiga should implement it.

  % Eugen Dedu:
  % In my opinion ekiga, because of its dependency on ptlib and opal
  % libraries and the use of complex protocols like SIP and H323, needs
  % highly technical people.  Students cannot help much, but engineers
  % are appropriate.

%\item \TODO{please expand!}
\end{itemize}

%\TODO{lmonade has very similar objectives but uses the gentoo prefix whereas Linux distributions use very different packaging systems:
%\begin{itemize}
%\item gentoo prefix (gentoo)
%\item pacman (arch),
%\item yum (redhat),
%\item apt (debian),
%\item easy\_install
%\item Python index packaging (pip)
%\end{itemize}}

\subsection{Modularization of the Sage distribution}
Separation of the different components of Sage (communication with third-party softwares, build system, Sage native code code). This is a prerequisite for easier packaging and integration in standard Linux distributions and lmonade, native integration within the IPython notebook and other interfaces (larchenv, Spyder, ...) and collaboration with sister projects.

\subsection{Deployment}
\begin{itemize}
\item Better Windows support
\item Live USB keys
\item Creation, deployment, and distribution of preconfigured virtual
machines for Sage as a cloud service, in particular within the
StratusLab infrastructure.
\end{itemize}
 % http://sagedebianlive.metelu.net/

\subsection{Interfaces with other computation systems}

\begin{itemize}
\item Support for \href{http://www.symbolic-computing.org/}{SCSCP}
%\item \TODO{expand}
\item foster collaboration with upstream libraries by sharing the
  development and maintenance of the interfaces (typically as
  standalone Python bindings)
\end{itemize}


\subsection{Dissemination and teaching}

\begin{itemize}
\item Documentation improvements: overview, cross links, overview of
  recent improvements
\item Thematic tutorials
\item Collections of pedagogical documents\\
  E.g. a complete collection of interactive class notes with computer
  lab projects for the ``Algèbre et Calcul formel'' option of the
  French math aggregation (starting from 2015, only open-source
  systems will be supported, and Sage is a major player).
  % See http://nicolas.thiery.name/Enseignement/Agregation/ as a starter
  % Math labs with Sage for first year students in France (L1): http://math.univ-lyon1.fr/~omarguin/
\item Localization of the \sage user interface and key documents in
  various European languages.
\item Distribution of the documents either in the main distribution of
  Sage or through the online repository (see collaborative tools).
\item Massive online introduction course to Sage, drawing on the sage tutorial/notebooks.
Could be "First year \sage course in a box".
\item Taking the opportunity of Python courses to propose \sage as a natural extension
for mathematics; an example is French's 
``Classes pr\'eparatoires''\footnote{
\url{http://en.wikipedia.org/wiki/Classe_préparatoire_aux_grandes_écoles}}, 
where Python has been recently selected as the language to learn programming\footnote{See 
the ``Annexe'' at 
\url{http://www.education.gouv.fr/pid25535/bulletin_officiel.html?cid_bo=71586}}.
%\item \TODO{please expand!}
\end{itemize}

% Jeroen: About teaching: in Gent, Sage is already integrated in the
% courses (maybe you can add this, don't know if it's relevant)
% starting in the first year. It's good for the students because it
% helps in 2 ways: it helps them to understand the mathematics better
% and it helps them to learn basic down-to-earth programming (they
% also have a programming course in Java but that contains a lot of
% theory about complicated class structures)
% Same thing in Orsay
% More python centered but same in UZH

\subsection{Community animation and training}

\begin{itemize}
\item Sage Days (training workshops and developers meetings)
  \begin{itemize}
  \item \TODO{<put suggestions of locations / dates here>}
  \end{itemize}
\item Exchange visits for students, researchers, ... in between
  partner institutions, and with developers worldwide
\item Strengthening of collaborations with sister projects: GAP,
  Linbox, Singular, PARI/GP, IPython, Scientific Python, R, to name but a
  few.
\item \TODO{please expand}
\end{itemize}



\subsection{Math features}

\begin{itemize}
\item Foundation work: categories, morphisms, coercion, ...
\item Generic parallelism tools for combinatorics
\item Representation theory of monoids and algebras
\item Ore algebras and applications to combinatorics
\item Further development of combinatorial species
\item Automatons and combinatorics on words
  % Interface with Vaucanson, ...
  % https://github.com/coulbois/sage-train-track
  % https://www.lrde.epita.fr/wiki/Vaucanson
  % More support for group theory (especially free group automorphisms, mapping class group of surfaces, braid groups)
\item Symbolic dynamical systems (substitutive systems, Bratelli diagrams,
  sofic shifts, tilings, interval exchange transformations)
\item Manifolds (symbolic computations in charts, differential equations, etc)
  % http://sagemanifolds.obspm.fr/
\item Intersection theory in algebraic geometry
  % http://www.math.sciences.univ-nantes.fr/~sorger/chow_en.html
\item Complexes of modules, resolutions.
\item Graded commutative differential algebras and applications to rational homotopy.

\item ...
%\item \TODO{please expand!}
\end{itemize}

\subsection{Parallel Computing}

\begin{itemize}
 \item Transparent integration of Ipython capabilities for cluster computing.
 \item Customization of blas/lapack implementation to take advantage of GPU computing.
 \item Implementation of a transparent abstraction over mpi.
 \item Develop or integrate existing solutions for MapReduce operations over big data.
\end{itemize}
