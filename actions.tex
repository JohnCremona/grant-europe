\section{Actions}

\subsection{Community animation and training}

\begin{itemize}
\item Sage Days (training workshops and developers meetings)
\item Exchange visits for students, researchers, ... in between
  partner institutions, and with developers worldwide
\item Strengthening of collaborations with sister projects: GAP,
  Linbox, Singular, Pari, IPython, Scientific Python, R, to name but a
  few.
\end{itemize}

\subsection{Collaborative tools}

\begin{itemize}
\item Development of a repository of notebooks
\item Workflow improvements: continuous integration, ...
\item \TODO{please expand!}
\end{itemize}

\subsection{Infrastructure}
Try to quantify how many engineers we need and for how long for each item.

\begin{itemize}
\item Modularization of the Sage distribution; Sage as a Python library
\item Packaging and integration in the standard Linux distributions
\item Better Windows support
\item Live USB keys
\item Creation, deployment, and distribution of preconfigured virtual
  machines for Sage as a cloud service, in particular within the
  StratusLab infrastructure.
\item User interface improvements: interactive widgets, robustness
\item Better integration with the IPython notebook
\item Exploration of alternative notebook environments like larchenv,
  Spyder, ...
\item Improved interfaces to other systems
\item Support for \href{http://www.symbolic-computing.org/}{SCSCP}
\item \TODO{please expand!}
\end{itemize}

\subsection{Dissemination and teaching}

\begin{itemize}
\item Documentation improvements: overview, cross links, overview of
  recent improvements
\item Thematic tutorials
\item Collections of pedagogical documents\\
  E.g. a complete collection of interactive class notes with computer
  lab projects for the ``Algèbre et Calcul formel'' option of the
  French math aggregation (starting from 2015, only open-source
  systems will be supported, and Sage is a major player).
  % See http://nicolas.thiery.name/Enseignement/Agregation/ as a starter
\item \TODO{please expand!}
\end{itemize}

\subsection{Math features}

\begin{itemize}
\item Foundation work: categories, morphisms, coercion, ...
\item Generic parallelism tools for combinatorics
\item Representation theory of monoids and algebras
\item Ore algebras and applications to combinatorics
\item Further development of combinatorial species
\item Automatons, languages, combinatorics on words, ...
  % Interface with Vaucanson, ...
\item Dynamical systems
\item Manifolds
  % http://sagemanifolds.obspm.fr/
\item Intersection theory in algebraic geometry
  % http://www.math.sciences.univ-nantes.fr/~sorger/chow_en.html
\item \TODO{please expand!}
\end{itemize}

