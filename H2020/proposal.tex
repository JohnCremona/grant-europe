\documentclass[a4paper,11pt]{article}

\newcommand{\XX}{\textbf{XX}\xspace}
\newcommand{\TheProject}{\XX}

\usepackage{lscape} % for landscape
\usepackage{comments}
% %\usepackage[final]{comments}
\usepackage{verbatim}
\usepackage{listings}
\usepackage{supertabular,array}
\makeatletter
\newcommand\arraybslash{\let\\\@arraycr}
\makeatother
% \setlength\tabcolsep{1mm}
% \renewcommand\arraystretch{1.3}
%% Related Projects
\newcommand{\scienceproject}{\mbox{\textsc{SCIEnce}}\xspace}
\newcommand{\OOMMFNB}{OOMMF-NB\xspace}
\newcommand{\VRE}{VRE\xspace}
\newcommand{\VREs}{VRE\xspace}
\newcommand{\software}[1]{\textsc{#1}\xspace}
\newcommand{\GAP}{\software{GAP}}
\newcommand{\HPCGAP}{\software{HPC-GAP}}
\newcommand{\libGAP}{\software{libGAP}}
\newcommand{\Singular}{\software{Singular}}
\newcommand{\Sage}{\software{Sage}}
\newcommand{\SageCombinat}{\software{Sage-Combinat}}
\newcommand{\MuPADCombinat}{\software{MuPAD-Combinat}}
\newcommand{\Sphinx}{\software{Sphinx}}
\newcommand{\SCSCP}{\software{SCSCP}}
\newcommand{\Python}{\software{Python}}
\newcommand{\IPython}{\software{IPython}}
\newcommand{\Jupyter}{\software{Jupyter}}
\newcommand{\Cython}{\software{Cython}}
\newcommand{\Pythran}{\software{Pythran}}
\newcommand{\Numpy}{\software{Numpy}}
\newcommand{\Pari}{\software{PARI}}
\newcommand{\PariGP}{\software{PARI/GP}}
\newcommand{\libpari}{\software{libpari}}
\newcommand{\GP}{\software{GP}}
\newcommand{\GPtoC}{\software{GP2C}}
\newcommand{\Linbox}{\software{LinBox}}
\newcommand{\LMFDB}{\software{LMFDB}}
\newcommand{\OpenEdX}{\software{OpenEdX}}
\newcommand{\Linux}{\software{Linux}}
\newcommand{\LATEX}{\software{\LaTeX}}
\newcommand{\SMC}{\software{SageMathCloud}}
\newcommand{\Simulagora}{\software{Simulagora}}
\newcommand{\KANT}{\software{KANT}}
\newcommand{\Magma}{\software{Magma}}
\newcommand{\Mathematica}{\software{Mathematica}}
\newcommand{\Maple}{\software{Maple}}
\newcommand{\Matlab}{\software{Matlab}}
\newcommand{\MuPAD}{\software{MuPAD}}
\newcommand{\MPIR}{\software{MPIR}}
\newcommand{\Arxiv}{\software{arXiv}}
\newcommand{\Givaro}{\software{Givaro}}
\newcommand{\fflas}{\software{fflas}}
\newcommand{\MathHub}{\software{MathHub}}
\newcommand{\FindStat}{\software{FindStat}}
\newcommand\DKS{\ensuremath{\mathcal{DKS}}\xspace}

%%% Local Variables: 
%%% mode: latex
%%% TeX-master: "proposal"
%%% End: 

% Partners
\newparticipant{PS}{Université Paris Sud}{UPS}{FR}
\newparticipant{LL}{Logilab}{Logilab}{FR}
\newparticipant{UV}{Université de Versailles Saint-Quentin}{UVSQ}{FR}
\newparticipant{UG}{Université Joseph Fourier}{UJF}{FR}
\newparticipant{UB}{Université Bordeaux}{UB}{FR}
\newparticipant{UO}{University of Oxford}{UO}{UK}
\newparticipant{USH}{Université of Sheffield}{USHEF}{UK}
\newparticipant{USO}{Université of Southhampton}{USO}{UK}
\newparticipant{SA}{University of St Andrews}{USTAN}{UK}
\newparticipant{UW}{University of Warwick}{UW}{UK}
\newparticipant{UK}{University of Kaiserslautern}{UK}{DE}
\newparticipant{US}{University of Silesia}{US}{PL}
\newparticipant{ZH}{Universit\"{a}t Z\"{u}rich}{UZH}{CH}
\newparticipant{SR}{Simula Research Laboratory}{Simula}{NO}
% Participant or third party?
\newparticipant{UWS}{University of Washington at Seattle}{UWS}{US}
% Jean-Pierre Flori
% Luca DeFeo
% Jean-Guillaume Dumas
% Clément Pernet

% Personalised comments for each author
\newcommand{\slcomment}[1]{\comment{SL}{#1}}
\newcommand{\akcomment}[1]{\comment{AK}{#1}}

%% Related Projects
\newcommand{\scienceproject}{\mbox{\textsc{SCIEnce}}}


%%% Local Variables: 
%%% mode: plain-tex
%%% TeX-master: "proposal"
%%% End: 


\begin{document}

\begin{titlepage}

\begin{center}
{\Large \textbf{COVER PAGE}}
\end{center}

\begin{tabular}{llr}
\textbf{Title of Proposal:} & \textbf{\TheProject{}: Collaborative ecosystems for mathematical research and software development} & \\[2ex] % \includeimage[scale=0.5]{logo} \\
\textbf{Date of preparation:} & \textbf{\today} & \comment{}{$
$Revision: 0.0$ $}\\[2ex]
\textbf{List of participants} && \\[2ex]


\end{tabular}

\begin{center}
\begin{tabular}{|l|p{3in}|l|l|}\hline
Participant no & Participant organisation name & Country\\

\hline 
1 (Coordinator) & \longparticipant{1} & \country{1}  \\ \hline
& & \\ \hline
& & \\ \hline
\end{tabular}
\end{center}

\tableofcontents

\end{titlepage}
\newpage

\begin{draft}
\section*{Outline of Project (for Proposers)}

\TODO{This is the place for various READMEs not included in the final submission}

\subsection*{Vision}

An internal attempt at specifying our vision through short
(unsubstantiated) answers.

\begin{verbatim}
> 1) Who are we?

Lead or core developers of some of the major open source components
for pure mathematics and applications:

- Computational components: GAP, Linbox, MPIR, Pari, Sage, Singular
- Databases: LMFDB (findstat as well)
- Knowledge management: MathHub

Together with, in a larger scientific domain, lead developers for:

- Collaborative user interfaces (IPython, SageMathCloud)
- Database and Scientific Computing for the industry (Logilab)
- Numerical code optimization/parallelisation (Pythran)

> 2) What is our goal?

Building blocks with a sustainable development model that can be
seamlessly combined together to build versatile high performance
VRE's, each tailored to a specific need in pure mathematics and
application.

> 2.5) What is our strategy?

Maximize sustainability and impact by reusing and improving existing
building blocks, and reaching toward larger communities whenever possible.
E.g. factoring out our common user interface needs at the level
of IPython/Jupyter will save us time (sustainability), and impact
the larger scientific computing community.
The improvements to the building blocks will impact all their users,
whether they use the VRE or not.

> 3) From where do we start?

- Building blocks with a sustainable development model
- Proof-of-concept prototypes of VRE (SMC, Simulagora)
- Experience on combining together some of the building blocks

> 4) How do we connect or differ from other projects?

The other projects focus on either one or a few of the building
blocks, or on a specific VRE.

We articulate our work with each of them.

> 5) Why are we excellent?

The consortium puts together recognized experts in all
areas and most building blocks that are relevant to the goal. There is
simultaneously a variety of point of views and a record of past
experiences collaborating together at smaller scale
(e.g. GAP-Singular). The approach is bottom up.  Most joint tasks
consist in bringing together people with a common need. There is
experience in community building.  Most participants are
simultaneously users and developers of their tools.

All of this makes me confident that we will indeed be able to
productively collaborate. And do stuff that is first class and useful.

On Sat, Dec 13, 2014 at 11:18:10PM +0100, Wolfram Decker wrote:
> 0) What precisely is our starting point and why are we the right people to
> achieve what we promise to do? Are we leaders in the area touched
> by the proposal? How do we connect? Is there some past
> collaborative success?
> 1) You still do not say what we actually will provide. What precisely will
> the VRE offer to its users?

I more or less answered those points above. Let me know if I should
elaborate.

> Who will be its users? Will those already familiar with the involved
> CAS use it? Will it make the CAS more attractive for a much larger
> community?

One objective is definitely to make CAS and others more attractive by
lowering a lot the entry barrier to access the soft (and db, ...). A
typical situation that most of us ran into is, when collaborating with
other less tech-savvy mathematicians, to have trouble sharing code,
data, and in-the-writing papers with them. SMC was launched with this
idea in mind, and the success proves the concept.

At the same time, the improvements in the building blocks will also
impact CAS users that are happy with their current user interface /
work-flow.

Improvements to IPython will impact a much larger community.

> 2) You motivate what we wish to do by the success of SageMathCloud.
> But why do we than need another VRE? How do we differ from
> SageMathCloud?

There is no one-size-fits-all VRE. One might want to run a VRE on
one's own computer resources for a variety of reason (speed of access,
specific resources, privacy, independence, ...). One might want a
different combination of software (e.g. a lightweight VRE with only
Singular).  One might want to focus on data with LMFDB-style database
searches, or on interactive computing, or on document writing, or some
combination thereof.

> Do we have a chance to compete? Or will we rather join forces? In
> which way?

We join forces (the plan is to have William/UW in the consortium, as
non funded participant). SMC focuses on one specific cloud based
VRE. We focus on the building blocks and the glue. Both project are
mutually beneficial. See the language p. 14 of the proposal.

> 3) You motivate what we wish to do by the success of LMFDB. But what
> are our connections to this database? Will we enhance it? Will we connect
> it to other stuff we do? Will we create other databases?

LMFDB is a prototype of large scale database. We want to make it
easier for other groups of mathematicians to setup similar databases
in their area. Reciprocally, like SMC, the LMFDB with benefit back
from the improved building blocks.

> 4) Why is Europe in the lead if there is already SageMathCloud?
> Where precisely is Europe in the lead?

Europe is the lead in many of the building blocks.
\end{verbatim}

% \subsection*{Mission statement for the grant}

% Our mission is to promote the next generation of community-developed
% open source software, databases, and services adapted to the needs of
% collaborative research in pure mathematics and applications.

% Our research will cover a wide variety of aspects, ranging from
% software development models, user interfaces \TODO{virtual
%   environments?}, deployment frameworks and novel collaborative tools,
% component architecture, design, and standardization of software
% \TODO{system?} and databases, to links to publication, data archival
% and reproducibility of experiments, development models and tools, and
% social aspects.

% It will consolidate Europe's leading position in computational
% mathematics and build on the remarkable success of the ecosystem of
% projects GAP, Python/Sage, Pari, Singular, LMFDB.

\subsection*{Description of the call}

\verbatiminput{call_description}

% \TODO{What do we mean by ``new generation''}.

\renewcommand{\thepage}{\arabic{page}}
\setcounter{page}{1}
\black
\cleardoublepage
\end{draft}

%%% Local Variables: 
%%% mode: latex
%%% TeX-master: "proposal"
%%% End: 

%  LocalWords:  verbatiminput renewcommand thepage setcounter cleardoublepage


% ---------------------------------------------------------------------------
%  Section 1: Excellence
% ---------------------------------------------------------------------------

\section{Excellence}

\subsection{Objectives}
\label{sect:objectives}


From their early days, computers have been used in pure mathematics,
either to prove theorems or, like the telescope for astronomers, to
explore new theories. Major achievements include the proof of the four
color theorem or ... Usage has grown to the point that certain areas
of mathematics now completely depend on experimental methods.



European mathematicians have been the early pioneers in the area, and
have grown a long tradition of collaborative open source software
development with systems like GAP, Singular, or Pari/GP playing a
major role for decades.

This project gathers developers of the leading mathematical software
in Europe and of key components, together with researchers in social
sciences, with mission to promote a new generation of
community-developed open source software, databases, and services
adapted to the needs of collaborative research in pure mathematics and
applications.

Our research will cover a wide variety of aspects, ranging from
software development models, user interfaces \TODO{virtual
  environments?}  deployment frameworks and novel collaborative tools,
component architecture, design, and standardization of software
\TODO{system?} and databases, to links to publication, data archival
and reproducibility of experiments, development models and tools, and
social aspects.

It will consolidate Europe's leading position in computational
mathematics and build on the remarkable success of the dynamic
Python/Sage ecosystem and its sister European projects (GAP, Pari,
Singular, LMFDB, ...).


\eucommentary{\emph{Describe the specific objectives for the project, 
which should be clear, measurable, realistic and achievable within the 
duration of the project. Objectives should be consistent with the expected 
exploitation and impact of the project (see section 2).}}

\TOWRITE{ALL}{This is an example of using TOWRITE command}

\TODO{This is an example of using TODO command}



\draftpage

\subsection{Relation to the Work Programme}

\eucommentary{
Indicate the work programme topic to which your proposal relates, and 
explain how your proposal addresses the specific challenge and scope 
of that topic, as set out in the work programme.}

\draftpage

\subsection{Concept and Approach}

\eucommentary{
-- Describe and explain the overall concept underpinning the project. 
Describe the main ideas, models or assumptions involved. Identify 
any trans-disciplinary considerations;\\
-- Describe any national or international research and innovation activities which will be
linked with the project, especially where the outputs from these will feed into the
project;\\
-- Describe and explain the overall approach and methodology, distinguishing, as
appropriate, activities indicated in the relevant section of the work programme, e.g.
Networking Activities, Service Activities and Joint Research Activities, as detailed in
the Part E of the Specific features for Research Infrastructures of the Horizon 2020
European Research Infrastructures (including e-Infrastructures) Work Programme 2014-
2015;\\
-- Describe how the Networking Activities will foster a culture of co-operation between the
participants and other relevant stakeholders.\\
-- Describe how the Service activities will offer access to state-of-the-art infrastructures,
high quality services, and will enable users to conduct excellent research.\\
-- Describe how the Joint Research Activities will contribute to quantitative and qualitative
improvements of the services provided by the infrastructures.\\
-- As per Part E of the Work Programme, where relevant, describe how the project will
share and use existing basic operations services (e.g. authorisation and accounting
systems, service registry, etc.) with other e-infrastructure providers and justify why such
services should be (re)developed if they already exist in other e-infrastructures. Describe
how the developed services will be discoverable on-line.\\
-- Where relevant, describe how sex and/or gender analysis is taken into account in the
project�s content.}

\draftpage

\subsection{Ambition}

\eucommentary{-- Describe the advance your proposal would provide beyond the 
state-of-the-art, and the extent the proposed work is ambitious. Your answer 
could refer to the ground-breaking nature of the objectives, concepts 
involved, issues and problems to be addressed, and approaches and methods to be used.\\
-- Describe the innovation potential which the proposal represents. Where relevant, refer to
products and services already available, e.g. in existing e-Infrastructures.}

\draftpage

% ---------------------------------------------------------------------------
%  Section 2: Impact
% ---------------------------------------------------------------------------

\section{Impact}
\label{sec:impact}

\subsection{Expected Impacts}

\eucommentary{Please be specific, and provide only information that applies 
to the proposal and its objectives. Wherever possible, use quantified 
indicators and targets.\\
Describe how your project will contribute to:\\
-- the expected impacts set out in the work programme, under the relevant topic
(including key performance indicators/metrics for monitoring results and impacts);\\
-- improving innovation capacity and the integration of new knowledge 
(strengthening the competitiveness and growth of companies by developing 
innovations meeting the needs of European and global markets; and, where 
relevant, by delivering such innovations to the markets;\\
-- any other environmental and socially important impacts (if not already 
covered above).\\
Describe any barriers/obstacles, and any framework conditions (such as 
regulation and standards), that may determine whether and to what extent 
the expected impacts will be achieved. (This should not include any risk 
factors concerning implementation, as covered in section 3.2.)}

\draftpage

\subsection{Measures to Maximise Impact}

\subsubsection{Dissemination and Exploitation of Results}
\label{subsubsect:dissemination}

\eucommentary{-- Provide a draft 'plan for the dissemination and exploitation 
of the project's results'. The plan, which should be proportionate to the 
scale of the project, should contain measures to be implemented both during 
and after the project.\\
Dissemination and exploitation measures should address the full range 
of potential users and uses including research, commercial, investment, 
social, environmental, policy making, setting standards, skills and 
educational training.\\
The approach to innovation should be as comprehensive as possible, 
and must be tailored to the specific technical, market and organisational 
issues to be addressed\\
-- Explain how the proposed measures will help to achieve the expected impact of the
project . Provide a draft business plan for financial sustainability as stated in the Part
E of the Specific features for Research Infrastructures of the Horizon 2020 European
Research Infrastructures (including e-Infrastructures) Work Programme 2014-2015.\\
-- Where relevant, include information on how the participants will 
manage the research data generated and/or collected during the 
project, in particular addressing the following issues: 
What types of data will the project generate/collect? What 
standards will be used? How will this data be exploited and/or 
shared/made accessible for verification and re-use (If data cannot 
be made available, explain why)? How will this data be curated and preserved?\\ \\
-- Include information about any open source software used or developed by the
project.\\
You will need an appropriate consortium agreement to manage (amongst other things) 
the ownership and access to key knowledge (IPR, data etc.). Where relevant, 
these will allow you, collectively and individually, to pursue market opportunities
arising from the project's results.\\
The appropriate structure of the consortium to support exploitation is addressed 
in section 3.3. \\ \\
-- Outline the strategy for knowledge management and protection. Include measures to
provide open access (free on-line access, such as the �green� or �gold� model) to
peer-reviewed scientific publications which might result from the project.\\
Open access publishing (also called 'gold' open access) means that an article is
immediately provided in open access mode by the scientific publisher. The associated costs
are usually shifted away from readers, and instead (for example) to the university or
research institute to which the researcher is affiliated, or to the funding agency supporting
the research.\\
Self-archiving (also called 'green' open access) means that the published article or the
final peer-reviewed manuscript is archived by the researcher - or a representative - in an
online repository before, after or alongside its publication. Access to this article is often -
but not necessarily - delayed (�embargo period�), as some scientific publishers may wish to
recoup their investment by selling subscriptions and charging pay-per-download/view fees
during an exclusivity period.}

\draftpage

\subsubsection{Communication activities}
\label{subsubsect:communication}

\eucommentary{Describe the proposed communication measures for promoting the 
project and its findings during the period of the grant. Where appropriate 
these measures should include social media and public events with user 
participation. Measures should be proportionate to the scale of the project, 
with clear objectives. They should be tailored to the needs of various audiences, 
including groups beyond the project's own community. Where relevant, include 
measures for public/societal engagement on issues related to the project.}

\clearpage

% ---------------------------------------------------------------------------
%  Section 3: Implementation
% ---------------------------------------------------------------------------



\section{Implementation}

\subsection{Work Plan --- Work packages, deliverables and milestones}
\label{sect:workplan}

\eucommentary{Please provide the following:\\
\begin{itemize}
\item
brief presentation of the overall structure of the work plan;
\item
timing of the different work packages and their components (Gantt chart or similar);
\item
detailed work description, i.e.:
\begin{itemize}
\item
a description of each work package (table 3.1a);
\item
a list of work packages (table 3.1b);
\item
a list of major deliverables (table 3.1c);
\end{itemize}
\item
graphical presentation of the components showing how they inter-relate (Pert chart or similar).
\end{itemize}
}

\subsubsection*{Overall Structure of the Work Plan}

The work plan is broken down into XX workpackages as shown
in Figure~\ref{}: WP2 deals with  ...
In addition, there is one management work package (WP1) and one
general dissemination work package (\ref{m}). The Gantt chart on
Page~\pageref{fig:gantt} illustrates the timeline for the
various tasks for these work packages, including inter-task
dependencies.

%\newpage
\subsubsection*{How the Work Packages will Achieve the Project Objectives}
\label{sssec:how_the_work_packages_will_achieve}

\TOWRITE{ALL}{This needs to explain that we're actually going to meet the 
objectives.  Needs to be done after objectives and WPs.}

The project objectives (Section~\ref{sect:objectives},
page~\pageref{sect:objectives}) and the corresponding work
packages that contribute to achieving those objectives are:

\begin{center}
\begin{tabular}{|l|l|l|}\hline
\textbf{Objective} & \textbf{Purpose} & \textbf{WPs} \\\hline \hline
Objective 1 & XX & \textbf{WPX} \\\hline
\end{tabular}
\end{center}

\paragraph*{Work Programme for Objective 1: }

Objective 1 is covered by WPX, which will ...

\landscape

\subsubsection*{Work Plan Timing: GANTT Chart showing Task Dependencies and Information Flows}


\vspace{-0.7in} \centerline{\hbox to \columnwidth{\hss%\includeimage[scale=0.85,angle=270]{ParaPhrase-Gantt2.pdf}
\hss}}
\label{fig:gantt}
\vspace{-1in} % Fool LaTeX into avoiding unnecessary page break
\endlandscape

\newpage

%\input{deliverables-dates}
%% Deliverables list.
%% Deliverables ordered by Workpackage
%% Workpackages are numbered automatically in sequence - the WP number has no effect

\workpackage{1}{Project Management}
\deliverable{mgt:mailinglists}
\deliverable{mgt:projectwebsite}
\deliverable{mgt:swrepository}
\deliverable{mgt:periodic-rep-1}
\deliverable{mgt:periodic-rep-2}
\deliverable{mgt:periodic-rep-3}
\deliverable{mgt:periodic-rep-4}
\deliverable{mgt:final-mgt-rep}

\workpackage{2}{}
\deliverable{del:xx}

\workpackage{3}{}
\deliverable{del:xx}

\workpackage{4}{}
\deliverable{del:xx}

\workpackage{5}{}
\deliverable{del:xx}

\workpackage{6}{}
\deliverable{del:xx}

\workpackage{7}{}
\deliverable{del:xx}

\workpackage{8}{}
\deliverable{del:xx}

\workpackage{9}{Dissemination, Exploitation and Communication}
\deliverable{del:pressrelease} % Press release.
\deliverable{del:website} % Project presentation (web site). 
\deliverable{del:workshop1}  % Report on first project workshop, year 1. 
\deliverable{del:dissemplan1} % Final plan for using and disseminating knowledge.
\deliverable{del:workshop2}  % Report on second project workshop, year 2
\deliverable{del:workshop3}  % Report on third project workshop, year 3
\deliverable{del:dissemplan2} % Final plan for using and disseminating knowledge.



\addtocounter{subsubsection}{1}
\addcontentsline{toc}{subsubsection}{\protect\numberline{\thesubsubsection}Work
Package List}
\fbox{\begin{minipage}{\textwidth}\begin{center}{\Large\bf
        Work package list} % (full duration of project)}
  \end{center}
  \end{minipage}}

\bigskip\bigskip

\begin{tabular}{|p{1.2cm}|p{9.15cm}|p{0.8cm}|p{1.2cm}|p{1cm}|p{0.9cm}|p{0.9cm}|}
\hline
{\bf Work \mbox{package} No} & {\bf Work package title} &
{\bf Lead \mbox{partic.} no.} &
{\bf Lead short name} &
{\bf Person months} & {\bf Start month} & {\bf End month} \\\hline 

\newcounter{wp}

\addtocounter{wp}{1}
\workpackageentry{\thewp}{SA}{}{1}{60}

\addtocounter{wp}{1}
\workpackageentry{\thewp}{}{}{}{}

\addtocounter{wp}{1}
\workpackageentry{\thewp}{}{}{}{}

\addtocounter{wp}{1}
\workpackageentry{\thewp}{}{}{}{}

\addtocounter{wp}{1}
\workpackageentry{\thewp}{}{}{}{}

\addtocounter{wp}{1}
\workpackageentry{\thewp}{}{}{}{}

\addtocounter{wp}{1}
\workpackageentry{\thewp}{}{}{}{}

\addtocounter{wp}{1}
\workpackageentry{\thewp}{}{}{}{}

\addtocounter{wp}{1}
\workpackageentry{\thewp}{SA}{}{}{}

{\textbf{Total}} & & & &
\textbf{\large XXX}&
&
\\\hline
\end{tabular}


% \textbf{Summary:}\\[1ex]

\newpage

\fbox{\begin{minipage}{\textwidth}\begin{center}\Large\bf List of Deliverables
  \end{center}
  \end{minipage}}

\label{sect:deliverables}

\bigskip\bigskip\bigskip

\begin{minipage}{\textwidth}
\begin{center}
\begin{tabular}{|p{0.8cm}|p{8.75cm}|p{0.8cm}|p{1.2cm}|p{1.2cm}|p{1.2cm}|p{1.2cm}|}  \hline
\textbf{Del. no.}              & \textbf{Deliverable name}        & \textbf{WP no.} & \textbf{Lead}
& \textbf{Type}              & \textbf{Dissemi- nation level}   & \textbf{Delivery date}
\\ \hline

%% Year 1

\ref{del:xx}  & Requirements Analysis                            
& WP? & & R & CO &  ?? \\
\hline
\end{tabular}
\end{center}
\end{minipage}


\newpage

%% Set up the milestone numbers.

The work in the \TheProject project is structured by four milestones,
which coincide with the four project meetings held at the end of each
year of the project (the other four meetings will be held in the middle
of each year). Given the nature of the project, with a
large number of essentially independent tasks, there is no need for
milestones attached to specific collections of tasks or
deliverables. Instead, given that the meetings are the main
face-to-face interaction points in the project, it's suitable to
schedule the milestones for these events, where they can be discussed
in detail, tracking the progress in each work package through status
reports on the tasks and deliverables.

We envisage that this setup will give the project the vital coherence
in spite of the broad interdisciplinary mix of various backgrounds of the
participants.

% \newcommand{\WPall}{\WPref{management}, \WPref{dissem}, \WPref{component-architecture}, \WPref{UI}, \WPref{hpc}, \WPref{dksbases}, \WPref{social-aspects}}

% \newcommand{\WPnoUI}{\WPref{management}, \WPref{dissem}, \WPref{component-architecture}, \WPref{hpc}, \WPref{dksbases}, \WPref{social-aspects}}

% \begin{center}
%   \begin{tabular}{|m{.05\textwidth}|m{.30\textwidth}|m{.15\textwidth}|m{.05\textwidth}|m{.22\textwidth}|}
%     \hline
%     Mile-stone nr. & Milestone name & Related work packages & Est. date & Means of verification \\\hline
%     M1 & Requirements study, design and prototype implementations. Start of
%          community building.
%        & \WPall 
%        & 12 
%        & 2nd Project meeting report. Completion of corresponding deliverables. \\\hline
%     M2 & First fully functional interface implementations.
%          Enhanced versions of \TheProject components.
%          Training early adopters.
%        & \WPall 
%        & 24 
%        & 4th Project meeting report. Completion of corresponding deliverables. \\\hline
%     M3 & Evaluating \TheProject software. Working with the community 
%          and building portfolio of experiments produced with \TheProject.
%        & \WPall 
%        & 36 
%        & 6th Project meeting report. Completion of corresponding deliverables. \\\hline
%     M4 & Project evaluation and final versions of all \TheProject components.
%        & \WPnoUI 
%        & 48 
%        & 8th Project meeting report. Completion of corresponding deliverables. \\\hline
%   \end{tabular}
% \end{center}

\begin{milestones}
  \milestone[id=startup,month=12,
  verif={Completed all corresponding deliverables and reported the progress in the 2nd Project meeting report.}]
  {Startup}
  {By Milestone 1 we will have carried out the requirements study, design and prototype implementations and started community building activities.}

  \milestone[id=proto1,month=24,
  verif={Completed all corresponding deliverables and reported the progress in the 4th Project meeting report.}]
  {Prototypes}
  {By Milestone 2 we will have constructed first fully functional interface implementations and released enhanced versions of \TheProject components, and train early adopters of \TheProject.}

  \milestone[id=community,month=36,
  verif={Completed all corresponding deliverables and reported the progress in the 6th Project meeting report.}]
  {Community/ Experiments}
  {By Milestone 3 we will have gathered and evaluated feedback on \TheProject software and established the portfolio of experiments produced with \TheProject through further engaging with the community.}

  \milestone[id=eval,month=48,
  verif={Completed all corresponding deliverables and reported the progress in the 8th Project meeting report.}]
  {Evaluation}
  {By Milestone 4 we will have released final versions of all \TheProject components and completed the project evaluation.}
\end{milestones}

%%% Local Variables:
%%% mode: latex
%%% TeX-master: "proposal"
%%% End:

%  LocalWords:  verif ldots


\fbox{\begin{minipage}{\textwidth}\begin{center}\Large\bf List of milestones
  \end{center}
  \end{minipage}}
\label{sect:milestones}

\bigskip\bigskip\bigskip

\begin{minipage}{\textwidth}
\begin{center}
\begin{tabular*}{\textwidth}{|p{1.5cm}|p{6.7cm}|p{2.5cm}|p{1.5cm}|p{3.6cm}|}  \hline
\textbf{Milestone number} & \textbf{Milestone name} & \textbf{Related work
  package(s)} & \textbf{Estimated date} & \textbf{Means of
  verification} (deliverables shown here + success criteria below) \\
\hline
\ref{mil:initial} &
  Completed initial requirements analysis.  &
  WPX &
  1 &
\ref{del:requirements-analysis}.
\\
\ref{mil:final} &
&
WPX &
&
\\
\hline
\end{tabular*}
\end{center}
\end{minipage}

\vspace{10pt}
\begin{center}
\begin{tabular*}{\textwidth}{|p{1.5cm}|p{13.3cm}|p{1.9cm}|}\hline
\textbf{Milestone} & \textbf{Success Criteria} & \textbf{Contributes to
  Objective(s)} \\\hline
\ref{mil:initial} &
Completed requirements analysis (Deliverable~\ref{del:requirements-analysis}). &
 \textbf{1, 3.}
\\
\ref{mil:final} & 
XX
& \textbf{XX}
\\\hline
\end{tabular*}
\end{center}


% ---------------------------------------------------------------------------
% Include Workpackage descriptions
% ---------------------------------------------------------------------------

%% WP titles and order are defined in deliverables.tex
%%% Workpackage style may be broken -- fix this!!

%% Local WP number counter - should possibly be global and hidden?

\newcounter{wpno}

\addtocounter{wpno}{1}

\begin{Workpackage}{\thewpno}
\WPTitle{\wpname{\thewpno}}
\WPStart{Month 1}
\WPParticipant{SA}{60}

\begin{WPObjectives}
The objectives of \theWP{} are to undertake all project management activities, including ...

\end{WPObjectives}

\begin{WPDescription}
This workpackage will perform ...
\end{WPDescription}

\begin{WPDeliverables}
\begin{itemize}
\item
\ref{mgt:mailinglists}
(Month 1): 
Internal and external mailing lists.
\item
\ref{mgt:swrepository}
(Month 1): 
Internal software repository.
\item
\ref{mgt:periodic-rep-1}
(Month 12): 
Project Periodic Report (first year).
\item
\ref{mgt:periodic-rep-2}
 (Month 24): 
Project Periodic Report (second year).
\item
\ref{mgt:periodic-rep-3}
(Month 36): 
Project Periodic Report (third year).
\item
\ref{mgt:periodic-rep-4}
(Month 48): 
Project Periodic Report (fourth year).
\item
\ref{mgt:final-mgt-rep}
(Month 36): 
Project Final Report
\end{itemize}
\end{WPDeliverables}
\end{Workpackage}

\addtocounter{wpno}{1}
\begin{Workpackage}{\thewpno}
\wplabel{wp:x}
\WPTitle{\wpname{\thewpno}}
\WPStart{Month 1}
\WPParticipant{SA}{1}

\begin{WPObjectives}
The objectives of \theWP{} are to:
\begin{itemize}
\item
\item
\item
\item
\item
\end{itemize}
\end{WPObjectives}

\begin{WPDescription}
This workpackage  ...
\end{WPDescription}

\begin{WPDeliverables}
\begin{itemize}
\item
\ref{del:x}
(Month X): 
X.
\end{itemize}
\end{WPDeliverables}
\end{Workpackage}

\addtocounter{wpno}{1}
\begin{Workpackage}{\thewpno}
\wplabel{wp:x}
\WPTitle{\wpname{\thewpno}}
\WPStart{Month 1}
\WPParticipant{SA}{1}

\begin{WPObjectives}
The objectives of \theWP{} are to:
\begin{itemize}
\item
\item
\item
\item
\item
\end{itemize}
\end{WPObjectives}

\begin{WPDescription}
This workpackage  ...
\end{WPDescription}

\begin{WPDeliverables}
\begin{itemize}
\item
\ref{del:x}
(Month X): 
X.
\end{itemize}
\end{WPDeliverables}
\end{Workpackage}

\endinput

\input{WPs/WP4}
\input{WPs/WP5}
\input{WPs/WP6}
\input{WPs/WP7}
\input{WPs/WP8}
\input{WPs/WP9}
\input{WPs/WP10}
\input{WPs/WP11}
\input{WPs/WP12}
\input{WPs/WP13}
\input{WPs/WP14}
\input{WPs/WP15}


\TODO{Milestones need to be discussed and then described here.}

\newpage

\TODO{Check this for any necessary changes.}


\subsection{Management Structure and Procedures}
\label{sect:mgt}

\eucommentary{Describe the organisational structure and the decision-making 
(including a list of milestones (table 3.2a)).\\
Explain why the organisational structure and decision-making mechanisms are
 appropriate to the complexity and scale of the project.\\
Describe, where relevant, how effective innovation management will be 
addressed in the management structure and work plan.\\
Describe any critical risks, relating to project implementation, that 
the stated project's objectives may not be achieved. Detail any risk 
mitigation measures. Please provide a table with critical risks 
identified and mitigating actions (table 3.2b).}

\draftpage
\subsection{Consortium as a Whole}
\eucommentary{\begin{itemize}
\item
Describe the consortium. How will it match the project's objectives? 
How do the members complement one another (and cover the value chain, 
where appropriate)? In what way does each of them contribute to the 
project? How will they be able to work effectively together?
\item
If applicable, describe the industrial/commercial involvement in the 
project to ensure exploitation of the results and explain why this is 
consistent with and will help to achieve the specific measures which 
are proposed for exploitation of the results of the project (see section 2.3).
\item
Other countries: If one or more of the participants requesting EU funding 
is based in a country that is not automatically eligible for such funding 
(entities from Member States of the EU, from Associated Countries and 
from one of the countries in the exhaustive list included in General 
Annex A of the work programme are automatically eligible for EU funding),
 explain why the participation of the entity in question is essential to carrying out the project
\end{itemize}
}

\draftpage

\subsection{Resources to be Committed}

\eucommentary{Please provide the following:
\begin{itemize}
\item
a table showing number of person/months required (table 3.4a)
\item
a table showing 'other direct costs' (table 3.4b) for participants where 
those costs exceed 15\% of the personnel costs (according to the budget 
table in section 3 of the administrative proposal forms)
\end{itemize}}

\TOWRITE{NT}{Rework this section}

\subsubsection{Travel, dissemination and outreach}
\label{sect:budget-details-travel}

This community building nature of this grant proposal requires a large
number of staff exchanges, work shops with project partners and
workshops engaging the wider community in addition to the usual
management and project review meetings. For dissemination, we need to
target the computer science and focussed community and their conferences
as well as the domains benefitting from \TheProject, such as 
Mathematics and Science.

\subsubsection{Travel, dissemination}

We use the following guidelines for expected travel expenses:
\euro{2200} for attendance of a typical one week international
conference outside Europe (including travel, subsistence,
accommodation and registration), \euro{1200} for a corresponding
conference in Europe, \euro{750} for a one-week visit to project
partners, for example for coding sprints and one-to-one visits for
particular research. We expect a similar cost per week when hosting
visitors. For the 6-monthly project meetings, we expect on average a
cost of \euro{400} for travel, accommodation and subsistence.

For a partner site with one investigator and one full time researcher,
we expect that that both will attend all of the 9 project meetings that take 
place every 6 months (cost of 9 * 2 * 400 =
\euro{7200}), and that the site spends \euro{2000} per year to host
external visitors contributing to the project (total \euro{8000}). We
expect the investigator and research together to do 4 one-week visits
to other sites (each at \euro{750}, totals \euro{12000}). 


For dissemination, we expect the researcher to attend on average 1
international conference and 2 European meetings per year (totals
\euro{18400}) and the investigator to attend one international and one
european gathering (totals \euro{13600}).

Where there are multiple investigators per site, they will share the
travel and associated costs outlined above. Where there are multiple
researchers, or researchers not employed for the full 48 months, the
travel budget is reduced accordingly.

\subsubsection{Outreach costs}

We request funds for outreach activities such as workshops that
facilitate community building, disseminate best practice and
encourages sustained contributions of the community to the project and
beyond the lifetime of the funding. Details are given in the tables
below and in the work packages. 

\TODO{Give details in work packages on what workshops are planned} 

\TODO{List workshop expenses and travel expenses in tables below}.


\eucommentary{Will get help from Orsay's grant services}

\eucommentary{Please provide the following:
\begin{itemize}
\item
a table showing number of person/months required (table 3.4a)
\item
a table showing 'other direct costs' (table 3.4b) for participants where
those costs exceed 15\% of the personnel costs (according to the budget
table in section 3 of the administrative proposal forms)
\end{itemize}}

%\newpage

%\landscape

\fbox{\begin{minipage}{\textwidth}

\begin{center}\Large\bf
Summary of staff effort
\end{center}
\end{minipage}}

\eucommentary{Please indicate the number of person/months over the whole 
duration of the planned work, for each work package, for each participant. 
Identify the work-package leader for each WP by showing the relevant 
person-month figure in bold.}

\bigskip

%--------------

\subsection{Resources Southampton}

Southampton requests 32 person months for a researcher (expected to start in month 4 of the project), 6 person months for the lead PI (Hans Fangohr) and 2 person months for the co investigator (Ian Hawke). The lead PI will take on all management responsibilities. The researcher will not be employed for the whole project duration, and the PIs will carry out all tasks for the project in the remaining period. 

\bigskip
\begin{table}[h!]
\begin{tabular}{|r|r|p{9cm}|}
\hline
\textbf{} & \textbf{Cost (\euro)} & \textbf{Justification} \\\hline
\textbf{Travel} & 51,500& Travel as outlined in \ref{sect:budget-details-travel}\\\hline
\textbf{Equipment} & 4,000 & 2 High Performance Laptops\\\hline
\textbf{Equipment} & 6,000 & HPC Workstation to host \OOMMFNB webserver, \taskref{UI}{oommf-nb-ve}\\\hline
\textbf{Other goods and services} & 19,600 & 4 Dissemination workshops (travel \& room hire), \taskref{dissem}{dissemination-of-oommf-nb-workshops}\\\hline
\textbf{Total} & 81,100\\\cline{1-2}
\end{tabular}
\caption{Overview: Non-staff resources to be committed at Southampton (all in \texteuro)}\label{tab:resources-non-staff-southampton}\vspace*{-1em}
\end{table}


%--------------


\subsection{Resources partner X}
\subsection{Resources partner Y}
\subsection{\ldots}










\subsection{Resources partner Summary}

\begin{table}[ht]\centering
  \TODO{Table 3.4.a: insert here table from Figure 3, and transpose; see
    Table 3.4.a in the word template}
  \TODO{The work package leader will usually have the largest investment}
\caption{Overview: Resources to be committed (all in \texteuro)}\label{tab:resources}\vspace*{-1em}
\end{table}
\fbox{\begin{minipage}{\textwidth}

\eucommentary{Please complete the table below for each participant if the sum of the costs for’ travel’, ‘equipment’,
and ‘goods and services’ exceeds 15% of the personnel costs for that participant (according to the
budget table in section 3 of the proposal administrative forms).}

\begin{center}\Large\bf
Other direct cost items
\end{center}
\end{minipage}}

\bigskip

\begin{tabular}{|r|l|p{9cm}|}
\hline
\textbf{} & \textbf{Cost (\euro)} & \textbf{Justification} \\\hline
\textbf{Travel} & & \\\hline
\textbf{Equipment} & & \\\hline
\textbf{Other goods and services} & & \\\hline
\textbf{Total} & \\\cline{1-2}
\end{tabular}




%%% Local Variables:
%%% mode: latex
%%% TeX-master: "proposal"
%%% End:

%  LocalWords:  newpage fbox minipage textwidth eucommentary bigskip rr rr rr hline
%  LocalWords:  vspace texteuro textbf textbf textbf


\subsubsection*{Management Level Description of Resources and Budget}

\TODO{This needs to be updated in line with the rest of the
project.}

The project will employ XX person-months of effort over YY years, 
comprising ...


% ---------------------------------------------------------------------------
%  Section 4: Members of the Consortium
% ---------------------------------------------------------------------------

\newpage

\eucommentary{This section is not covered by the page limit.\\ 
The information provided here will be used to judge the operational capacity.}

\section{Members of the Consortium}

\subsection{Participants}

\eucommentary{Please provide, for each participant, the following (if available):\\
\begin{itemize}
\item
a description of the legal entity and its main tasks, 
with an explanation of how its profile matches the tasks in the proposal;
\item
a curriculum vitae or description of the profile of the persons, 
including their gender, who will be primarily responsible for carrying 
out the proposed research and/or innovation activities;
\item
a list of up to 5 relevant publications, and/or products, services 
(including widely-used datasets or software), or other achievements 
relevant to the call content;
\item
a list of up to 5 relevant previous projects or activities, connected 
to the subject of this proposal;
\item
a description of any significant infrastructure and/or any major items 
of technical equipment, relevant to the proposed work;
\item
any other supporting documents specified in the work programme for this call.
\end{itemize}}

\cite{science-project}

\subsection{Third Parties Involved in the Project (including use of third party resources)}

\eucommentary{Please complete, for each participant, the table
(see page 27 of "VRETemplate.PDF"), 
or simply state "No third parties involved", if applicable.}

No third parties involved.


% ---------------------------------------------------------------------------
%  Section 5: Ethics and Security
% ---------------------------------------------------------------------------

\newpage

\section{Ethics and Security}

\eucommentary{This section is not covered by the page limit.}

\subsection{Ethics}

\eucommentary{
If you have entered any ethics issues in the ethical issue table in the administrative proposal forms, you must:\\
$\bullet$ submit an ethics self-assessment, which: \\
-- describes how the proposal meets the national legal and ethical requirements of the
country or countries where the tasks raising ethical issues are to be carried out;\\
-- explains in detail how you intend to address the issues in the ethical issues table, in
particular as regards: 
research objectives (e.g. study of vulnerable populations, dual use, etc.), 
research methodology (e.g. clinical trials, involvement of children and related
consent procedures, protection of any data collected, etc.), 
the potential impact of the research (e.g. dual use issues, environmental damage,
stigmatisation of particular social groups, political or financial retaliation,
benefit-sharing, malevolent use , etc.)\\
$\bullet$ provide the documents that you need under national law(if you already have them), e.g.:\\
-- an ethics committee opinion;\\
-- the document notifying activities raising ethical issues or authorising such activities\\
If these documents are not in English, you must also submit an English summary of them
(containing, if available, the conclusions of the committee or authority concerned).\\
If you plan to request these documents specifically for the project 
you are proposing, your request must contain an explicit reference to the project title}

\subsection{Security}

Please indicate if your proposal will involve:

\begin{itemize}
\item
activities or results raising security issues: NO
\item
'EU-classified information' as background or results: NO
\end{itemize}

\newpage

\label{bibliography}
\addcontentsline{toc}{section}{References}

\bibliographystyle{abbrv}
\bibliography{bibliography}

\end{document}
