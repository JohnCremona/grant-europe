% ---------------------------------------------------------------------------
%  Section 2: Impact
% ---------------------------------------------------------------------------

\section{Impact}
\label{sec:impact}

\TODO{Orsay's grant services will help here in December}

\subsection{Expected Impacts}

\eucommentary{Please be specific, and provide only information that applies
to the proposal and its objectives. Wherever possible, use quantified
indicators and targets.\\
Describe how your project will contribute to:\\
-- the expected impacts set out in the work programme, under the relevant topic
(including key performance indicators/metrics for monitoring results and impacts);\\
-- improving innovation capacity and the integration of new knowledge
(strengthening the competitiveness and growth of companies by developing
innovations meeting the needs of European and global markets; and, where
relevant, by delivering such innovations to the markets;\\
-- any other environmental and socially important impacts (if not already
covered above).\\
Describe any barriers/obstacles, and any framework conditions (such as
regulation and standards), that may determine whether and to what extent
the expected impacts will be achieved. (This should not include any risk
factors concerning implementation, as covered in section 3.2.)}

\draftpage

\subsection{Measures to Maximise Impact}

\subsubsection{Dissemination and Exploitation of Results}
\label{subsubsect:dissemination}

\eucommentary{-- Provide a draft 'plan for the dissemination and exploitation
of the project's results'. The plan, which should be proportionate to the
scale of the project, should contain measures to be implemented both during
and after the project.\\
Dissemination and exploitation measures should address the full range
of potential users and uses including research, commercial, investment,
social, environmental, policy making, setting standards, skills and
educational training.\\
The approach to innovation should be as comprehensive as possible,
and must be tailored to the specific technical, market and organisational
issues to be addressed\\
-- Explain how the proposed measures will help to achieve the expected impact of the
project . Provide a draft business plan for financial sustainability as stated in the Part
E of the Specific features for Research Infrastructures of the Horizon 2020 European
Research Infrastructures (including e-Infrastructures) Work Programme 2014-2015.\\
-- Where relevant, include information on how the participants will
manage the research data generated and/or collected during the
project, in particular addressing the following issues:
What types of data will the project generate/collect? What
standards will be used? How will this data be exploited and/or
shared/made accessible for verification and re-use (If data cannot
be made available, explain why)? How will this data be curated and preserved?\\ \\
-- Include information about any open source software used or developed by the
project.\\
You will need an appropriate consortium agreement to manage (amongst other things)
the ownership and access to key knowledge (IPR, data etc.). Where relevant,
these will allow you, collectively and individually, to pursue market opportunities
arising from the project's results.\\
The appropriate structure of the consortium to support exploitation is addressed
in section 3.3. \\ \\
-- Outline the strategy for knowledge management and protection. Include measures to
provide open access (free on-line access, such as the ``green'' or ``gold'' model) to
peer-reviewed scientific publications which might result from the project.\\
Open access publishing (also called 'gold' open access) means that an article is
immediately provided in open access mode by the scientific publisher. The associated costs
are usually shifted away from readers, and instead (for example) to the university or
research institute to which the researcher is affiliated, or to the funding agency supporting
the research.\\
Self-archiving (also called ``green'' open access) means that the published article or the
final peer-reviewed manuscript is archived by the researcher - or a representative - in an
online repository before, after or alongside its publication. Access to this article is often -
but not necessarily - delayed (``embargo period''), as some scientific publishers may wish to
recoup their investment by selling subscriptions and charging pay-per-download/view fees
during an exclusivity period.}


\paragraph{Long term sustainability}

By design (Objective~\ref{objective:framework}), the VRE's promoted by
\TheProject will consist of a thin layer on top of an ecosystem of
components. Hence, the long term sustainability of those VRE is
guaranteed by the sustainability of the ecosystem of components, that
is by Objective~\ref{objective:sustainable}.

\draftpage

\subsubsection{Communication activities}
\label{subsubsect:communication}

\eucommentary{Describe the proposed communication measures for promoting the
project and its findings during the period of the grant. Where appropriate
these measures should include social media and public events with user
participation. Measures should be proportionate to the scale of the project,
with clear objectives. They should be tailored to the needs of various audiences,
including groups beyond the project's own community. Where relevant, include
measures for public/societal engagement on issues related to the project.}
