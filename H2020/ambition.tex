\subsection{Ambition}



\eucommentary{1-2 pages}

\eucommentary{-- Describe the advance your proposal would provide beyond the
state-of-the-art, and the extent the proposed work is ambitious. Your answer
could refer to the ground-breaking nature of the objectives, concepts
involved, issues and problems to be addressed, and approaches and methods to be used.\\
-- Describe the innovation potential which the proposal represents. Where relevant, refer to
products and services already available, e.g. in existing
e-Infrastructures.}

For most pure mathematicians using computational tools in their
research, the state of the art in the beginning of 2015 is still a collection of
programs each of which must be installed individually on their
desktop or laptop computer, respecting a complicated dependencies graph.
Alternatively software may be installed on a
departmental server or cluster and used via text-based remote
login. The software performs computations (using excellent
implementations of extremely sophisticated algorithms) with inputs and
outputs usually in a bespoke text-based format. 
Multiple computations involved in producing a mathematical
result must be managed by editing, naming and filing multiple scripts
or programs, and there is no automatic support for rerunning
computations to check for human or algorithmic error. The results of
computations are incorporated into publications by cut-and-paste and
collaboration is through exchange of programs and data by email,
shared general-purpose file servers or, rarely, a service such as
GitHub. Amongst other problems,  this approach creates a serious obstacle to the reproducibility
of published computational experiments both by other researchers and
the authors themselves at a later time.
% see e.g. "Case Studies and Challenges in Reproducibility in the 
% Computational Sciences", http://arxiv.org/abs/1408.2123, submitted

There are commercial ``symbolic computation systems'' such as
\Mathematica or \Maple which offer somewhat more modern frameworks, but
they lack the specialised algorithms for research work in many fields
of pure mathematics, including for instance
abstract algebra, number theory and algebraic geometry and they
are often not well-suited to support them. 
\TOWRITE{AK/MK}{This statement needs verification. It's not only the lack of algorithms
in these areas. Moreover, we want to cater for wider areas of
mathematics == made it a bit less precise SL}

The need for a more modern, more productive and less error-prone
environment for this kind of mathematical research computing is widely
acknowledged, but the separate groups developing the systems have
individually neither the time nor the expertise to develop it. There
have been a number of interesting projects which have explored
different aspects of what is needed, in particular
\SMC, \HPCGAP, \scienceproject (for all three, see \ref{linked-projects});
\Sage and its notebooks;
Polymath and MathOverflow (see MathSoMac entry in \ref{linked-projects});
and \software{Recomputation.org}.
We will build on the experiences, and where useful, on the software, of all of these.
\TOWRITE{AK/MK}{Cleanups in the list of projects}

Our ambitious plan in this project is to learn from, and leapfrog,
these piecemeal developments and provide a toolkit of software and
interfaces, which supports the whole mathematical research process in
a way which is \textbf{modern}, \textbf{seamless},
\textbf{collaborative}, mathematically \textbf{rigourous} and
\textbf{adaptable} to the diverse needs of different mathematical
research areas and of different mathematicians and collaborations.

The system will be \textbf{modern} in its construction: following
best practices in distributed software development,
internationalisation, use of web and clouds services, etc.; in its user
experience, offering a modern supportive UI that automates all of the
routine tasks that it can; and in its support for important new
research areas that may cross traditional subdiscipline boundaries. It
will combine \textbf{seamlessly} a range of software components,
hardware resources and databases, so that the user can work, or
program with any combination of them in the same way (but, where
relevant, can still attribute credit correctly). It will be
\textbf{collaborative}, with shared projects the norm and discussion
and exploration integrated with computation and writing. It will be
\textbf{rigourous} in that, for instance, data passed between systems
will be translated according to its mathematical meaning, not just its
textual presentation. Finally it will be \textbf{adaptable} allowing
an environment to be easily built and deployed to suit anything
from a lone researcher tackling a problem for a week or two up to a
complex project with subteams and multiple publications.
  

\subsubsection{Challenges specific to  mathematics}

Mathematical research, especially pure mathematics, presents some
unique challenges to the realisation of this ambition.


\begin{itemize}
\item The community mainly consists of individuals or \textit{very} small
  groups (perhaps a professor and a few students). There are far fewer formal or structured research
  teams such as you might find in an equipment-intensive science. There are 
  certainly examples of large scale collaborations, for instance the
  project to prove the Classification of Finite Simple Groups in the
  1980s and the Polymath experiments in the last few years,
  but these are driven by individuals, not defined by formal structures or funding bodies.
\item Many top researchers have little or no formal research
  funding. If they need computational resources, these are limited to what 
  is already available nearby, such as personal laptops or
  departmental clusters or to what they can access by asking favours
  of friends.
\item Many mathematical computations are highly complex and irregular.
  Run times are not predictable and simple decomposition paradigms do
  not work well. Thus,
  traditional HPC approaches coming from numerical simulations and linear algebra do not apply.
\item Mathematical notations have been refined over many centuries to be
  used by humans with pen, paper and blackboard. Even such simple
  problems as selecting a sub-expression are hard to handle well on a
  computer. For instance $a+c$ is naturally seen as a subexpression of
  $a+b+c$ by a human.
\item The mathematical correctness of widely used algorithms hinges on
  quite complex chains of reasoning. Subtle coding errors may easily
  produce plausible, but wrong, answers.

\item Mathematical data differ in several ways from typical
  scientific data
  \begin{itemize}
  \item More often rather than not, data is the result of a computation (and
    not a measurement of the real world). The role of databases is thus primarily
    to store results for later search and reuse. 
    Because of this, many issues (semantics, ontologies,
    reproducibility) involve the software which produced the data as
    much as the data itself.
%  \item Extreme reification in mathematics makes classical ontologies
%   techniques (such as e.g. RDF) impractical. \TODO{Someone explain this}
  \item The stored form of mathematical data (ultimately as strings of
    bits) is much further from the meaning of the data as perceived
    by a mathematician than is usual in other sciences. To make the
    link, many related objects and conventions must be considered and
    most interesting mathematical objects have multiple
    representations. Many mathematical theorems are implicit in these
    forms of representation, so that proving an ontology consistent
    may be very difficult.
  \end{itemize}
\end{itemize}

\subsubsection{Challenges of a community built around multiple
  existing software projects}

Another source of unique challenges for this project is the need to
interact with several large and diverse ecosystems of software
developers. For instance the \GAP package development community, the
\Sage development community, the wider Python community, the developers
of key open-source libraries on which we rely and so on.

These communities exist in a delicate balance between collaboration
and competition. For instance the \scienceproject and \Sage were
simultaneously exploring two different approaches to linking
open-source mathematical software. Many technical developments (better
IO handling in \GAP, for instance) could usefully be shared, and at
the end of the day we all want to do better mathematics, but a certain
degree of competition is both natural and healthy.

In this project we need to build a sustainable ``meta-ecosystem'' in
which systems may compete to have the best designs or algorithms, but
all agree to cooperate on interfaces, bug reporting, testing, etc. to
keep the final user experience seamless and reliable.

\TOWRITE{All}{Describe innovation potential}

\paragraph{Innovation Potential}

Nothing similar to the proposed \TheProject VRE has been developed
before, so the whole project is aimed at innovation. The closest model
is \SMC, the first usable VRE with extensive support specifically for
pure mathematics.
% Comment by William:
% There are probably 10-20 webapps out there that feel somewhat like SMC
% -- browser based code editor, terminal, etc., -- but most are aimed at
% web developers, and the exceptions just have IPython +
% numpy/scipy/matplotib as their extra math functionality... which
% doesn't address pure math.
It differs from \TheProject in consisting of a single software
component, deployed at a single site, and with no public API
for other web services to build on it.
\TOWRITE{SL}{Improve this according to William's comments}
Apart from a collaborative document editor it offers no support for
publication of data or programs, or citability, or for automatic
reproduction of published results.
\TheProject will make it easy for SMC and other VRE's build on this
toolkit to address these and other limitations

The specific innovations in this project will also have have wider
applicability. Indeed each and every improvement we will contribute to
software components of the \TheProject, and in particular key tools
like \Jupyter, will benefit their larger user communities (typically
scientific computing) independently of whether they use VRE's or not.


%%% Local Variables:
%%% mode: latex
%%% TeX-master: "proposal"
%%% End:

%  LocalWords:  eucommentary textsuperscript textregistered textsuperscript specialised
%  LocalWords:  textregistered recomputation textbf textbf rigourous centred flagshsip
%  LocalWords:  subsubsection realisation textit
