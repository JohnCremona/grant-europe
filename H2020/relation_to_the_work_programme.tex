\subsection{Relation to the Work Programme}

% \eucommentary{
% Indicate the work programme topic to which your proposal relates, and
% explain how your proposal addresses the specific challenge and scope
% of that topic, as set out in the work programme.}

\enlargethispage{5cm}

\TheProject addresses the topic “E-infrastructures for Virtual Research
Environments (VRE)”, under E-Infrastructures-2015 call. In the table
below we explain how this project addresses the specific challenge and
the scope of that topic, as set out in the work program.

\TOWRITE{NT}{Fix the margins for the table below}

%\begin{table}[H]
%\tablehead{}
\begin{tabular}{|m{4.0cm}|m{9.5cm}|}
\hline
Specific challenge &
\TheProject contribution \\\hline
Empower researchers through development and deployment of service-driven
digital research environments, services and tools tailored to their
specific needs. &
\TheProject will empower researchers in mathematics and applications by
developing a service-driven tool, based on software, knowledge and data
integration. Tailored to the researchers' specific needs and workflows,
the VRE will support the entire life-cycle of computational work in
mathematical research. It will improve the productivity within the
community by investigating better collaboration processes, and
identifying, sharing and promoting software development best
practices.\\\hline
VRE should integrate resources across all layers of the e-infrastructure
(networking, computing, data, software, user interfaces) &
\TheProject will indeed integrate resources across all layers of the
e-infrastructure : software development models, collaborative tools,
data, component architecture, deployment frameworks, standardization,
social aspects, but also fostering collaboration inside the community,
community enlargement  and links with other scientific communities.
\\\hline
VRE should foster cross-disciplinary data interoperability. &
\TheProject will foster a sustainable ecosystem of interoperable source
components developed by overlapping communities, and data
interoperability between different fields of mathematics.\\\hline
VRE should provide functions allowing data citation and promoting data
sharing and trust.  &
The project will allow an easy, safe and efficient storage, reuse and
sharing of rich mathematical data, taking account of provenance and
citability. It will allow data sharing in a semantically sound way, and
make software sustainable, reusable and easily accessible.\\\hline
Scope &
\TheProject contribution\\\hline
Each VRE should abstract from the underlying e-infrastructures using
standardized building blocks and workflows, well documented interfaces,
in particular regarding APIs, and interoperable components &
We will use building blocks with a sustainable development model that
can be seamlessly combined together to build versatile high performance
VREs, each tailored to a specific need in pure mathematics and
application.
%
We will develop and demonstrate (WP3)  a set of APIs enabling components
such as database interfaces, computational modules, separate systems
such as \GAP or \Sage to be flexibly combined
and run smoothly across a wide range of environments (cloud, local,
server etc.). Through well defined APIs, we will enable discovery of
subsystems, functionality, documentation and computational
resources.\\\hline
The VRE proposals should clearly identify and build on requirements from
real use cases &
\TheProject will be built on the requirements from the real use cases,
including those involving industrial stakeholders. At the end of the
project, the effectiveness of the VRE will be demonstrated for a number
of real use cases from different domains.\\\hline
They should re-use tools and services from existing infrastructures and
projects at national and/or European level as appropriate.  &
\TheProject project brings together and integrates already existing tools
and interactive scientific computing environments : \GAP, \Sage, \Linbox,
\PariGP, \Singular and \IPython, connected to databases, that will allow a
huge gain in efficiency and productivity, enabling a large-scale
collaboration on software, knowledge, and data.\\\hline
Where data are concerned, projects will define the semantics,
ontologies, the \emph{what} metadata, as
well as the best computing models and levels of abstraction (e.g. by
means of open web services) to process the rich semantics at machine
level, as to ensure interoperability. &
We will investigate patterns to share data, ontologies, and semantics
across computational systems, possibly
connected remotely. We will leverage the well established semantics used
in mathematics (categories, type systems ) to give powerful
abstractions on computational objects.\\\hline
\end{tabular}
%\caption{Relation to the work program}
%\end{table}

\clearpage

%%% Local Variables: 
%%% mode: latex
%%% TeX-master: "proposal"
%%% End: 

%  LocalWords:  Programme eucommentary enlargethispage tablehead hline citability Linbox
%  LocalWords:  IPython emph clearpage
