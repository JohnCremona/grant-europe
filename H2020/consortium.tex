\TOWRITE{HPL}{Proofread 3.4 consortium pass 1}
\TOWRITE{ALL}{Proofread 3.4 consortium pass 2}

\eucommentary{\begin{itemize}
\item
Describe the consortium. How will it match the project's objectives?
How do the members complement one another (and cover the value chain,
where appropriate)? In what way does each of them contribute to the
project? How will they be able to work effectively together?
\item
If applicable, describe the industrial/commercial involvement in the
project to ensure exploitation of the results and explain why this is
consistent with and will help to achieve the specific measures which
are proposed for exploitation of the results of the project (see section 2.3).
\item
Other countries: If one or more of the participants requesting EU funding
is based in a country that is not automatically eligible for such funding
(entities from Member States of the EU, from Associated Countries and
from one of the countries in the exhaustive list included in General
Annex A of the work programme are automatically eligible for EU funding),
 explain why the participation of the entity in question is essential to carrying out the project
\end{itemize}
}

\TODO{Convert site names to standard abbreviations}

The consortium brings together:
\begin{enumerate}
\item \label{mathsoftware} Lead or core developers of a cross-section of the major open
  source computational components for pure mathematics and applications: \GAP (St.~Andrews,
  Oxford), \Linbox (Grenoble), \MPIR (Kaiserslautern), \Pari (CNRS Aquitaine, Versailles), \Sage
  (Orsay, Versailles, CNRS Aquitaine, Oxford, Warwick, Zürich), \Singular (Kaiserslautern).
\item \label{mathdb} Lead developers of a major online mathematical database: \LMFDB
  (Warwick, Zürich).
\item \label{mathknowledge} Experts in mathematical knowledge management (Bremen).
\item \label{smc} The lead developers of the closest thing currently existing to a Virtual
  Research Environment for mathematics: \SMC (Seattle). Because of the key role of \Sage
  in several aspects of the project, and the relevant of \SMC as a forerunner of the types
  of systems we want to build, the participation of this group is essential, even though
  they are not eligible for Horizon 2020 funding. They have agreed to provide the benefit
  of their experience on an unfunded basis, since they wish to remain closely involved in
  all developments in mathematical VREs.
\item \label{jupyter} Experts and major ``promoters'' of the \Jupyter collaborative user
  interfaces for interactive and exploratory computing in a variety of scientific domains
  (Southampton, Simula, Sheffield, Silesia).
\item \label{pythran} Lead developers of the Pythran system for automatic conversion of
  Python to C++ and experts in numerical code optimization/parallelization (\site{LL},
  Grenoble)
\item \label{logilab} A company specialized in open-source based Database and Scientific
  Computing for industry (Logilab); it develops in particular its own virtual environment
  \Simulagora.
\item \label{social} Leading researchers in the sociology of mathematical research and
  collaboration. In particular the coordinating partner of the ``Social Machine of
  Mathematics'' project which has been studying how mathematicians collaborate.
\end{enumerate}





There are many existing points of contact between these groups  and
communities, although many of them are also new to one another. This,
together with the fact that each community is internally collaborative
and part of the broader free software community gives us confidence in
their ability to work together.

\TOWRITE{ALL}{Long track record of collaborations between many of the
  sites. Some of the language below can be used.

Writing interfaces between computer algebra systems from different areas and collaborative 
software development are important themes within the DFG Priority Project SPP1489.
As in the {\sc{Sage}} community, networking measures include the regular exchange 
of developers and the regular organization of software workshops (coding sprints) which 
bring whole teams together for solution finding and intense code writing. Particular tight 
collaborations exist between the {\sc{GAP}} and the {\sc{Singular}} communities, with 
major {\sc{GAP-Singular}} developers meetings taking alternately place at St~Andrews,
Kaiserslautern, and Aachen. See \url{http://www.computeralgebra.de/}.
}

The exact role of each partner in each work package is defined
in \ref{sec:workpackages}, but in general terms:
\begin{itemize}
\item Groups \ref{mathsoftware}, \ref{mathdb}, and
\ref{mathknowledge} will collaborate be to design the \TheProject VRE
architecture -- the set of interfaces and standards that allow
components to be assembled into bespoke VREs for particular projects
or areas. This architecture will be informed  by the
experience of \ref{logilab} and \ref{smc}; by the sociological
understanding of \ref{social} and from a 
\textbf{diverse range of real world use cases from all areas of scientific
  computing, in academia and industry} drawn from their own user bases
and contributed by \ref{logilab} and \ref{smc}.

They will be supported in this work by \ref{smc}, \ref{jupyter} and \ref{pythran} which
bring respectively expertise in the key technologies \SMC and \Jupyter and the cython
technology (Seattle).

\item All the participants will make use of the \TheProject VRE, providing feedback to the
  developers and contributing later to the development of demonstrator projects
  (Objective~\ref{objective:demo}). They will also all participate actively in
  dissemination (Objective~\ref{objective:disseminate}) activities.

\item Group \ref{jupyter} will host and mentor core \Jupyter developers to improve this
  key technology (Objective~\ref{objectives:core}), while \ref{mathsoftware},
  \ref{mathdb}, and \ref{mathknowledge} will update their mathematical software components
  (Objective~\ref{objective:updates}) to comply with the newly developed
  interfaces. \ref{pythran} will be a key asset for this work, providing expertise in
  massive parallelism and HPC, bringing in and further developing the specific \Pythran
  optimization technology and providing expertise for development of related technologies
  in other components.  \TOWRITE{ALL}{Experience in developing sustainable large open
    source software}

\item Group \ref{logilab} will further bring in expertise in semantic databases,
  distribution of large software, and open source based business models.

\item Groups \ref{mathsoftware}, \ref{smc}, and \ref{jupyter} already have strong
  experience in community building and engagement (Objective~\ref{objective:community})
  for instance through the very active user and developer communities around \GAP and
  \Sage. These communities and the dissemination to them of the availability new free
  software constitute the primary exploitation route for this project.
  \TOWRITE{ALL}{Experience in community building, engagement, dissemination}
\end{itemize}
