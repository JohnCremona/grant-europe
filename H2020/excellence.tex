\TOWRITE{ALL}{Proofread 1. Excellence introduction pass 2}
% Some guiding questions of Wolfram:

% Agreed, but why is such a thing needed? "I have worked with CAS for
% years and I am fine with what is there". "I still can think with my
% own head and do not need a CAS". "I do not understand what a MathVre
% is".

% The art here is to explain in, say, at most 10-15 lines, why the
% design of CAS is a success story for math, what a MathVRE provides
% in addition and why we need it, and why we are the right people to
% create it. Again, we might even run into referees whom we first have
% to convince that CAS is a good thing to have!

\COMMENT{For good or bad, the ``pure math'' aspect does not appear in
  this introduction.}

\TOWRITE{All}{Try to include the following suggestions by Wolfram:
  \begin{itemize}
  \item The mathematics involved has originally not been developed for
    the application (as for the Radon transform). It was already
    there, when needed.
  \item In a ever faster changing world, we need a reasonably ample
    mathematical algorithmic tool kit to choose from, when the need
    arises.
  \item At the beginning, applications have been dominated by inexact
    but fast numerical methods.  With more powerful computers
    available, exact computer algebra methods have become more and
    more crucial (examples such as cell phones, internet security.)
    Will now enter all application areas.
 \item  The things we provide are indispensable for the quick reactions needed
   and for allowing the potential applicants use the systems we develop
   (so far only used by specialists) Mention the success of mathlab as an
    important example on the numerical side.
  \end{itemize}
}

% maths at the core of technology and innovation
\TOWRITE{All}{Improve the examples? Something from health care depending on
  pure maths?}

Improvements of the economy, ecology, health care and
society overall are driven though innovation. The key tools
for innovation are mathematical knowledge and
algorithms. Examples include the global positioning system (GPS)
needing relativistic mathematics, mobile phone connectivity relying on
combinatorial optimization algorithm for frequency allocation, and
communication security depending on cryptographic methods derived from
computational number theory. Engineering, Science and Business
innovations that enrich society and mankind are made possible through
mathematical foundations whih are often developed long before their potential
applications.
%
% recent developments
Reciprocally, modern mathematical research is increasingly accelerated by and
enabled through collaborative tools, computational environments and
online databases. These digital tools have the potential to
revolutionise the way research is conducted.

% what is this proposal about - aim
In this project, we will provide mathematicians and scientists with a
generic unified toolkit, the Open Digital Research Environment Toolkit
for the Advancement of Mathematics (\TheProject), that allows
building of specific \emph{Virtual Research Environments} (VREs) for
particular tasks and communities.
%and (ii) more effective communication of research.


% How will we achieve this?
We will achieve this by focusing on a \emph{toolkit of software
  components} from which \emph{tailored VREs can be assembled
  flexibly} to cater for the diversity and evolution of needs in
mathematics, science and engineering.  We are at a critical point providing
an opportunity to do so: collaborative tools for code sharing (e.g.
\texttt{github}) now allow us to bring together very large communities
of open source code developers. % working on the same codebase.

Simultaneously the last decade has witnessed the emergence of fundamental
open source building blocks, at the forefront of which are computational
components such as the general purpose mathematical software system \Sage
and the interactive computing environment \Jupyter (successor of \IPython).
Throughout this project we will reuse and extend open source code, and
\TheProject will benefit from future open source contributions during
and beyond the lifetime of the project. By unifying tools with
overlapping functionality, such as \Jupyter and \Sage with their notebooks, we focus the
effort of the computational community onto \TheProject, producing
additional economies of scale.
\COMMENT{By a reader: what does "producing additional economies of scale" mean?}
Finally, thanks to the ``by users for
users'' model, the development will be steered by the actual needs of
the community.

% other things we should say somewhere
\TOWRITE{AK}{Describe two typical combinations, one for the pure
  mathematician together with the one for the numerical scientist}
In more detail, VREs based on \TheProject can combine symbolic
mathematics, automatic code generation, numerical computation, data
bases, post-processing and visualisation in a single collaborative
workspace. The basic units are executable documents, i.e., data- and
code- driven narratives that combine live code, equations, text,
interactive dashboards and other rich media. Potential applications
include active scientific logbooks, papers, lecture notes, etc.,
covering the whole lifecycle of a mathematical research project.

% Impact
\COMMENT{By a reader: what are 'step changes'}\COMMENT{NL: they are changes in function value where the function is not continuous at the point of change}
This will enable step changes in effective research, research
communication, and reproducibility in computational mathematics and
science. It will further provide end-to-end toolchains that link
fundamental mathematics to domain specific specialised computation,
thus bridging the gap between fundamental research and technology, and
paving the way towards faster application, exploitation and
commercialisation of basic research.

% other things we do to make this a holistic project [maybe expand here]
As part of this project, we will also study the social challenges
associated with large-scale open source code development and
publications based on executable documents, and implement
demonstrator VREs based on \TheProject.

% about the team
The \TheProject team is a Europe-wide collaboration that brings
together a leading body of mathematicians and transdisciplinary
computational researchers, with an extensive track record of
delivering innovative open source software solutions.

% conclusion
By focusing on a toolkit rather than a monolithic VRE, and by
concentrating the efforts on improving and unifying existing general
purpose building blocks, and in the forefront \Jupyter, \TheProject
will simultaneously maximize sustainability and broad impact. Indeed,
though the primary target users are \emph{researchers in
  mathematics}, the set of beneficiaries extends to workers in scientific
computing, physics, chemistry, biology, engineering, medicine, earth
sciences and geography, social sciences and finance, and includes researchers as well as teachers
and practitioners in the industry. \TheProject will further foster
development models that are mutually beneficial to academia and highly
innovative SMEs.






\clearpage


%%% Local Variables:
%%% mode: latex
%%% TeX-master: "proposal"
%%% End:
