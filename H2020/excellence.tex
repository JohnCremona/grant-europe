% maths at the core of technology and innovation
Improvements of the economy, ecology, health care and society overall
are driven though innovation. The key enabling tool for innovation
advances is mathematics, examples including the global positioning
system (GPS) needing relativistic mathematics, and mobile phone
connectivity and communication depending on combinatorial optimization
and cryptographic algorithms derived from number theory. Engineering,
Science and business innovation that enrich society and mankind are
made possible due to these mathematical foundations.

% recent developments
Modern mathematical research is increasingly accelerated by and
enabled through e.g. collaborative tools, computational environments,
and online databases.
%  computational software, such as \Sage --- an open source
% mathematics software system --- or \Jupyter, an open source browser-based
% notebook with support for code, text, mathematical expressions, inline
% plots and other media. 
These tools have the potential to
revolutionise the way research is conducted. 

% what is this proposal about - aim
In this project, we will provide mathematicians and scientists with a
generic unified toolkit, the Open Digital Research Environment Toolkit
for the Advancement of Mathematics (\TheProject), that allows
building of specific \emph{Virtual Research Environments} (VREs)
%and (ii) more effective communication of research.


% How will we achieve this?
We will achieve this by investing into creation of a \emph{toolkit of
  software components} from which \emph{tailored VREs can be assembled
  flexibly} to cater for a variety of needs in maths, science and
engineering.  We are at a critical point providing an opportunity to
do so: emerging collaboration tools for code sharing, such as \texttt{github},
allow to bring together very large communities of open source code developers
working on the same codebase. \TODO{Collaborative} Simultaneously, specialised
computational open source tools are emerging. Throughout this project
we will reuse and extend open source code, and \TheProject will benefit
from future open source contributions during and beyond the life time
of the project. By unifying tools with overlapping functionality, such as \Jupyter and \Sage, we
focus the effort of the computational community onto \TheProject, producing additional economies of scale. 

% other things we should say somewhere
In more detail, VREs based on \TheProject can combine symbolic
mathematics, automatic code generation, numerical computation, data
bases, post-processing and visualisation in a single document. The
document consists of a number of executable cells which contain code,
that can be interactively executed, and the output of the code. Cells
can also contain arbitrary text and equations as necessary to
describe/document the results. These executable documents can be
shared with others and fully define and document a computational study
-- providing step changes in effective research, research
communication and reproducibility in computational mathematics and science. 

The VREs built on \TheProject will provide end-to-end toolchains that
link fundamental mathematics to domain specific specialised
computation, thus bridging the gap between fundamental research and
technology, and paving the way towards faster commercialisation of
basic research.

% other things we do to make this a holistic project [maybe expand here]
As part of this project, we will also study the social challenges
associated with large-scale open source code development, publications based on executable documents, and develop
demonstrator VREs based on \TheProject.

% about the team
The \TheProject team is a Europe-wide collaboration that assimilates a
leading body of mathematicians and transdisciplinary computational
researchers with a track record of delivering innovative open source
software solutions. All partners are simultaneously code developers and
end-users of the toolkit.

% conclusion
By focusing on a toolkit rather than a monolithic VRE, and by
concentrating the efforts on improving and unifying existing general
purpose building blocks, and in the forefront \Jupyter, \TheProject
will simultaneously maximize sustainability and broad impact. Indeed,
even if the primary target users are \emph{researchers in
  mathematics}, the set of beneficiaries extends to scientific
computing, physics, chemistry, biology, engineering, medicine, earth
sciences and geography, and include researchers as well as teachers
and practitioners in the industry. \TheProject will further foster
development models that are mutually beneficial to academia and highly
innovative SMEs.






\clearpage


%%% Local Variables:
%%% mode: latex
%%% TeX-master: "proposal"
%%% End:
