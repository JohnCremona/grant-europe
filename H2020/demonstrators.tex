\subsection{Micromagnetic VRE}
Micromagnetics is a continuum theory description of the behaviour of
the magnetisation vector field at length scales of the order of
micrometers and below. It is widely used in the research and
development of magnetic data storage media and devices, for magnetic
sensing, permanent magnets and healthcare applications such as cancer
treatement and diagnostics. The mathematical model is a time dependent
nonlinear partial differential equations with multiple length and time
scales in the problem, and solution strategies are based on finite
difference and finite element space discretisations and sophisticated
numerical solution of the equations. As in many other research fields,
the groups carrying out the simulations are often not the code
developers, nor have they extensive computational background. More
commonly, these are material scientists, engineers and physicists that
use the simulation to interpret their experiments and support their
device design planning. Industrial users include Seagate, Hitachi,
TDK, Samsung, Bosch and Toyota.

In this project, we will use the \TheProject components to create a
micromagnetic VRE to (i) support the micromagnetic simulation
community, (ii) exploit that experience to evaluate the real value of
this VRE and \TheProject to a large and diverse set of end-users. 

In more detail, we will embed the most popular micromagnetic
simulation software: the Object Oriented MicroMagnetic Framework
\cite{oommf-url} within a micromagnetic VRE, complement this with
value-adding features, develop a substantial number of executable
documents inside this VRE that act as tutorials and documentation,
disseminate the software and documents as open source and through
workshops for the micromagnetic community, and carry out repeated
evaluations of the value of this, feeding results back into the
\TheProject work where possible. This will also be a case study for
the sustainability of the approach and tool beyond the life time of
this H2020 project.



