\paragraph{Principal Investigator Prof Hans Fangohr} Hans Fangohr is Professor of Computational Modelling at the University of Southampton. He heads the University's interdisciplinary Computational Modelling Group (http://cmg.soton.ac.uk), and has more than 100 publications on applied computer simulation in magnetism as well as development of computational methods. He has attracted over \EUR{12}m as investigator and co-investigator for teaching, research and research infrastructure grants.

In 2013, he attracted \EUR{5}m from the UK's Engineering and Physical
Sciences Research Council (EPSRC) to fund the only Centre for Doctoral
Training in Next Generation Computational Modelling (ngcm.soton.ac.uk)
in the UK. This flagship activity will train about 75 PhD students (10
to 15 starting every year, beginning in September 2014) in the
state-of-the-art and best-practice in computational modelling, the
programming of existing and emerging parallel hardware and to apply
these skills and tools to PhD research projects across a range of
topics from Science and Engineering. The centre has chosen IPython as
a key tool to deliver this teaching, document and communicate
computational exploration and drive reproducible computation to push
for excellent computational science.

Hans Fangohr has led the development of the Open Source Nmag software
(http://nmag.soton.ac.uk), which prodives a finite-element
micromagnetic simulation suite to a community of material scientists,
engineers and physicists who research magnetic nanostructures in
academia and industry. He has designed the package in 2005 so
that it has an IPython-compatible Python interface, to make the
workflow of using the simulation package as accessible as possible to
scientists without substantial computational background. He has
extensive experience in micromagnetic simulation use and development.


