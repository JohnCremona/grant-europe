\begin{participant}[type=PI,gender=male]{Prof. Dr. Wolfram Decker}

Wolfram Decker is a professor of mathematics at TU Kaiserslautern.
He formerly was a research fellow at Berkeley with a NATO grant,
a visiting researcher at Kyoto with a JSPS grant, and a professor
at Saarbr\"ucken, Germany. Decker has more than thirty publications
including two books on computational algebraic geometry and papers 
in Compositio, Crelle, and Mathematische Annalen. He has held several 
grants in four different priority programmes of the German Research 
Council DFG and is now coordinator of the
priority programme SPP 1489 ``Algorithmic and Experimental 
Methods in Algebra, Geometry, and Number Theory''. He was also
coordinator of the European algebraic geometry network
EuroProj (1996--1999) and Chair of the programme management 
committee of the European algebraic geometry network EAGER
(2000--2004). He held seven grants for EU Highlevel Scientific 
Conferences and (co-)organized about 50 conferences, summer 
schools, workshops, and coding sprints. He was Chair of 
the Minisymposium on Computer Algebra during the third ECM.
Decker has supervised 13 PhD students. He has been a frequent
lecturer at the African Institute of Mathematics (AIMS) at
Cape Town, and he has run 8 schools on computational
algebraic geometry in different countries.

Decker's research interests lie in areas of algebraic geometry and 
computer algebra. In addition to writing theoretical papers, he is
a leader in mathematical software development and has written thousands of 
lines of code himself. He has made contributions to the systems 
{\sc{Macaulay2}} and, much more substantially, {\sc{Singular}}. 
Since 2009 he is the head of the {\sc{Singular}} development team.
Current tasks of the team include crosslinking {\sc{Singular}} to
other systems, most notably to {\sc{GAP}}, and parallelizing
{\sc{Singular}}. These tasks are fundamental to the
\textbf{MathVRE} project.
\end{participant}
%%% Local Variables:
%%% mode: latex
%%% TeX-master: "../proposal"
%%% End:






