\begin{participant}[type=PI,PM=12,salary=7500]{Nicolas M. Thiéry}
  Professor at the Laboratoire de Recherche en Informatique, Nicolas M. Thiéry is a senior
  researcher in Algebraic Combinatorics with 15 papers published in international
  journals. Among other things, he is a member of the permanent committee of FPSAC, the
  main international conference of the domain, and has collaborators in Canada, India, and
  in the US where he spent three years (Colorado School of Mines, UC Davis); he also
  coorganized fourteen international workshops, in particular Sage Days, and the semester
  long program on "Automorphic Forms, Combinatorial Representation Theory and Multiple
  Dirichlet Series" hosted in Providence (RI, USA) by the Institute for Computational and
  Experimental Research in Mathematics.

  Algebraic combinatorics is a field at the frontier between mathematics and computer
  science, with heavy needs for computer exploration. Pioneer in community-developed open
  source software for research in this field, Thiéry founded in 2000 the \SageCombinat
  software project (incarnated as \MuPADCombinat until 2008); with 50 researchers in Europe and abroad, this project has grown under
  his leadership to be one of the largest organized community of Sage developers, gaining
  a leading position in its field, and making a major impact on one hundred
  publications\footnote{\url{http://sagemath.org/library-publications-combinat.html},
    \url{http://sagemath.org/library-publications-mupad.html}}. Along the way,
%this occasion
%Thiéry gained a strong community building experience, and
  he coauthored part of the proposal for NSF \SageCombinat grant OCI-1147247.

  With 150 tickets (co)authored and as many refereed, Thiéry is himself a core Sage
  developer, with contributions including key components of the Sage infrastructure
  (e.g. categories), specialized research libraries (e.g. root systems), thematic
  tutorials, and two chapters of the book ``Calcul Mathématique avec Sage''.
\end{participant}
%%% Local Variables:
%%% mode: latex
%%% TeX-master: "../proposal"
%%% End:
