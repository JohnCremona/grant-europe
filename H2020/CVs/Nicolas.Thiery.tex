\paragraph{Nicolas M. Thiéry}

Professor at the Laboratoire de Recherche en Informatique, Nicolas
M. Thiéry is a senior researcher in Algebraic Combinatorics with an
international recognition. Among other things, he is a member of the
permanent committee of FPSAC, the main international conference of the
domain, and has collaborators in Canada, India, and in the US where he
spent several years; he also coorganized many international workshops,
in particular Sage Days, and a semester long program hosted by ICERM.

Algebraic combinatorics is a field at the frontier between mathematics
and computer science, with heavy needs for computer
exploration. Pioneer in community-developed open source software for
research in this field, Thiéry founded in 2000 the Sage-Combinat
software project; with 50 researchers in Europe and abroad, this
project has grown under his leadership to be one of the largest
organized community of Sage developers, gaining a leading position in
its field, and making a major impact on one hundred publications. At
this occasion, Thiéry gained a strong community building experience,
and coauthored some of NSF Sage-Combinat grant OCI-1147247.

With 150 tickets (co)authored and as many refereed, Thiéry is himself
a core Sage developer, with contributions including key components of
the Sage infrastructure (e.g. categories), specialized research
libraries (e.g. root systems), thematic tutorials, and two chapters of
the book ``Calcul Mathématique avec Sage''.
