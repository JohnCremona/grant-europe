\subsection*{University of Warwick}

% PIC: 999976784
% see: http://ec.europa.eu/research/participants/portal/desktop/en/orga

The Mathematics Institute at the University of Warwick was ranked 23rd
worldwide in the 2013 QS world university subject rankings.  Five
members of the Department are Fellows of the Royal Society, and one,
Regius Professor Martin Hairer, was awarded a Fields Medal in 2014.
Mathematics and Statistics at Warwick currently hold £35.8M in
research grants from EPSRC (the next highest in the UK being Cambridge
at £22.8M and Oxford at £24.2M).  Nine members of the department
currently hold ERC grants.


% See ../proposal.tex, section Members of the Consortium for a
% complete description of what should go there

\subsubsection*{Curriculum vitae}

% Curriculum of the personnel at this institution

\begin{picv}{John E. Cremona}
  Professor of Mathematics.  DPhil (Oxford, 1981) under Birch.  Previous posts: Michigan,
  Dartmouth (US), Exeter, and Nottingham (as chair and Head of Pure Mathematics). Cremona
  has around 50 publications, including a book and papers in Compositio and Crelle.  He
  has held grants from EPSRC and other UK sources worth \pounds2.5M as well as \euro2.5m
  from the EU for Marie-Curie Research Training Networks in 2000-2004 and 2006-2010.  He
  was a Scientist in Charge of one of twelve teams in both of these networks, and leader
  of the research project ``Effective Cohomology Computations'' in the second.  He has
  been on the Scientific Committee of 30 international conferences (including several Sage
  Days), and given many invited lecture series.  He co-organised semester-long research
  programmes at IHP Paris (2004) and MSRI (2011).  He has been an editor for five
  journals.  He has supervised 16 PhD students, a dozen Masters students, two EU-funded
  postdoctoral fellows and currently has three EPSRC-funded postdoctoral research
  assistants.  Cremona has given over 30 invited conference addresses and seminars in 9
  countries in the last 10 years.

  Cremona's research includes areas of particular relevance to the current project.  His
  methods for systematically enumerating elliptic curves, which are the subject of a book
  and numerous papers, have been used to compile a definitive database of elliptic curves
  which is very widely cited, and now forms part of the LMFDB.  Cremona's experience in
  managing such computations and the management, publication and electronic dissemination
  of the resulting large datasets set a standard which large-scale number-theoretical
  database projects such as the LMFDB now seek to match.  Cremona's experience and
  reputation in this field have been important for the LMFDB project.

  Cremona has been a leading computational number theorist in the UK since his PhD thesis
  in 1981, following in the tradition of Birch and Swinnerton-Dyer.  He has written
  thousands of lines of code in his C++ library eclib (one of the standard packages
  included in Sage since its inception) which includes his widely-use program {\tt mwrank}
  for computing ranks of elliptic curves.  As well as writing thousands of lines of new
  python code for Sage, he has also contributed to the active number-theoretical packages
  Pari/GP and Magma.
\end{picv}
%%% Local Variables:
%%% mode: latex
%%% TeX-master: "../proposal"
%%% End:


%\input{CVs/First.Last.tex}
%\input{CVs/First.Last.tex}
%\input{CVs/First.Last.tex}

\subsubsection*{Publications, products, achievements}
\begin{enumerate}
\item
The Number Theory research group at Warwick was started only in 2006,
but has rapidly risen to international status and one of the largest
and most vibrant groups in Europe, comprising 25 members (professors,
lecturers, postdoctoral researchers and early stage researchers).  Of
the group's members, two (Loeffler and Dokchitser) hold Royal Society
Research Fellowships and one (Bartel) a Royal Commission 1851
Fellowship.  Loeffler won a Leverhulme Foundation Prize jointly with
Zerbes.
\item
Several members of the Number Theory group at Warwick are \Sage
developers, including John Cremona, who has contributed thousands of
lines of code to \Sage\ since 2006 both through his {\tt eclib} C++
library and through original \Python code which forms part of the
\Sage\ library; David Loeffler, who has contributed substantially to
the modular forms module in Sage; and postdoc Marc Masdeu, who has
worked on the \Sage-Flint interface.
\end{enumerate}

\subsubsection*{Previous projects or activities}

\begin{enumerate}
\item
In 2013 Professors John Cremona and Samir Siksek, together with
co-investigators at Bristol, were awarded a six-year major grant of
£2.2M from the UK Engineering and Physical Sciences Research Council
(EPSRC) to support the L-functions and Modular Forms Database (LMFDB)
project.  This grant funds three postdoctoral researchers at Warwick,
computer equipment to host its database and website, and regular LMFDB
workshops.
\item
Each year Warwick hosts a year-long Warwick EPSRC Symposium focussing
on one area of mathematical research.  The 2012-13 Number Theory
Symposium included six research workshops and a summer school ``Number
Theory for Cryptography'' and raised the international profile of the
number theory group substantially.
\end{enumerate}

\subsubsection*{Significant infrastructure}

Computing infrastructure available to the group is excellent, with
seven dedicated machines (over 300 cores) as well as access through
Warwick's Centre for Scientific Computing which hosts a 6000-core
linux cluster and a 3500-core cluster of workstations.
