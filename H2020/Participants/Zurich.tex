\subsection*{University of Z\"{u}rich}

% PIC: 
% see: http://ec.europa.eu/research/participants/portal/desktop/en/orga

% See ../proposal.tex, section Members of the Consortium for a
% complete description of what should go there

The University of Zurich consistently ranks among the top 15 research institutions in Europe. It is the largest university in Switzerland, with over 26000 students, and offers the  most comprehensive academic program of the country.  It has close to 600 professors and over 5000 academic staff. 

Switzerland ranks high in innovation, competitiveness and research spending, and much of this is enthusiasm for research is concentrated around Zurich. UZH also benefits from synergies with the ETH Zurich. 

The Mathematics Institute has 17 professors and around 60 PhD students, part of a graduate school run jointly with ETH Zurich. Also joint is a Computational Science program uniting 47 researchers, mostly in the sciences, who make use of computational methods. 


\subsubsection*{Curriculum vitae}

% Curriculum of the personnel at this institution

\paragraph{Paul-Olivier Dehaye}

% months = 6
% 
%
% Fair evaluation of the number of months you will be spending on this
% specific project along the four years

% salary=YYY
%
% Approximate monthly gross salary (in term of total cost for the
% employer). If you are uncomfortable having this information in a
% public file, you can alternatively send the information to Nicolas,
% or to your institution leader if the latter will be willing to fill
% in himself the budget forms on the eu portal.

% The above information will be used to evaluate the cost of the
% project for the institutions. You may remove the above comments once
% you have filled in the months= and salary= lines.

Paul-Olivier Dehaye is a Swiss National Science Foundation Assistant Professor 
at the University of Zurich. After his Phd at Stanford (2006), he has also worked in Oxford, 
at the Institut des Hautes Etudes Scientifiques and at ETH Zurich. He currently has 13 
papers published in international peer-reviewed journals. 

His main research is at the intersection of Number Theory and Combinatorics, and in 
particular in Random Matrix Theory conjectures. He has additional interests in FLOSS, 
semantic tools, massive online education and crowdsourcing, all with the view of 
enabling larger scale mathematical and scientific collaborations. 

He is a contributor to the sage, LMFDB and OpenEdX projects, and has organised two 
conferences relating to these projects. The first was held in 2013 in Edinburgh, and organised
jointly with Nicolas Thiery. Its official title was  "Online databases: from L-functions to combinatorics", 
and it served as a precursor to some aspects of this grant. The second was held in June 2014 
in Zurich and organised jointly with Stanford. It aimed at building a community around the open 
source MOOC platform OpenEdX, and has initiated a series of conferences held twice annually. 

Dehaye has also taught for two years now a python course using OpenEdX, which aims to bring 
first year students to the level of potential contributor to sage. This course also has a 
project-based component. It is now run locally for a small audience, but could be scaled up 
in various ways. 



\subsubsection*{Publications, products, achievements}
\begin{enumerate}
\item Dehaye is editor for the LMFDB, and has contributed to the project since its inception (2007). His students are also contributors. 
\item papers
\item edx-presenter
\item OpenEdX installation
\item 
 \TOWRITE{POD}{LMFDB, papers, python course}
\end{enumerate}

\subsubsection*{Previous projects or activities}
\begin{enumerate}
\item 
\end{enumerate}

\subsubsection*{Significant infrastructure}
\begin{enumerate}
\item The Faculty of Sciences of the UZH benefits from very strong specialized IT support in the form of the S3IT group. They operate for instance a research cloud and a local supercomputer,  and provide further assistance for the design of hardware and software systems to further research.
\item UZH has a stake in Piz Daint, currently the sixth largest supercomputer in the world, and the most energy-efficient. This supercomputer is now currently expanded. 
\end{enumerate}

\TOWRITE{XXX}{...}


