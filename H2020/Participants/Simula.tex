\begin{sitedescription}{SR}

Dedicated to tackling scientific challenges with long-term impact and of genuine importance to real life, Simula Research Laboratory (Simula) offers an environment that emphasises and promotes basic research. At the same time, we are deeply involved in research education and application-driven innovation and commercialisation. 

Simula was established as a non-profit, limited company in 2001, and is fully owned by the Norwegian Ministry of Education and Research. Its research is funded through competitive grants from national funding agencies and the EC, research contracts with industry, and a basic allowance from the state.  Simula's operations are conducted in a seamless integration with the two subsidiaries Simula School of Research and Innovation and Simula Innovation.

At its outset, the laboratory was given the mandate of becoming an internationally leading research institution within select fields in information and communications technology. These fields are (i) communication systems, including cyber-security; (ii) scientific computing, aiming at fast and reliable solutions of mathematical models in biomedicine, geoscience, and renewable energy; and (iii) software engineering, focusing on testing and verification of mission-critical software systems, and on planning and cost estimation of large software development projects. Recent evaluations state that Simula has met its challenge and is an acknowledged contributor to top-level research in its focus areas. Specifically, in the 2012 national evaluation of ICT research organised by the Research Council of Norway and conducted by an international expert panel, Simula received the highest average score (4.67) on a 1-5 scale among all evaluated institutions.  In comparison, the national average was 3.38. Only five of the 62 research groups evaluated were awarded the top grade (5), and two of these five groups are located at Simula.

Simula is currently hosting one Norwegian Centre of Excellence, Centre for Biomedical Computing (2007-2017), and one Norwegian Centre for Research-based Innovation, Certus (2011-2018). In addition, we participate as research partner in another Centre for Research-based Innovation, Centre for Cardiological Innovation (2011-2018), hosted by Oslo University Hospital. These two centre-oriented schemes are the most prestigious funding instruments offered by the Research Council of Norway. 

\subsubsection*{Curriculum vitae}

% Curriculum of the personnel at this institution

\begin{participant}[PM = 8, type=leadPI, gender=male]{Hans Petter Langtangen}
Hans Petter Langtangen is director of Center for Biomedical Computing at Simula Research Laboratory, a Norwegian Center of Excellence doing inter-disciplinary research in the intersection of mathematics, physics, computer science, geoscience and medicine. Langtangen is on 80\% leave from a position as professor at the Department of Informatics, University of Oslo.

Langtangen received his PhD from the Department of Mathematics, University of Oslo, in 1989, and then worked at SINTEF before being hired as assistant professor at the University of Oslo in 1991. After being promoted to full professor of mechanics at the Department of Mathematics in 1998, he moved in 1999 to a professorship in computer science. In the period 1999-2002 he also held an adjunct professor position at the Department of Scientific Computing at Uppsala University in Sweden. The Simula Research Laboratory was formed in 2001, and Langtangen has since then worked with research and management at this laboratory. The scientific computing activity at Simula has been awarded the highest grade, Excellent, by five panels of top-ranked international scientists in the period 2001-2012.

Langtangen's research is inter-disciplinary and involves continuum mechanical modeling, applied mathematics, stochastic uncertainty quantification, and scientific computing, with applications to biomedicine and geoscience in particular. He has also been occupied with developing and distributing scientific software to make the research results more widely accessible and help accelerate research elsewhere. For over three decades he has been very active with teaching and supervision.

The scientific production consists of 4 authored books, 3 edited books, about 60 papers in international journals, about 60 peer-reviewed book chapters and conference papers, and over 130 scientific presentations. The publications cover fluid flow, elasticity, wave propagation, heat transfer, finite element methods, uncertainty quantification, and implementation techniques for scientific software. Langtangen is on the editorial board of 7 journals and serves as Editor-in-Chief of the leading SIAM Journal on Scientific Computing. He is also a member of the Norwegian Academy of Science and Letters.

\end{participant}

\begin{participant}[PM=24, type=R]{NN}
  We will hire a post-doctoral senior research fellow to carry out the work
  at Simula, under the leadership of and together with Hans Petter Langtangen. The fellow will have a background in computational science,
 combined with profound IPython and
  \Jupyter{} Notebook experience, and past experience of software
  engineering. Ideal candidate will also have good communication skills and team working
  abilities, and in particular interest and skill in the development of
  education materials to best support this part of the project.
\end{participant}

%\input{CVs/First.Last.tex}

\subsubsection*{Publications, products, achievements}

\begin{compactenum}
\item  A. Logg, K.-A. Mardal, G. N. Wells et al. Automated Solution of Differential Equations by the Finite Element Method, Springer (2012). [doi:10.1007/978-3-642-23099-8]
\item P. E. Farrell, D. A. Ham, S. W. Funke, and M. E. Rognes.	Automated Derivation of the Adjoint of High-Level Transient Finite Element Programs. SIAM J. Sci. Comput. 35-4 (2013), pp. C369-C393
\item H.P. Langtangen. A Primer on Scientific Programming with Python. Texts in Computational Science and Engineering, Springer (2014), 792 pp..
\item M. S. Aln\ae s, A. Logg, K. B. \O lgaard, M. E. Rognes, G. N. Wells. Unified Form Language: A domain-specific language for weak formulations of partial differential equations, ACM Transactions on Mathematical Software, 40(2) (2014).
\end{compactenum}

\subsubsection*{Previous projects or activities}

\begin{compactenum}
\item The Centre for Biomedical Computing, a Norwegian Centre of Excellence, awarded by the Research Council of Norway (\euro 10m, 2007--2017).
\item The FEniCS Project (\url{www.fenicsproject.org}, ongoing since 2007). 
\end{compactenum}

\subsubsection*{Significant infrastructure}

The fully owned Simula subsidiary Simula Innovation handles pre-commercial innovation projects, creation and follow-up of company spin-offs, and general support for entrepreneurs.
\end{sitedescription}



%KEY-MORE-TODOS


%%% Local Variables:
%%% mode: latex
%%% TeX-master: "../proposal"
%%% End:

%  LocalWords:  sitedescription Simula Simula commercialisation Certus subsubsection Logg
%  LocalWords:  Mardal Funke Rognes Sci Comput Langtangen FEniCS Aln ae lgaard vspace
%  LocalWords:  TOWRITE emphasises organised
