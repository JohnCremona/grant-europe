\begin{sitedescription}{USH}

% PIC: 999976881
% see: http://ec.europa.eu/research/participants/portal/desktop/en/orga

% See ../proposal.tex, section Members of the Consortium for a
% complete description of what should go there
The University of Sheffield is a leading Reasearch University in the
United Kingdom that was ranked 69th in the World in the most recent
2014 QS World University Rankings and was ranked in the top ten for
the most recent UK wide research asssment exercise.  Professor
Lawrence is based in The Sheffield Institute for Translational
Neuroscience (SITraN) and the Department of Computer Science at the
University of Sheffield have a unique partnership based on two shared
professorial appointments, Professor Winston Hide and Professor Neil
Lawrence.  SITraN is a world leading research centre for
neurodegenerative disease, located in a purpose built building on the
University of Sheffield campus adjacent to the Medical School at
Sheffield Teaching Hospitals NHS Trust Royal Hallamshire Hospital. 

The Department of Computer Science was ranked 5th across UK
departments in ``Research Quality'' by the recent UK-wide Research
Evaluation Framework. It has a particular history of working with data
with internationally leading groups in Machine Learning, Speech and
Language Processing.

The University of Sheffield will focus on tasks on HPC (\taskref{hpc}{hpc-jupyter}), dissemination (\taskref{dissem}{project-intro}) and social aspects of the project (\taskref{}{social-output}). It will host one of the main project meetings and run regular workshops on data science and Gaussian processes which incorporate \TheProject outputs. 

\subsubsection*{Curriculum vitae}

% Curriculum of the personnel at this institution

\begin{picv}[PM=6,salary=10000]{Neil Lawrence}
  Neil Lawrence received his bachelor's degree in Mechanical Engineering from the
  University of Southampton in 1994. Following a period as an field engineer on oil rigs
  in the North Sea he returned to academia to complete his PhD in 2000 at the Computer Lab
  in Cambridge University. He spent a year at Microsoft Research in Cambridge before
  leaving to take up a Lectureship at the University of Sheffield, where he was
  subsequently appointed Senior Lecturer in 2005. In January 2007 he took up a post as a
  Senior Research Fellow at the School of Computer Science in the University of Manchester
  where he worked in the Machine Learning and Optimisation research group. In August 2010
  he returned to Sheffield to take up a collaborative Chair in Neuroscience and Computer
  Science.

  Neil's main research interest is machine learning through probabilistic models. He
  focuses on both the algorithmic side of these models and their application. He has a
  particular focus on applications in personalized health and computational biology, but
  happily dabbles in other areas such as speech, vision and graphics.

  Neil was Associate Editor in Chief for IEEE Transactions on Pattern Analysis and Machine
  Intelligence (from 2011-2013) and is an Action Editor for the Journal of Machine
  Learning Research. He was the founding editor of the JMLR Workshop and Conference
  Proceedings (2006) and is currently series editor. He is Programme Chair for AISTATS
  2012 and has served on the programme committee of several international conferences. He
  was an area chair for the NIPS conference in 2005, 2006, 2012 and 2013, Workshops Chair
  in 2010 and Tutorials Chair in 2013. He was general chair of AISTATS in 2010 and AISTATS
  Programme Chair in 2012. He is Program Chair of NIPS in 2014.

  Neil is a strong advocate of open source software in machine learning and has given many
  invited talks on the subject. Since 2004 his research group has made all their
  implementaitons available, most recently using Python and IPython as the main medium for
  communicating their work. Their Gaussian process python software
  framework\footnote{\url{https://github.com/SheffieldML/GPy}} is becoming the standard
  platform for research in these methods and underpins a series of Summer Schools and four
  day road shows that Neil has led in the
  area.\footnote{\url{http://ml.dcs.shef.ac.uk/gpss/}}
\end{picv}
%%% Local Variables:
%%% mode: latex
%%% TeX-master: "../proposal"
%%% End:

\begin{participant}[type=R, PM=6, salary=7500,gender=male]{Michael Croucher} is a Research Software Engineer at The University of Sheffield. He received his bachelor's degree in Theoretical Physics from The University of Sheffield in 1999 and completed his Theoretical Physics PhD there in 2005. He subsequently took up a research-support post in The University of Manchester's IT Services department before being appointed as Head of Application Software Support for the Faculty of Engineering and Physical sciences.
 
Michael is passionate about improving the quality of research software. He enables researchers to ask larger and more complex research questions by improving the software they develop. By teaching and demonstrating fundamental software engineering principles, he assists academic colleagues in producing robust, reproducible, fast and correct software.
 
He achieves these aims via a number of means:
 
Consultancy: He works directly on research code written in various languages. For a sample of recent client testimonials, see \href{http://www.walkingrandomly.com/?page_id=5122}{http://www.walkingrandomly.com/?page\_id=5122}
 
Outreach: He is the author of \url{http://www.walkingrandomly.com/} - a blog focused on mathematics and scientific computing with over 400,000 unique visitors a year. The associated twitter account, @walkingrandomly, has almost 3000 followers.  He is a fellow of the Software Sustainability Institute, an organisation that promotes and supports research software engineering.
 
Education: He has taught programming and Software Carpentry classes to hundreds of researchers and uses his contacts with education and industry to arrange specialised teaching events relating to research software.
 
Mentoring: He acts as a 'code-coach' to new researchers, providing code reviews and private tutorials.
 
Vendor liaison: He has strong relationships with several vendors of scientific software including Mathworks, Wolfram Research, Maplesoft and NAG. These relationships have led to many fruitful collaborations between them and academic colleagues.
 
High Performance Computing: He has been involved with teams that develop and support Institutional HPC systems such as the Manchester University Condor Pool - a 3000+ CPU core facility made by utilising spare time on hundreds of desktop PCs. In this team, his primary role was to assist researchers in transitioning their workflow from the desktop to the Condor system.
\end{participant}

\begin{participant}[type=R, PM=36]{NN}
  We will hire a post-doctoral research fellow to carry out the work
  at Sheffield in collaboration with Neil Lawrence and Mike
  Croucher. The fellow will ideally have a background in data science,
  combined with solid IPython and \Jupyter{} Notebook experience, and
  past experience of software engineering. We further require good
  communication and team working skills, and in particular interest
  and skill in the development of collaboration and education
  materials. 
\end{participant}
\begin{participant}[type=R, PM=6]{NN}
  We will fund existing post-doctoral research resource to carry out work on implementing \Jupyter on SGE for ease of HPC performance. They will work collaboratively with Neil Lawrence, Mike Croucher and the other Sheffield researcher appointment to achieve this goal. The fellow will have a background in computational science, with experience of HPC and 
  \Jupyter{} Notebook experience. As is normal we will require good communication and team working
  skills.
\end{participant}


%\input{CVs/First.Last.tex}

\subsubsection*{Publications, products, achievements}

\begin{enumerate}
\item N. Fusi, C. Lippert, N. D. Lawrence and O. Stegle. (2014) "Warped linear mixed models for the genetic analysis of transformed phenotypes" in Nature Communications 5 (4890)
\item J. Hensman, M. Rattray and N. D. Lawrence. (2014) "Fast nonparametric clustering of structured time-series" in IEEE Transactions on Pattern Analysis and Machine Intelligence
\item M. A. \'Alvarez, D. Luengo and N. D. Lawrence. (2013) "Linear latent force models using Gaussian processes" in IEEE Transactions on Pattern Analysis and Machine Intelligence 35 (11), pp 2693--2705
\item N. Fusi, O. Stegle and N. D. Lawrence. (2012) "Joint modelling of confounding factors and prominent genetic regulators provides increased accuracy in genetical genomics studies" in PLoS Computat Biol 8, pp e1002330
\end{enumerate}

\subsubsection*{Previous projects or activities}

\begin{enumerate}
\item Organisers of the Gaussian Process Summer Schools (three 3-day workshops on Gaussian process models in python and the IPython notebook).
\item Organisers of five Gaussian Process and Data Science Road Shows (educating on data science and Gaussian process models in Uganda, Colombia, Italy, Australia and Kenya) Each workshop is 3-4 days long. 
\end{enumerate}

\subsubsection*{Significant infrastructure}
\begin{enumerate}
\item The Sheffield Institute for Translational Neuroscience is a 18 million pound world leading institute for research into neurodegenerative disorders. It houses clinicians, biologists and computationalists under a single roof and contains the Sheffield Microarray and Next Generation sequencing Core Facility. The institute provides an exemplar of how mathematical ideas can be rapidly translated to analysis through provision of appropriate software frameworks.
\item The Sheffield group are regular contributors to open source software including a python framework for Gaussian process modelling (\url{https://github.com/SheffieldML/GPy}), contributions to the Bioconductor framework for computational biology (puma, tigre, gprege), and more recently frameworks for open data science (\url{https://github.com/sods/ods}).  
\item The Sheffield group has taken a leading role in education in data science and machine learning with a series of workshops and summer schools which use \Jupyter as the main interface for practical implementation of ideas (\url{http://ml.dcs.shef.ac.uk/gpss/}).

\end{enumerate}
\end{sitedescription}


\begin{draft}
\vspace{1cm}\TOWRITE{NL}{Complete check list below -- delete completed items if you wish}

\begin{verbatim}
- [ ] checked that sum of person months put into finance request is
  the same as sum of person months associated with the Work Packages
  (in proposal.tex, as defined as part of the \begin{workpackage}"
  command.
  
  Take into account person months associated with work package 1, time
  of all staff to be hired and work on the project (including
  investigators). Figure 5 helps with a quick check of the sums over
  different work packages.

- [ ] completed site specific resource summary in resources.tex,
  including table of non-staff costs. This is compulsory (EU
  regulations) if the non-staff cost exceed 15% of the total cost, and
  is likely to be the case for most of the partners. We ask everybody
  to do it, to be consistent and show transparently how we have
  planned our total budget.

- [ ] Have all our tasks a designated lead institution? Check in the
  Work Packages that all the tasks you are involved in have a
  dedicated lead party. If the lead party is "USO", then use:
  \begin{task}[lead=USO]

- [ ] Have all our deliverables a designated lead institution [using
  the 'lead=' key]?

- [ ] In the "Members of the consortium section", have we addressed "a
  description of the legal entity and its main tasks, with an
  explanation of how its profile matches the tasks in the
  proposal"? See Entry for Paris Sud and Southampton as examples.

- [X] In the Members of the consortium section, have we given
  descriptions of all the people we intend to hire (even if we don't
  know who that is yet). 
  
- [ ] Do all our tasks include us in the list of sites involved?
\end{verbatim}
\end{draft}

%KEY-MORE-TODOS
