\begin{sitedescription}{USO}


% PIC: 
% see: http://ec.europa.eu/research/participants/portal/desktop/en/orga

% See ../proposal.tex, section Members of the Consortium for a
% complete description of what should go there


The University of Southampton (UoS) is one of the leading universities
in the United Kingdom, was founded in 1952 and is a member of
prestigious Russell Group of UK Universities. UoS has more than 19,000
undergraduate students and 4,000 postgraduates and is an excellent
venue for conducting cutting-edge research and for providing high
quality education. The university is truly international, drawing
students from over 130 different countries and benefiting from a wide
and varied culture. It is ranked in the top 1\% of universities
worldwide (QS world university rankings 2014-15) and in the top 15 of
research led universities in the UK, and is participating in a high
number of collaborative research projects and related initiatives. UoS
has a successful track record of industrial collaborations and is at
the centre of a cluster of local high technology companies. It has an
enviable track record in the generation of patentable work, with a
portfolio of over 350 patents. To ensure the impact of its research
projects, University of Southampton’s Research \& Innovation Services
(R\&IS) is responsible for professional protection of IP and supporting
commercial development with industry. R\&IS has had considerable
success, licensing annual revenue in excess of \EUR{1}million and launching
twelve successful spin-out companies since 2000.  UoS has a strong
track record of working in European projects, especially within the
Framework Programme. The EC 6th FP7 Monitoring Report ranked UoS 17th
out of all higher and secondary education organisations for number of
FP7 participations during 2007-2012. Throughout the FP7 UoS has
received \EUR{132M} in research grants and has been involved in 319
projects, including 63 ICT and 8 INFRASTRUCTURES Collaborative Projects. In 2013/14
alone UoS has received over \euro181.5M in research grants and contracts,
including over \euro16.3M from the European Commission.

The Faculty of Engineering and the Environment (FEE) is one of the
lead engineering faculties in Europe, educating a range of
professionals and generating research of the highest
quality. Southampton's world-leading engineering ranking is confirmed
by being ranked first in the UK for the volume and quality of the
research in Electronic Engineering, Electrical Engineering and
General Engineering in the latest Research Excellence Framework (REF)
2014.

FEE brings together a wide range of disciplines, offering
undergraduate and postgraduate programmes in audiology and
environmental science as well as acoustical, civil and environmental,
mechanical, and aeronautical/astronautical engineering and ship
science. It consists of 370 research postgraduate students and 340
academic and research staff. FEE also hosts the University Technology
Centres and Research Framework Agreements with key partners including:
Airbus, Rolls-Royce, Lloyd’s Register, Microsoft and Network Rail. FEE
has a strong background in working on international research projects,
including 84 EU FP7 projects worth over \euro 28M.  In 2013/14 only
FEE has received about \euro50M in research grants and contracts, of
which over \euro1.7M from EU funding programmes.



\subsubsection*{Curriculum vitae}

% Curriculum of the personnel at this institution

\begin{participant}{Hans Fangohr}
  Hans Fangohr is Professor of Computational Modelling at the University of
  Southampton. He has studied Physics with specialisation in Computer Science and Applied
  Mathematics, gained his PhD in High Performance Computing (2002) and has since work in
  an Engineering department, carrying out interdisciplinary work based on computational
  science.

  He heads the University's interdisciplinary Computational Modelling Group
  (\url{http://cmg.soton.ac.uk}), and has more than 100 publications on applied computer
  simulation in magnetism, superconductivity, biochemistry, astrophysics and aircraft
  design, as well as development of computational methods.


  In 2013, he has attracted \EUR{5}m from the UK's Engineering and Physical Sciences
  Research Council (EPSRC) together with additional moneys from industry and his
  University to fund the \EUR{12}m Centre for Doctoral Training in Next Generation
  Computational Modelling (ngcm.soton.ac.uk) in the UK. This flagship activity will train
  about 75 PhD students (10 to 15 starting every year, beginning in September 2014) in the
  state-of-the-art and best-practice in computational modelling, the programming of
  existing and emerging parallel hardware and to apply these skills and tools to PhD
  research projects across a range of topics from Science and Engineering. The centre has
  chosen IPython as a key tool to deliver this teaching, document and communicate
  computational exploration and drive reproducible computation to push for excellent
  computational science.

  Hans Fangohr has led the development of the Open Source Nmag software
  (http://nmag.soton.ac.uk), which prodives a finite-element micromagnetic simulation
  suite to a community of material scientists, engineers and physicists who research
  magnetic nanostructures in academia and industry. He has designed the package in 2005 so
  that it has an IPython-compatible Python interface, to make the workflow of using the
  simulation package as accessible as possible to scientists without substantial
  computational background. He has extensive experience in micromagnetic simulation use
  and development.

  Hans has deep interested in excellence and innovation in learning and teaching. He has
  been awarded the prestigious Vice Chancellor’s teaching award ($\pounds 1000$) three
  times (in 2006, 2010, 2013) for initiating and realising three separate innovations in
  the university's teaching delivery, and has been voted ``best lecturer'' and ``funniest
  lecturer'' of the year by the students. Other Universities in the UK and elsewhere have
  adopted his teaching methods and materials. He has attracted grants to further develop
  learning and teaching activities, and given invited talks at international meetings on
  efficient learning and teaching of computational methods.
 
  Hans Fangohr is chairing the UK's national Scientific Advisory Committee for High
  Performance Computing.
\end{participant}

%%% Local Variables:
%%% mode: latex
%%% TeX-master: "../proposal"
%%% End:


%\input{CVs/First.Last.tex}
%\input{CVs/First.Last.tex}

\subsubsection*{Publications, products, achievements}

\begin{enumerate}
\item Open Source micromagnetic simulation framework Nmag,
  \href{http://nmag.soton.ac.uk}{http://nmag.soton.ac.uk}, Thomas
  Fischbacher, Matteo Franchin, Giuliano Bordignon, Hans Fangohr: \emph{
A Systematic Approach to Multiphysics Extensions of Finite-Element-Based Micromagnetic Simulations: Nmag 
IEEE Transactions on Magnetics \textbf{43}, 6, 2896-2898 (2007)}
\item Other open source contributions to the micromagnetic simulation
  community: OVF2VTK, higher order anisotropy extensions to OOMMF,
  OVF2MFM, summarised at
  \href{http://www.southampton.ac.uk/~fangohr/software/index.html}{http://www.southampton.ac.uk/~fangohr/software/index.html} 
\item H. Fangohr.
\emph{A Comparison of C, Matlab and Python as Teaching Languages in Engineering}
Lecture Notes on Computational Science \textbf{3039}, 1210-1217 (2004)
\end{enumerate}

\subsubsection*{Previous projects or activities}

\begin{enumerate}
\item EPSRC Doctoral Training Centre in Complex Systems Simulations
  (\href{http://icss.soton.ac.uk}{http://icss.soton.ac.uk}), jointly
funded by EPSRC and the University of Southampton, \EUR{14m}, (2009--2018)
\end{enumerate}

\subsubsection*{Significant infrastructure}
\begin{enumerate}
\item The University of Southampton hosts the largest university owned
  Supercomputer "Iridis 4" in the UK (12,300 cores, 250 TFlops),
  the hardware (\EUR{3.75}m) is refreshed every 3 years.
\item A community of 200 academics and over 500 researchers and
  doctoral students are users of this facility and provide a wide
  network pushing forward excellent computational science in the
  context of solving real world problems.
\item EPSRC Centre for Doctoral Training in Computational Modelling in
  the United Kingdom,
  \EUR{12}m. (\href{http://ngcm.soton.ac.uk}{http://ngcm.soton.ac.uk}),
  (2013--2022)
\end{enumerate}
\end{sitedescription}
%%% Local Variables:
%%% mode: latex
%%% TeX-master: "../proposal"
%%% End:
