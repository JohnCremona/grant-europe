\begin{sitedescription}{USO}

% PIC: 
% see: http://ec.europa.eu/research/participants/portal/desktop/en/orga

% See ../proposal.tex, section Members of the Consortium for a
% complete description of what should go there


  The University of Southampton (USO) is one of the leading
  universities in the United Kingdom, was founded in 1952 and is a
  member of prestigious Russell Group of UK Universities. The
  University of Southampton has more than 19,000 undergraduate
  students and 4,000 postgraduates and is an excellent venue for
  conducting cutting-edge research and for providing high quality
  education. The university is truly international, drawing students
  from over 130 different countries and benefiting from a wide and
  varied culture. It is ranked in the top 1\% of universities
  worldwide (QS world university rankings 2014-15) and in the top 15
  of research led universities in the UK, and is participating in a
  high number of collaborative research projects and related
  initiatives. To ensure the impact of its research projects,
  University of Southampton’s Research \& Innovation Services (R\&IS)
  is responsible for professional protection of IP and supporting
  commercial development with industry. R\&IS has had considerable
  success, licensing annual revenue in excess of \EUR{1}million and
  launching twelve successful spin-out companies since 2000.  The
  university has a strong track record of working in European
  projects, especially within the Framework Programme. The EC 6th FP7
  Monitoring Report ranked USO 17th out of all higher and secondary
  education organisations for number of FP7 participations during
  2007-2012. Throughout the FP7 USO has received \EUR{132M} in
  research grants and has been involved in 319 projects, including 63
  ICT and 8 INFRASTRUCTURES Collaborative Projects. 

  The Faculty of Engineering and the Environment (FEE) consists of 370
  research postgraduate students and 340 academic and research
  staff, and is one of the lead engineering faculties in Europe, educating
  a range of professionals and generating research of the highest
  quality. Southampton's world-leading engineering ranking is
  confirmed by being ranked first in the UK for the volume and quality
  of the research in Electronic Engineering, Electrical Engineering
  and General Engineering in the latest Research Excellence Framework
  (REF) 2014. The faculty also hosts the University Technology Centres
  and Research Framework Agreements with key partners including:
  Airbus, Rolls-Royce, Lloyd’s Register, Microsoft and Network
  Rail. FEE has a strong background in working on international
  research projects, including 84 EU FP7 projects worth over \euro
  28M.


\medskip In the context of this proposal, Southampton has long
standing experience in high performance computer simulation to advance
science and engineering, with significant hardware infrastructure, a
critical mass of several hundred researchers working in the field, and
a dedicated doctoral training center in computational modelling.
Southampton's main tasks in this project are the extension and application of the
\Jupyter{} Notebook technology in computational materials research to
provide a virtual research environment demonstrator for a large research community.

\subsubsection*{Curriculum vitae}

% Curriculum of the personnel at this institution

\begin{participant}{Hans Fangohr}
  Hans Fangohr is Professor of Computational Modelling at the University of
  Southampton. He has studied Physics with specialisation in Computer Science and Applied
  Mathematics, gained his PhD in High Performance Computing (2002) and has since work in
  an Engineering department, carrying out interdisciplinary work based on computational
  science.

  He heads the University's interdisciplinary Computational Modelling Group
  (\url{http://cmg.soton.ac.uk}), and has more than 100 publications on applied computer
  simulation in magnetism, superconductivity, biochemistry, astrophysics and aircraft
  design, as well as development of computational methods.


  In 2013, he has attracted \EUR{5}m from the UK's Engineering and Physical Sciences
  Research Council (EPSRC) together with additional moneys from industry and his
  University to fund the \EUR{12}m Centre for Doctoral Training in Next Generation
  Computational Modelling (ngcm.soton.ac.uk) in the UK. This flagship activity will train
  about 75 PhD students (10 to 15 starting every year, beginning in September 2014) in the
  state-of-the-art and best-practice in computational modelling, the programming of
  existing and emerging parallel hardware and to apply these skills and tools to PhD
  research projects across a range of topics from Science and Engineering. The centre has
  chosen IPython as a key tool to deliver this teaching, document and communicate
  computational exploration and drive reproducible computation to push for excellent
  computational science.

  Hans Fangohr has led the development of the Open Source Nmag software
  (http://nmag.soton.ac.uk), which prodives a finite-element micromagnetic simulation
  suite to a community of material scientists, engineers and physicists who research
  magnetic nanostructures in academia and industry. He has designed the package in 2005 so
  that it has an IPython-compatible Python interface, to make the workflow of using the
  simulation package as accessible as possible to scientists without substantial
  computational background. He has extensive experience in micromagnetic simulation use
  and development.

  Hans has deep interested in excellence and innovation in learning and teaching. He has
  been awarded the prestigious Vice Chancellor’s teaching award ($\pounds 1000$) three
  times (in 2006, 2010, 2013) for initiating and realising three separate innovations in
  the university's teaching delivery, and has been voted ``best lecturer'' and ``funniest
  lecturer'' of the year by the students. Other Universities in the UK and elsewhere have
  adopted his teaching methods and materials. He has attracted grants to further develop
  learning and teaching activities, and given invited talks at international meetings on
  efficient learning and teaching of computational methods.
 
  Hans Fangohr is chairing the UK's national Scientific Advisory Committee for High
  Performance Computing.
\end{participant}

%%% Local Variables:
%%% mode: latex
%%% TeX-master: "../proposal"
%%% End:


\begin{participant}[PM=2,type=PI]{Ian Hawke}
%
Ian Hawke is a lecturer in Applied Mathematics at the University of
Southampton and a co-director of the \EUR{12}m EPSRC Centre for Doctoral
Training in Next Generation Computational Modelling. An expert in
nonlinear simulations of relativistic matter and numerical techniques,
he has taught numerical methods in many contexts for ten years. He has
worked on IPython (\Jupyter{}) Notebooks in education, particularly as an
instructor on the ``Practical Numerical Methods in Python'' MOOC,
which builds on other open technologies including OpenEdX and github. The
initial author of the ``Whisky'' relativistic hydrodynamics code, he has
been a contributor to and maintainer of a range of projects used
across the numerical relativity community, including the Einstein
Toolkit, the Cactus infrastructure and the Carpet mesh refinement
code. His recent research has concentrated on numerical methods for
relativistic matter beyond ideal fluids, including modelling sharp
transitions and surfaces, relativistic elasticity, and the first
nonlinear simulations of relativistic multifluids.
%
\end{participant}

%%% Local Variables:
%%% mode: latex
%%% TeX-master: "../proposal"
%%% End:


\begin{participant}[type=R, PM=32]{NN}
  We will hire a post-doctoral senior research fellow to carry out the work
  at Southampton, under the leadership of and together with Hans
  Fangohr. The fellow will have a background in computational science,
  ideally in micromagnetics, combined with solid IPython and
  \Jupyter{} Notebook experience, and past experience of software
  engineering. We further require good communication and team working
  skills, and in particular interest and skill in the development of
  education materials to best support this part of the project.
\end{participant}
%\input{CVs/First.Last.tex}

\subsubsection*{Publications, products, achievements}

\begin{enumerate}
\item Open Source micromagnetic simulation framework Nmag,
  \href{http://nmag.soton.ac.uk}{http://nmag.soton.ac.uk}, Thomas
  Fischbacher, Matteo Franchin, Giuliano Bordignon, Hans Fangohr: \emph{
A Systematic Approach to Multiphysics Extensions of Finite-Element-Based Micromagnetic Simulations: Nmag 
IEEE Transactions on Magnetics \textbf{43}, 6, 2896-2898 (2007)}
\item Other open source contributions to the micromagnetic simulation
  community: OVF2VTK, higher order anisotropy extensions to OOMMF,
  OVF2MFM, summarised at
  \href{http://www.southampton.ac.uk/~fangohr/software/index.html}{http://www.southampton.ac.uk/~fangohr/software/index.html} 
\item H. Fangohr.
\emph{A Comparison of C, Matlab and Python as Teaching Languages in Engineering}
Lecture Notes on Computational Science \textbf{3039}, 1210-1217 (2004)
\end{enumerate}

\subsubsection*{Previous projects or activities}

\begin{enumerate}
\item EPSRC Doctoral Training Centre in Complex Systems Simulations
  (\href{http://icss.soton.ac.uk}{http://icss.soton.ac.uk}), jointly
funded by EPSRC and the University of Southampton, \EUR{14m}, (2009--2018)
\end{enumerate}

\subsubsection*{Significant infrastructure}
\begin{enumerate}
\item The University of Southampton hosts the largest university owned
  Supercomputer "Iridis 4" in the UK (12,300 cores, 250 TFlops),
  the hardware (\EUR{3.75}m) is refreshed every 3 years.
\item A community of 200 academics and over 500 researchers and
  doctoral students are users of this facility and provide a wide
  network pushing forward excellent computational science in the
  context of solving real world problems.
\item EPSRC Centre for Doctoral Training in Computational Modelling in
  the United Kingdom,
  \EUR{12}m. (\href{http://ngcm.soton.ac.uk}{http://ngcm.soton.ac.uk}),
  (2013--2022)
\end{enumerate}
\end{sitedescription}



%KEY-MORE-TODOS



%%% Local Variables:
%%% mode: latex
%%% TeX-master: "../proposal"
%%% End:

%  LocalWords:  sitedescription Programme organisations programmes Centres subsubsection
%  LocalWords:  micromagnetic Nmag Fischbacher Franchin Bordignon Fangohr emph textbf
%  LocalWords:  Multiphysics summarised Iridis TFlops Modelling
