\begin{sitedescription}{UO}

% PIC: 999984350 THE CHANCELLOR, MASTERS AND SCHOLARS OF THE UNIVERSITY OF OXFORD OXFORD UK GB12550673

% See proposal.tex, Members of the Consortium for a complete description of what should go there

The University of Oxford is in the top ten universities worldwide in the Shanghai 2013 and 2014
rankings (the Universities of Cambridge and Oxford are the only non-US university there).
It employs over 10,000 staff and has a student population of over 21,000.
Most staff are directly appointed and managed by one of the University’s 130 departments or
other units within a highly devolved operational structure - this includes 5,900 ‘academic related’
staff (postgraduate research, computing, senior library, and administrative staff) and
2,820 ‘support’ staff (including clerical, library, technical, and manual staff). There are also
over 1,600 academic staff (professors, readers, lecturers), whose appointments are in the
main overseen by a combination of broader divisional and local faculty board/departmental
structures. 

The Department of Computer Science is one of the longest established
Computer Science departments in the country. Formerly known as the
University Computing Laboratory, it is home to a community of world-class research and
teaching. Research activities encompass core Computer Science, as well as Security,
Algorithms, computational biology, quantum computing, computational linguistics,
information systems, software verification and software engineering. 

The department is home to undergraduates, full-time and part-time Master's
students, and has a strong doctoral programme.  Students are highly skilled and
motivated, and as practice shows, easily start contributing to open-source
projects such as Sagemath.

\subsubsection*{Curriculum vitae}

\paragraph{Ursula Martin.}

% months=2.4
%
% Fair evaluation of the number of months you will be spending on this
% specific project along the four years

% salary=XXX
%
Professor Ursula Martin has  recently joined the University of Oxford, where
she holds a Professorship, in conjunction with a Senior EPSRC Fellowship, on a
joint arrangement between  the Department of Computer Science and the
Mathematical Institute. Her current  research concerns social and computational
techniques for creating mathematics, building on a significant track record at
the interface of mathematics and computing. Prior to this she worked at  Queen
Mary University of London, where as Vice Principal for Science and Engineering
she led strategic change, and was active in knowledge transfer
activities and developing young staff. 

Her work is characterized by strongly interdisciplinary collaboration in new
problem domains at the interface of mathematics and computer science,
identifying novel interactions between theory and practice, with real-world
problems inspiring scientific advance. Major achievements include results
linking randomness and symmetry, new unifying explanations of the power of
computational logic, and new practical techniques for using computational logic
and algebra in industry.

The work to be undertaken in the Work Package 5 (Social Aspects) fits very well
into her current project, which concerns crowdsourced mathematics: the overarching goal is
to understand and extend the human and computer creation of mathematics. 
It is mostly funded by her
2014 EPSRC Advanced Fellowship (EPSRC awards only one or two of these annually
in Computer Science) is a partnership of industry, government and international
academia.  

\begin{participant}[PM=2,salary=8000]{Edith Elkind}
  is an Assoiciate Professor at the Department of Computer Science of the University of
  Oxford. She is a leading expert in algorithmic game theory, computational social choice
  and voting theory, and in multiagent systems. In 2014 she received an ERC Starting Grant
  for the project ``Algorithms for Complex Collective Decisions on Structured
  Domains''. The settings considered in the present proposal provide a natural application
  domain for tools to be developed in the ERC project.

  Edith holds a PhD (2005) from Princeton; before coming to Oxford in 2013 she held a
  Singapore National Research Foundation Fellowship totalling \euro 1.6M, graduating 3 PhD
  students and supervising a number of postdoctoral researchers. She chaired/is chairing
  program committees of major international conferences, such as AAMAS, and is an
  associate editor of a number of major journals, such as Artificial Intelligence Journal.
\end{participant}
%%% Local Variables:
%%% mode: latex
%%% TeX-master: "../proposal"
%%% End:

\paragraph{Dmitrii Pasechnik}

% months=24
%
% Fair evaluation of the number of months you will be spending on this
% specific project along the four years

% salary=7000
%
% Approximate monthly gross salary (in term of total cost for the
% employer). If you are uncomfortable having this information in a
% public file, you can alternatively send the information to Nicolas,
% or to your institution leader if the latter will be willing to fill
% in himself the budget forms on the eu portal.

% The above information will be used to evaluate the cost of the
% project for the institutions. You may remove the above comments once
% you have filled in the months= and salary= lines.

% About half a page of free text; for whatever it's worth, you may see
% Nicolas.Thiery.tex for an example.
is a Senior Research Fellow at the Department of Computer Science
of  the University of Oxford, where he also holds a Lectureship
at Pembroke College. Before moving to Oxford in 2013, he taught mathematics for 8
years in Nanyang Technological University (Singapore). While there, he
was successful in receiving individual grant funding totalling over 500K, 
graduated 2 PhD students, supervised post-doctoral researchers, 
and co-organized a 2-months research program
at Singapore Institute for Mathematical Sciences
on a range of topics in computational mathematics, involving over 100 participants. 

He works on a wide area of interconnected topics, related to computational algebra
and optimization, combinatorics, algorithm, symbolic computing, and game theory, and authored over
70 papers on these topics, several of them using Sage and/or its components, such as GAP.

He is an active Sage developer, and regularly contributes, himself or together
with his undergraduate or graduate students, new or improved Sage interfaces to
various mathematical packages and databases.
He has good understanding of the overall Sage development process, as well as 
of development of other open-source sofware and databases, including their
social/community aspects. 


\subsubsection*{Publications, products, achievements}
\begin{enumerate}
\item U. Martin. Computational logic and the social. J.Logic \& Computation (2014) [doi:10.1093/logcom/exu036]
\item U. Martin, A. Pease. Mathematical Practice, Crowdsourcing, and Social Machines, in Springer LNCS vol. 7961 (2013) [doi:10.1007/978-3-642-39320-4\_7]
\item G. Chalkiadakis, E. Elkind, M. Wooldridge. Computational Aspects of Cooperative Game Theory, Morgan \& Claypool, 2011 [doi:10.2200/S00355ED1V01Y201107AIM016]
\item E. Elkind, D. Pasechnik, Y. Zick. Dynamic weighted voting games, in Proc. AAMAS 2013 [http://dl.acm.org/citation.cfm?id=2484920.2485003]
\item S.H. Chan, H.D.L. Hollmann, D. Pasechnik. Sandpile groups of generalized de Bruijn and Kautz graphs and circulant matrices over finite fields,
J. Algebra 421(2015) [doi:10.1016/j.jalgebra.2014.08.029]
\end{enumerate}

\subsubsection*{Previous projects or activities}

\begin{enumerate}
\item % {Projects and activities in Oxford} 
U. Martin holds an EPSRC Senior Fellowship, 2014-2017, to study the social machine of mathematics.
\item E. Elkind will hold an ERC Starter Grant, 2015-2020, to study and develop algorithms for collective decision
making on structured domains.
\item D. Pasechnik supervised  students contributing,  and significantly contributes himself, to Sagemath, to 
OEIS, and to GAP.
\end{enumerate}

\subsubsection*{Significant infrastructure}
Oxford has world-class computational facilities, including numerous HPC clusters and a dedicated
centre, Advanced Research Computing, to support HPC users.
Another dedicated facility, Oxford's e-Research Centre, facilitates 
digital research and drives innovation in technology. Last but not the least,
the library of Oxford University is one of the most complete libraries in the UK.
\end{sitedescription}




\begin{draft}
\vspace{1cm}\TOWRITE{DP}{Complete check list below -- delete completed items if you wish}

\begin{verbatim}
- [X] checked that sum of person months put into finance request is
  the same as sum of person months associated with the Work Packages
  (in proposal.tex, as defined as part of the \begin{workpackage}"
  command.
  
  Take into account person months associated with work package 1, time
  of all staff to be hired and work on the project (including
  investigators). Figure 5 helps with a quick check of the sums over
  different work packages.

- [ ] completed site specific resource summary in resources.tex,
  including table of non-staff costs. This is compulsory (EU
  regulations) if the non-staff cost exceed 15% of the total cost, and
  is likely to be the case for most of the partners. We ask everybody
  to do it, to be consistent and show transparently how we have
  planned our total budget.

- [ ] Have all our tasks a designated lead institution? Check in the
  Work Packages that all the tasks you are involved in have a
  dedicated lead party. If the lead party is "USO", then use:
  \begin{task}[lead=USO]

- [ ] Have all our deliverables a designated lead institution [using
  the 'lead=' key]?

- [ ] In the "Members of the consortium section", have we addressed "a
  description of the legal entity and its main tasks, with an
  explanation of how its profile matches the tasks in the
  proposal"? See Entry for Paris Sud and Southampton as examples.

- [ ] In the Members of the consortium section, have we given
  descriptions of all the people we intend to hire (even if we don't
  know who that is yet). 
\end{verbatim}
\end{draft}

%KEY-MORE-TODOS


%%% Local Variables:
%%% mode: latex
%%% TeX-master: "../proposal"
%%% End:

%  LocalWords:  sitedescription programme Sagemath subsubsection Pease Chalkiadakis Zick
%  LocalWords:  Elkind Wooldridge Claypool Pasechnik Hollmann Bruijn Kautz TOWRITE
