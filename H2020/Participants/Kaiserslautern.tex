\begin{sitedescription}{UK}
% See proposal.tex, Members of the Consortium for a complete description of what should go there
The University of Kaiserslautern (UK) is a medium sized university founded in 1970. It currently 
consists of 12 departments, ranging from mathematics and business studies and economics, 
computer sciences and electrical and computer technology over mechanical and process engineering, 
biology, chemistry and physics to architecture, regional and environmental planning, and social sciences. 
The university has 13,725 students, 3636 of whom are remote study students. The scientific location 
of Kaiserslautern is also distinguished by the presence of multiple external research institutes of 
considerable reputation, including two facilities of the Fraunhofer network and the German Research 
Institution for Artificial Intelligence. All these institutions maintain close links and even share staff 
with the corresponding departments of UK, which is chairing the Science Alliance Kaiserslautern, 
a network of these research institutions. The university conducts a number of international 
collaborations and successfully participated in projects funded under several EU Framework 
Programs and has gathered comprehensive experience both as coordinator and partner in research 
networks and projects. Besides projects with national funding, the UK is 
also very active in the field of international research. In this context, the funding instruments 
available in the EU Framework Programmes play an important role. In total, the UK is partner 
to a total of 11 projects (as of January 2015) conducted under the 7th FP and Horizon 2020. 
Nine further individual projects funded by ERC (2) or Marie-Curie measures (7) are being co-ordinated 
by researchers at UK. By this involvement to date, UK has procured more than 13 million Euros under the 7th FP.

\medskip In the context of this proposal, the
\emph{Algebra, Geometry, and Computer Algebra Group}  of the Department
of Mathematics at UK is widely known for its long tradition in 
computational algebraic geometry and algebra, with particular emphasis on the 
development of the computer algebra system \Singular and its satellites and  
subsystems such as {\sc{Factory}}, {\sc{PolyBori}}, and {\sc{Plural}}.
Kaiserslautern's main tasks in this project are to add very fine-grained 
parallelism to some key components of {\sc{Singular}} and to work
on the maintainance and improvement of \MPIR.

\subsubsection*{Curriculum vitae}

\paragraph{Principal investigator Prof. Dr. Wolfram Decker}

Wolfram Decker is a professor of mathematics at TU Kaiserslautern.
He formerly was a research fellow at Berkeley with a NATO grant,
a visiting researcher at Kyoto with a JSPS grant, and a professor
at Saarbr\"ucken, Germany. Decker has more than thirty publications
including two books on computational algebraic geometry and papers 
in Compositio, Crelle, and Mathematische Annalen. He has held several 
grants in four different priority programmes of the German Research 
Council DFG and is now coordinator of the
priority programme SPP 1489 ``Algorithmic and Experimental 
Methods in Algebra, Geometry, and Number Theory''. He was also
coordinator of the European algebraic geometry network
EuroProj (1996--1999) and Chair of the programme management 
committee of the European algebraic geometry network EAGER
(2000--2004). He held seven grants for EU Highlevel Scientific 
Conferences and (co-)organized about 50 conferences, summer 
schools, workshops, and coding sprints. He was Chair of 
the Minisymposium on Computer Algebra during the third ECM.
Decker has supervised 13 PhD students. He has been a frequent
lecturer at the African Institute of Mathematics (AIMS) at
Cape Town, and he has run 8 schools on computational
algebraic geometry in different countries.

Decker's research interests lie in areas of algebraic geometry and 
computer algebra. In addition to writing theoretical papers, he is
a leader in mathematical software development and has written thousands of 
lines of code himself. He has made contributions to the systems 
{\sc{Macaulay2}} and, much more substantially, {\sc{Singular}}. 
Since 2009 he is the head of the {\sc{Singular}} development team.
Current tasks of the team include crosslinking {\sc{Singular}} to
other systems, most notably to {\sc{GAP}}, and parallelizing
{\sc{Singular}}. These tasks are fundamental to the
\textbf{MathVRE} project.







\begin{participant}{Dr. William Hart}

William Hart is a postdoctoral researcher at the University of
Kaiserslautern. He is the lead developer of the Flint and MPIR projects
as well as the main author and maintainer of the BSDNT bignum library, the
Nemo and ANTIC libraries and a contributor to various other software packages.

Before coming to Kaiserslautern, held a prestigious five year Career
Acceleration Fellowship ``Algorithms in Algebraic Number Theory'' at Warwick
University in the UK. He has been involved in a number of high performance
computing records, including computation of congruent numbers (subject to the
BSD conjecture) up to a trillion ($10^{12}$).

William is the main author of the FFT code for multiplication of large integers
and polynomials in both MPIR and Flint, which are used extensively by the Sage,
Singular and Macaulay 2 computer algebra systems.

The main focus of William's research has been in algorithms for fast arithmetic, fast integer and polynomial factorisation and to algebraic number theory,
including computation of modular equations and class invariants.
\end{participant}
%%% Local Variables:
%%% mode: latex
%%% TeX-master: "../proposal"
%%% End:


\begin{participant}[type=res,PM=48]{NN}
We will hire a full time experienced software developer to work
  under the leadership of Wolfram Decker on adding very fine-grained 
  parallelism to some key components of {\sc{Singular}}. The fellow
  will have past experience of parallism in software development.
  We further require good communication and team working
  skills.
\end{participant}

\begin{participant}[type=res,PM=12]{NN}
  We will hire a full time highly specialized software developer and
  assembly expert, to work under the leadership of William Hart on the
  performance task \taskref{hpc}{hpc-mpir} for \MPIR.
\end{participant}

\subsubsection*{Publications, products, achievements}

\begin{enumerate}
\item {\sc{Singular}} computer Algebra system.
\item Wolfram Decker is coordinator of the DFG Priority Project SPP1489 \emph{Algorithmic and Experimental Methods in Algebra, Geometry, and
Number Theory'}.
\item {\sc{Flint and MPIR}} C libraries for number theory and bignum arithmetic.
\item William Hart held an EPSRC Career Acceleration Fellowship EP/G004870/1 
from 2008-2013, \emph{Algorithms in Algebraic Number Theory}
\end{enumerate}

\subsubsection*{Previous projects or activities}

\begin{enumerate}
\item Member of the DFG Priority Project \emph{Algorithmic Number Theory and Algebra}.
\end{enumerate}

\subsubsection*{Significant infrastructure}

Excellent computing infrastructure (high end servers), access to 
different types of compute clusters through the IT-Center of the 
TU Kaiserslautern.
\end{sitedescription}



%KEY-MORE-TODOS



%%% Local Variables:
%%% mode: latex
%%% TeX-master: "../proposal"
%%% End:

%  LocalWords:  sitedescription subsubsection sc emph bignum
