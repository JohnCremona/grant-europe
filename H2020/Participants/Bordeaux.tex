\begin{sitedescription}{UB}
% PIC: 
% see: http://ec.europa.eu/research/participants/portal/desktop/en/orga

% See ../proposal.tex, section Members of the Consortium for a
% complete description of what should go there

\begin{itemize}
\item INRIA, LaBRI, IMB
\item journal de th\'eorie des nombres
\item Plafrim and a mesocentre Avakas
\item Several softwares developped in Bordeaux: pari/GP, tulip, etc
\end{itemize}


\subsubsection*{Curriculum vitae}

% Curriculum of the personnel at this institution

\begin{participant}[type=leadPI,PM=12,salary=4700,gender=male]{Vincent Delecroix}
CNRS researcher at the LaBRI (Bordeaux, France) since october 2013, Vincent
Delecroix is a junior researcher in Dynamical Systems with strong links with
Combinatorics and Number Theory. He published 7 articles in international
journals with several collaborators around the world (England, Mexico,
United-States).

Since 2010 he is an important contributor of Sage with 30 tikets authored and
around 50 reviewed. He organized several Sage days and Sage workshops in
Bordeaux, Marseille, Orsay, Perpignan, Bobo Dioulasso (Burkina Faso),
Saint-Louis (S\'en\'egal).
\end{participant}
%%% Local Variables:
%%% mode: latex
%%% TeX-master: "../proposal"
%%% End:


%\begin{participant}[type=R,PM=6,gender=male]{Adrien Boussicault}
  Maître de Conférences at the LaBRI (Laboratoire Bordealais de Recherche en 
  informatique), Adrien Boussicault is a young researcher in Algebraic and 
  Enumerative Combinatorics. He has 8 papers in international journals. 
  His contributions to \Sage include writing 3 tickets to implement 
  combinatorial objects and co-organising \SageCombinat Days 57.
\end{participant}
%%% Local Variables:
%%% mode: latex
%%% TeX-master: "../proposal"
%%% End:


\begin{participant}[type=PI,PM=12,salary=7500,gender=male]{Karim Belabas}
  Professor of Mathematics in Bordeaux (France) since 2005, Karim Belabas is
a senior researcher in Number Theory, with particular interest in
computational and effective aspects. Karim has published about 25 articles
in international journals, including papers in Duke and Compositio (one of
which co-authored with Manjul Bhargava, 2014 Fields Medal), and
edited a book on ``Explicit methods in number theory''.

Karim was head of the Pure Math teaching department in Bordeaux from 2010 to
2014 and is vice-head of the Institut de Math\'ematiques de Bordeaux since
2015. He has held a grant from the French ANR worth $200$ kEuros (ALGOL
project, 2007--2011) and was part of a 2.5MEuros Marie-Curie Research
Training Network (GTEM project 2006--2010); he was responsible for three
deliverables and supervision of an early stage researcher during her PhD in
the Work Package ``Constructive Galois Theory''. He has (co-)organized 8
international conferences, including a special Trimester at IHP in 2004 and
an influential recurrent workshop on ``Explicit methods'' in Oberwolfach
(every two years since 2007) and 5 \PariGP Ateliers. He has (co-)supervised
11 PhD students and about 15 masters students.

Karim is a leading computational number theorist in France. He
is the project leader for the \PariGP free computer algebra system since
1995, which has had a major impact on hundreds of publications. He is one of
the system's main developers (about 60000 lines of code written, most of the
documentation, and 1300 bug-tracking tickets authored).
\end{participant}
%%% Local Variables:
%%% mode: latex
%%% TeX-master: "../proposal"
%%% End:

\begin{participant}[type=PI,PM=6,gender=male]{Bill Allombert}
is a research engineer in Bordeaux. He has a great expertise in
algorithmic number theory and is one of the main developer of \PariGP.
He is the developer of the \software{GP2C} compiler to convert \software{GP} code into
\software{pari} code, and the \software{GALPOL} database (database of polynomial with
prescribed Galois groups).

He also has 5 articles in international journals and is 
a regular contributor of the software \GAP and the \software{Debian} project.
\end{participant}

%%% Local Variables:
%%% mode: latex
%%% TeX-master: "../proposal"
%%% End:

%?? others ??


\subsubsection*{Publications, products, achievements}

Some recent Publications :
\begin{enumerate}
\item 
Belabas, Karim; Friedman, Eduardo; Computing the residue of the Dedekind
zeta function.  Math. Comp. 84 (2015), no. 291, 357–369. 

\item
The PARI Group; PARI/GP version 2.7.0, Bordeaux, 2014,
http://pari.math.u-bordeaux.fr/.

\item
Belabas, Karim et al. Explicit methods in number theory. Rational points and
Diophantine equations, 179 pages, Panoramas et Synthèses 36, 179p., 2012.

\item
Allombert, Bill; Bilu, Yuri and Pizarro-Madariaga, Amalia;) CM-Points on
Straight Lines , to appear in "Analytic Number Theory" (dedicated do H. Maier),
Springer. 
\end{enumerate}



\subsubsection*{Previous projects or activities}

Current grants:
\begin{enumerate}
\item
 ANR PEACE (2012-2015)
    Goal: The discrete logarithm problem on algebraic curves is one of the rare
    contact points between deep theoretical questions in arithmetic geometry and
    every day applications. On the one side it involves a better understanding,
    from an effective point of view, of moduli space of curves, of abelian
    varieties, the maps that link these spaces and the objects they classify.
    On the other side, new and efficient algorithms to compute the discrete
    logarithm problem would have dramatic consequences on the security and
    efficiency of already deployed cryptographic devices. 

\item
ERC starting grant ANTICS (2011-2016) 
    Goal: "Rebuild algorithmic number theory on the firm grounds of theoretical
    computer science".
    Challenges: complexity (how fast can an algorithm be?), reliability
    (how correct should an algorithm be?), parallelisation.
\end{enumerate}

\subsubsection*{Significant infrastructure}
\TOWRITE{VD}{this still needs to be done}

Two center for computations: Plafrim and Avakas.
\end{sitedescription}

%%% Local Variables:
%%% mode: latex
%%% TeX-master: "../proposal"
%%% End:
