\begin{sitedescription}{UB}
% PIC: 
% see: http://ec.europa.eu/research/participants/portal/desktop/en/orga

% See ../proposal.tex, section Members of the Consortium for a
% complete description of what should go there

The Centre National de la Recherche Scientifique (National Center for
Scientific Research, CNRS) is a public organization under the responsibility of
the French Ministry of Education and Research. Its missions are
\begin{compactitem}
\item To evaluate and carry out research and
bringing social, cultural, and economic benefits for society.
\item To contribute to the application and promotion of research results.
\item To develop scientific information.
\item To support research training.
\item To participate in the analysis of the national and international
scientific climate in order to develop a
national policy.
\end{compactitem}
It consists of ten institutes in Biology, Humanities, Information Sciences and Technologies, etc. CNRS laboratories (or research units) are located throughout France, and employ a large body of tenured researchers, engineers, and support staff. 

Three research units are concerned by this proposal. The Laboratoire
Math\'ematiques d'Orsay (LMO) which is a joint unit with Universit\'e Paris-Sud
which is a partner of this proposal and has already been described
(see~\ref{desc:ParisSud}). Two laboratories in Bordeaux (FR), the Laboratoire
d'Informatique Bordelais (CNRS-LaBRI) and the Institut Math\'ematiques de
Bordeaux (CNRS-IMB). Those two units are joint with the Universit\'e de
Bordeaux which is a third party of this proposal
(see~\ref{section:ThirdParties}).

Bordeaux is an important center of studies and research in France
with approximately 50,000 students, 2,000 PhD students and 5,000
researchers. The University of Bordeaux was founded in the 15th
century and nowadays the city regroups two universities, dozens of
schools and 100 research laboratories with partner institution such
as CNRS, Inserm, INRA and INRIA.

The Institut Math\'emathiques de Bordeaux (CNRS-IMB) is a leading
institution in Number Theory. It is the home of the software
\PariGP and the Journal de Th\'eorie des Nombres de Bordeaux.

The city of Bordeaux also hosts two important young laboratories
for computer science: Laboratoire Bordelais d'Informatique (CNRS-LaBRI)
and INRIA-Bordeaux.

\medskip
In the context of this proposal, Bordeaux has a long standing experience in
Algorithmic Number Theory and two significant hardware infrastructures
(Plafrim and Avakas). The main task of Bordeaux in this project is the
extension of \PariGP in relation with the other software components and
the .

The Bordeaux site will lead the development of \PariGP within
workpackages \WPref{UI},\WPref{hpc}, \WPref{dissem}. It will also play
an important role in \WPref{hpc} with respect to combinatorics
and the organisation of workshop in favour of developing countries in \WPref{dissem}.

\subsubsection*{Curriculum vitae}
%Note that for purely administrative reasons, Loic Gouarin will be attached to
%CNRS but is naturally based at Paris sud - commented out, this is
%duplicated at Loic's CV (AK)

% Curriculum of the personnel at this institution

\begin{participant}[type=leadPI,PM=12,salary=4700,gender=male]{Vincent Delecroix}
CNRS researcher at the LaBRI (Bordeaux, France) since october 2013, Vincent
Delecroix is a junior researcher in Dynamical Systems with strong links with
Combinatorics and Number Theory. He published 7 articles in international
journals with several collaborators around the world (England, Mexico,
United-States).

Since 2010 he is an important contributor of Sage with 30 tikets authored and
around 50 reviewed. He organized several Sage days and Sage workshops in
Bordeaux, Marseille, Orsay, Perpignan, Bobo Dioulasso (Burkina Faso),
Saint-Louis (S\'en\'egal).
\end{participant}
%%% Local Variables:
%%% mode: latex
%%% TeX-master: "../proposal"
%%% End:


\begin{participant}[type=PI,PM=12,salary=7500,gender=male]{Karim Belabas}
  Professor of Mathematics in Bordeaux (France) since 2005, Karim Belabas is
a senior researcher in Number Theory, with particular interest in
computational and effective aspects. Karim has published about 25 articles
in international journals, including papers in Duke and Compositio (one of
which co-authored with Manjul Bhargava, 2014 Fields Medal), and
edited a book on ``Explicit methods in number theory''.

Karim was head of the Pure Math teaching department in Bordeaux from 2010 to
2014 and is vice-head of the Institut de Math\'ematiques de Bordeaux since
2015. He has held a grant from the French ANR worth $200$ kEuros (ALGOL
project, 2007--2011) and was part of a 2.5MEuros Marie-Curie Research
Training Network (GTEM project 2006--2010); he was responsible for three
deliverables and supervision of an early stage researcher during her PhD in
the Work Package ``Constructive Galois Theory''. He has (co-)organized 8
international conferences, including a special Trimester at IHP in 2004 and
an influential recurrent workshop on ``Explicit methods'' in Oberwolfach
(every two years since 2007) and 5 \PariGP Ateliers. He has (co-)supervised
11 PhD students and about 15 masters students.

Karim is a leading computational number theorist in France. He
is the project leader for the \PariGP free computer algebra system since
1995, which has had a major impact on hundreds of publications. He is one of
the system's main developers (about 60000 lines of code written, most of the
documentation, and 1300 bug-tracking tickets authored).
\end{participant}
%%% Local Variables:
%%% mode: latex
%%% TeX-master: "../proposal"
%%% End:

\begin{participant}[type=PI,PM=6,gender=male]{Bill Allombert}
is a research engineer in Bordeaux. He has a great expertise in
algorithmic number theory and is one of the main developer of \PariGP.
He is the developer of the \software{GP2C} compiler to convert \software{GP} code into
\software{pari} code, and the \software{GALPOL} database (database of polynomial with
prescribed Galois groups).

He also has 5 articles in international journals and is 
a regular contributor of the software \GAP and the \software{Debian} project.
\end{participant}

%%% Local Variables:
%%% mode: latex
%%% TeX-master: "../proposal"
%%% End:

\begin{participant}[type=R,PM=6,gender=male]{Adrien Boussicault}
  Maître de Conférences at the LaBRI (Laboratoire Bordealais de Recherche en 
  informatique), Adrien Boussicault is a young researcher in Algebraic and 
  Enumerative Combinatorics. He has 8 papers in international journals. 
  His contributions to \Sage include writing 3 tickets to implement 
  combinatorial objects and co-organising \SageCombinat Days 57.
\end{participant}
%%% Local Variables:
%%% mode: latex
%%% TeX-master: "../proposal"
%%% End:


\begin{participant}[type=R, PM=48]{NN}
We will hire a computational support research engineer to work at Bordeaux
 under the leadership of Professor Karim Belabas and Doctor Vincent
Delecroix on the tasks of \WPref{hpc}, \WPref{component-architecture} and \WPref{UI}.
He or she will work on some or all of the following tasks
\begin{itemize}
\item parallelization of some algorithms in \PariGP,
\item the creation of a Python library for \PariGP and its usage within \Sage,
\item and the graphics capabilities inside Sage.
\item \TOWRITE{VD}{OTHERS??}
\end{itemize}
\end{participant}

\begin{participant}[type=R,PM=6,gender=male,salary=5600]{Lo\"ic Gouarin}
Lo\"ic Gouarin is a CNRS employee but we refer to the Paris-Sud Universit\'e
description ({sec:ParisSud}) for his CV.
\end{participant}

\subsubsection*{Publications, products, achievements}

Some recent publications :
\begin{compactenum}
\item 
Karim Belabas, Eduardo Friedman,
\textit{Computing the residue of the Dedekind zeta function}.
Math. Comp. 84 (2015), no. 291, 357--369. 

\item
The PARI Group; \PariGP version 2.7.0, Bordeaux, 2014,
\url{http://pari.math.u-bordeaux.fr/}.

\item
Karim Belabas et al.
\textit{Explicit methods in number theory. Rational points and Diophantine equations},
179 pages, Panoramas et Synthèses 36, 179p., 2012.

\item
Bill Allombert, Yuri Bilu and Amalia Pizarro-Madariaga,
\textit{CM-Points on Straight Lines}, to appear in ``Analytic Number Theory'' (dedicated do H. Maier),
Springer.

\item
Vincent Delecroix,
\textit{Cardinality of Rauzy classes}
Ann. Inst. Fourier, 63 no 5 (2013), p. 1651-1715.

\item
Jean-Christophe Aval, Adrien Boussicault, Mathilde Bouvel, Matteo Silimbani
\textit{Combinatorics of non-ambiguous trees},
Advances in Applied Mathematics 56 (2014), p. 78-108.
\end{compactenum}

\subsubsection*{Previous projects or activities}

Current grants:
\begin{compactenum}
\item
 ANR PEACE (2012-2015).
    Goal: The discrete logarithm problem on algebraic curves is one of the rare
    contact points between deep theoretical questions in arithmetic geometry and
    every day applications. On the one side it involves a better understanding,
    from an effective point of view, of moduli space of curves, of abelian
    varieties, the maps that link these spaces and the objects they classify.
    On the other side, new and efficient algorithms to compute the discrete
    logarithm problem would have dramatic consequences on the security and
    efficiency of already deployed cryptographic devices. 

\item
ERC starting grant ANTICS (2011-2016).
    Goal: "Rebuild algorithmic number theory on the firm grounds of theoretical
    computer science".
    Challenges: complexity (how fast can an algorithm be?), reliability
    (how correct should an algorithm be?), parallelisation.
\end{compactenum}

\subsubsection*{Significant infrastructure}
\begin{compactenum}
\item The Plafrim is a regional federation hosted at INRIA Bordeaux (in partnership with the CNRS-LaBRI and CNRS-IMB). It has an important cluster devoted to experimental code (1188 cores).
\item The M\'esocenter de Calcul Intensif Aquitain (MCIA) is localised
in Bordeaux. It hosts the Avakas cluster (3328 cores,  38 TFlops) and the
M3PEC cluster (432 cores).
\end{compactenum}

\end{sitedescription}

%KEY-MORE-TODOS


%%% Local Variables:
%%% mode: latex
%%% TeX-master: "../proposal"
%%% End:

%  LocalWords:  sitedescription th eorie des nombres Plafrim mesocentre Avakas developped
%  LocalWords:  subsubsection Belabas Synthèses Allombert Bilu Pizarro-Madariaga Maier
%  LocalWords:  parallelisation TOWRITE
