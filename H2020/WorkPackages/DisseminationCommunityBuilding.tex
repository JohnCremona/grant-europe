\TOWRITE{HF}{Proofread WP 2 Dissemination pass 1}
\TOWRITE{ALL}{Proofread WP 2 Dissemination pass 2}

% This work package must start in month 1. Not sure what the
% appropriate load factor should be, and why that is important?
\begin{workpackage}[id=dissem,wphases=0-48!.5,
  short={Community Building/Dissemination},
  title={Community Building, Training, Dissemination, Exploitation, and Outreach},
  lead=PS,
  PSRM=10,
  SARM=18,
  USORM=10,
  USHRM=17,
  USRM=30,
  UVRM=2,
  UKRM=2,
  UBRM=20,
  SRRM=2,
  LLRM=6
]


\begin{wpobjectives}
  The objective of this work package is to further develop the community at the
  European scale, foster cross team collaborations, spread the
  expertise, and engage the greater community to participate in the
  definition and refinement of the requirements, and the implementation and use of the
  produced solutions. This includes:
  \begin{compactitem}
  \item ensuring awareness of the results in the user community;
  \item engaging cross communities discussions to foster scientific collaborations and conjoint developments;
  \item spreading the expertise through workshops and trainings;
  \item providing training for partners inside the project, the
    community engaging with contributions to the project, and
    end-users of \TheProject
  \item reviewing emerging technologies,
  \item develop demonstrators,
  \item defining individual exploitation plans; and,
  \item managing existing and new intellectual property.
  \end{compactitem}
\end{wpobjectives}

\begin{wpdescription}
  We will organize regular open workshops (e.g. Sage Days, Pari Days,
  summer schools, etc.); some of them will be focused on development
  and coding sprints, and others on training. This is also an occasion
  to organize cross communities workshops like Sage-Jupyter days.

  Some of the networking activities and internal training will come
  from short to long term visits between the participants, to
  collaborate on specific features. A typical such visit would bring
  together an \Jupyter developer with a GAP developer for a couple of
  days to implement a first prototype of a notebook interface to GAP.

  A number of demonstrators will be developed in the project and
  disseminated in different ways.

  All the code, documents, test and build infrastructure produced as
  part of the project will be made available as open source.

  This work package will complement and lean on a parallel COST
  network whose role is to build and engage the greater community.
\end{wpdescription}

\begin{tasklist}

\begin{task}[title=Dissemination and Communication activities, lead=PS, partners={SA}, id=dissemination-communication, PM=11, wphases=0-48 ]

  This task comprises all forms of direct dissemination and public
  communication activities such as press releases, creation of the
  project web-site (\delivref{management}{tickets}) including visitor analysis and monitoring tools,
  scientific and technical publications, outreach activities
  (seminars, keynote talks, media interviews, press releases),
  pro-motion through social media (e.g. Twitter, Facebook, LinkedIn),
  creation of advertisement materials such as flyers, posters, and
  electronic feeds as well as their distribution. We will use standard
  community building technology such as mailing lists, Wikis and
  Forums, to support dissemination and engagement of the community to
  support this. We will also generate press releases at opportune
  moments (\delivref{dissem}{press-release-1}, \delivref{dissem}{press-release-2}). %, making use of the public relation support services in the respective institutions.
  %At least two press releases will be
  %generated in the course of the project.


  % News articles will be produced by experienced professional staff
  % at relevant partners including ... and communicated to local,
  % national and international media, as appropriate.
\end{task}

\begin{task}[title=Training and training portal,
id=training-portal,lead=PS,PM=1,wphases=0-1]
Training is at the heart of this project: through our open source
approach, networking activities, workshops, demonstrators
(\localtaskref{index-librorum-salvificorum}), interactive books (such
as in \taskref{UI}{oommf-tutorial-and-documentation} and
\localtaskref{ibook}), and training for teachers and trainers
(\localtaskref{project-intro}), we have firmly integrated the training
aspect into the core of our project plan.

Each of these activities will create dedicated webpages hosting the
material to make it accessible to a public as wide as possible ---
inline with our philosophy for software, we believe in the benefits of
sharing the material and maximising the value of the financial
investment into this project.

In this task, we create a central \TheProject training portal that
serves as an inclusive point of entry to explore available training
materials. This will be hosted on the projects website (\localtaskref{dissemination-communication}).
\end{task}

\begin{task}[title=Community Building: Development Workshops, lead=PS,PM=24, partners={UB,UK,SR,SA,USH}, id=devel-workshops, wphases=0-48]

  We will organize development workshops all throughout the
  project. The aim of these workshops is to bring together developers
  from the different communities to design and implement some key
  aspects of \TheProject such as user interface, and documentation and
  to ensure cross compatibility. These meetings will gather not only
  participants of \TheProject but also members of the different
  communities involved. Bringing talented people together is the best
  way to make actual progress on the different aspects of \TheProject
  and to kick-start new challenges. It is also a way to work within
  the communities we're reaching in and include them in the discussions
  and development. It fosters collaborations between scientists and
  developers from different backgrounds to build tools that are needed
  by all.

  Each workshop will be aimed at a specific software component (\Sage,
  \GAP, \SMC, \IPython, \Singular, etc.) or be joint meeting between
  different communities to improve interoperability and joint
  developments. We are planning to have 4 or 5 such events per
  year. The currently anticipated list of workshops includes:

\begin{compactitem}
\item One \Sage worshop per year in Cernay France (near Orsay) where
  similar gatherings took place before. One of them will be a
  Sage-Sphinx day, dedicated to documentation.

\item One Atelier \Pari in Bordeaux per year. The team in Bordeaux has
  a great experience in organizing this kind of \Pari events.

\item Two \Singular workshops and two \GAP-\Singular workshops in Kaiserslautern
  over the four years.

\item Two workshops dedicated to high performance mathematical
  computing in relations with \WPref{hpc}. One of them should be in
  Grenoble and the second one in Bordeaux to foster the work with
  \Pari towards \taskref{hpc}{hpc-pari}.

\item Two Data Science workshops which use \TheProject to develop effective machine learning and data modelling practice organised by Sheffield.

\item A joint meeting on the topics of \SMC and \Jupyter in Simula in
  relations with \WPref{UI}.

\item A joint event between \GAP, \Sage, and \Singular in ICMS,
  Edinburgh.

\item A joint \Jupyter and \Sage event in Orsay.

\item A joint \LMFDB and \Sage event in Warwick to work towards
  \WPref{dksbases}.

\end{compactitem}

Yearly reports will be delivered on the impact of these workshops on the community (\delivref{dissem}{workshops-1}, \delivref{dissem}{workshops-2}, \delivref{dissem}{workshops-3}, \delivref{dissem}{workshops-4}).

\end{task}


\begin{task}[title=Reviewing emerging technologies, id=tech-review, lead=PS, partners={SA,USO,USH,US,UV,UB,SR},PM=11, wphases=0-48]
  In this task, we will produce periodic reviews (\delivref{dissem}{techno}) of emerging
  technologies and relevant developments elsewhere, and implications
  for our plans, taking into account input from the communities. This
  include the review of standard components and service for storage
  and sharing, computational resources, authentication, package
  management, etc. This may further include negotiating access or
  shared development when appropriate. This information will be fed to
  the other work packages, in particular Work
  Package~\WPref{component-architecture} Component Architecture.
\end{task}


\begin{task}[title=Dissemination: reaching towards users and fostering diversity, lead=PS,PM=12, partners={UB,USH,SA}, id=dissemination, wphases=0-48]

  As lead developers of \TheProject, most of us consider themselves as
  both scientists and developers. We have experience in reducing the
  gap between those two worlds. We organize training, workshops such
  as Sage-days to promote our tools and bring more users and
  developers from the scientific world. On the other hand, we often
  attend and present to more development-oriented gatherings like
  PyCon and SciPy to exchange with engineers and foster
  collaborations.

  The aim of this task is to exploit this winning strategy within
  \TheProject. Three events should be organized in the spirit of
  Sage-days to gather and train more users and foster scientific
  development around \TheProject. These conferences usually welcome
  around 50 participants and have a big impact on the scientific
  community. One of them will be at CIRM in Marseille, another one at
  ICMS in Edinburgh and a third one probably in Dagstuhl Germany. In
  the same spirit, we will also have training sessions organised
  within the universities (Orsay and Grenoble). We will also run a
  series of 4 workshops in developing countries especially Africa and
  South America. Some of these workshops will be joined to CIMPA
  schools.  The CIMPA is an international organization based in Nice
  (France) that promote research in mathematics in developing
  countries. It organizes each year around 20 schools.

  The under-representation of women in the scientific world is even
  more perceptible when we intersect science with software
  development. As we know we have many talented women in our
  community, and we will organize some events targeted at women in the
  spirit of the "Women in Sage" days that happened many times in the
  US already. We are planning to have two of them in Orsay and at least
  one in Oxford where "Women in CS" days already took place.

  Apart from these different events, we will also be present at
  important events of both our scientific community (international
  mathematical conferences such as FPSAC for combinatorics) and the
  python / open-source software development community: PyCon, SciPy,
  EuroPython, etc. The material we develop for presentation at these
  events will be made publicly available.
\end{task}


%Mike Croucher and Neil Lawrence,Sheffield
\begin{task}[title=Introduce \TheProject to Researchers and Teachers, id=project-intro,lead=USH,PM=20,partners={USO}]

  In this task, we will develop and deliver materials that will
  introduce \TheProject to potential users---both researchers and
  teachers. Our initial focus will be on teachers, but as the results
  from \WPref{social} become available we will deploy them with
  researchers, both local to the University of Sheffield and across
  the wider computational biology and machine learning fields.

  We will develop a `taster' seminar (1-2 hours) and follow-up short course
  (1-2 days) on \TheProject for researchers and lecturers in all
  disciplines \delivref{dissem}{short-course}. At Sheffield, this will
  be added to the set of courses that are offered as part of IT
  Services' research support department. As such, it could potentially
  reach all disciplines. It will also be made publicly available for
  widespread dissemination and collaborative modification.

  Elements of this work will also be integrated with the GP Summer
  Schools and Roadshows (\url{http://ml.dcs.shef.ac.uk/gpss/}). The
  Summer School is now in its fourth edition (over 140 students
  educated). The Roadshows have taken place in Uganda, Colombia and in
  2015 they are scheduled for Italy, Australia and Kenya. The Kenya
  school will be the first to have more of a `data science' focus that
  we think will be particularly appropriate for dissemination of
  \TheProject (\delivref{dissem}{datascience-course}).

  These seminars and short-courses will also be used to identify
  potential collaborators who are interested in utilising \TheProject
  immediately. We will act as consultants to these collaborators in
  two ways:

  We will work with lecturers at Sheffield to introduce \TheProject to
  various disciplines via the production of interactive lecture notes
  (\delivref{dissem}{lecture-notes}). The focus for the student here
  will not necessarily be on programming but rather on interaction
  with the subject matter via use of \TheProject. Interactive lecture
  notes are an area where commercial vendors such as MapleSoft and
  Wolfram Research are spending a lot of time and money developing
  material. We will provide technical and programming expertise to
  lecturers---helping them to develop the interactive part of notes
  while they provide the subject material.

  We will work as consultants with researchers at Sheffield to
  introduce \TheProject to their workflow. Any projects that
  successfully do this will be promoted as case studies for
  \TheProject.

  Finally, our aim will be to introduce ideas from
  \WPref{social-aspects} in our teaching materials. By the end of the
  project we will have produced a series of interactive notebook
  demonstrators \localdelivref{notebook-repo} of \TheProject with a
  particular focus on computational biology, data science and machine
  learning. These notebooks will expand the use of VREs in these
  domains, appealing to researchers used to the domains of
  Bioconductor and MATLAB. We will make use of live notebook posters
  (\delivref{social}{social-poster}) and commenting systems
  (\delivref{social}{jupyter-comment}). These interactive notebooks
  will be provided in a public repository
  (\delivref{dissem}{notebook-repo}).
\end{task}

\begin{task}[id=dissemination-of-oommf-nb-virtual-environment,
  title=Open source dissemination of micromagnetic VRE,
  lead=USO,PM=4,partners={SR,USH,PS},wphases=24-28]
  % 3 months person time + 1 months investigator time
  Tasks \taskref{UI}{oommf-py-ipython-attributes} and
  \taskref{UI}{oommf-tutorial-and-documentation} provide the
  micromagnetic VRE demonstrator
  (\ref{sec:introduction-micromagnetic-vre-demonstrator}) built on top
  of \TheProject.  In this task, we set up of the infrastructure
  (\delivref{dissem}{oommfnb-source-and-testing-setup}) to encourage
  and invite code contributions from the micromagnetic community to
  both code and created notebooks, while automating quality control
  and maintaining trust effectively.

  The source code of the micromagnetic VRE will be made available as
  open source on public repository hosting sites (such as
  GitHub/Bitbucket), and announced to the community via appropriate
  mailing lists and other means. We will set up a publicly accessible
  Jenkins/Travis continuous integration (CI) system to (i) run
  regression tests (from
  \taskref{component-architecture}{oommf-python-interface} and
  \taskref{UI}{oommf-py-ipython-attributes}) routinely when the
  micromagnetic VRE code or underlying OOMMF core code changes, (ii)
  re-execute notebooks (from
  \taskref{UI}{oommf-tutorial-and-documentation}) and use them as
  regression tests (using the outcome of task
  \taskref{UI}{notebook-verification}), and (iii) re-build
  downloadable installation files and virtual machine images.
  %This set
  %up will test user-contributions automatically.

  %versions (\delivref{dissem}{oommfnb-source-and-testing-setup}).

\end{task}

\begin{task}[title=Micromagnetic VRE dissemination workshops,
id=dissemination-of-oommf-nb-workshops,lead=USO,PM=6,wphases=14-40!0.23] % funny ratio here, that's okay (HF)

  % 3 months person time, 2 months investigator time

We will run a series of workshops
(\delivref{dissem}{oommfnb-workshops}) during the evenings of 4 major
international meetings on magnetism research\footnote{Anticipated most
  significant international meetings in the appropriate time frame are
  61st Conference on Magnetism and Magnetic Materials (MMM2016), October
  31-November 4, 2016, New Orleans, Louisiana; 62nd Conference on
  Magnetism and Magnetic Materials (MMM 2017), November 6-10, 2017, Pittsburgh,
  Pennsylvania; 21st International Conference on Magnetism (ICM 2018),
  July 16–20, 2018, San Francisco, California; 14th Joint MMM-Intermag
  Conference (MMM2019), January 14-18, 2019, Washington, DC). Each of those
  meetings is one week long, and serves as a focal point of networking
  for the european and international research community. Other training events have been held in the
past at these conferences and were well attended.} to
disseminate the micromagnetic virtual research environment
(Sect. \ref{sec:introduction-micromagnetic-vre-demonstrator} and
\localtaskref{dissemination-of-oommf-nb-virtual-environment}) in the
micromagnetic community. Each conference
attracts around 1500 participants, and we expect at least 30 for our
workshops at every event. Depending on demand, multiple workshops
will be given per conference.

The taught material will include (i) use of the \Jupyter-based
micromagnetic VRE, and an (ii) introduction to the standard techniques
for contributing to open source software (version control, pull
requests, testing frameworks) to foster excellence in computational
science and to make the micromagnetic VRE project self-sustaining as quickly
as possible. In addition, all teaching materials, including videos,
will be made available on a website.

For each workshop, we request \euro{500} room hire at the magnetism conference
location and the travel expenses for two teachers from Southampton to
attend the one week international conference, totalling (\euro{500} +
2x\euro{2200}=\euro{4900}) per workshop. There are no other costs.
\end{task}

\begin{task}[title=Demonstrator: Interactive books,
id=ibook,lead=US,partners={USO},PM=36,wphases=0-36,40-46]
  % 2x12 _ 3x 3 months for students
  % 6 months Southampton, ibook4: maths for engineering, in months 40-46

One of the important elements of VREs is a common flexible writing format which
enables the creation of large structured documents. There are many
known solutions to that problem, but they usually compromise the
interactivity of the notebook interface and typesetting quality of desktop
publishing software like LaTeX.

Recently, a few approaches tried to bring both interactivity and the
typographic features. The modestly tagged markup language
\href{http://hplgit.github.io/doconce/doc/web/}{DocOnce}
targets the problem of reusability of the document source code for
producing traditional LaTeX-based printed books, IPython notebooks, Sphinx
documents (with Sage cells), and many other formats. MathBook XML
is a lightweight XML application for authors of scientific articles,
textbooks and monographs extensively using Sage cells for
interactive elements. The Sphinx documentation software has been
successfully applied for creation of interactive books containing Sage
cells. Additional interactivity is offered using the \href{http://runestoneinteractive.org}{Runestone tools}.

The technical aspects of format for interactive publications is a
subject of the task ``Structured documents'' in
\taskref{UI}{structdocs}. In this task we will demonstrate usability
of the results of \taskref{UI}{structdocs} in creation of scientific
textbooks. Three interactive books will be created:

\begin{compactitem}
\item Nonlinear Processes in Biology (\delivref{dissem}{ibook1})
\item Linear Algebra (\delivref{dissem}{ibook2})
\item Computational Mathematics for Engineering (\delivref{dissem}{ibook4})
\item Problems in Physics with Sage/Python (\delivref{dissem}{ibook3a}, \delivref{dissem}{ibook3b}, \delivref{dissem}{ibook3c})
\end{compactitem}

The choice of those particular topics has been made for the sake of
maximal diversity. The ``Nonlinear Processes in Biology'' will heavily
use numerical solution of ODEs and PDEs. The Linear Algebra book will
be a classical mathematical textbook natively desingned within
VRE. The last example will focus mostly on collaborative editing and
modularity of content which is produced using VRE technologies. We
will demonstrate the power of symbolic algebra where appropriate throughout.

The main research aspect for this task will be to integrate modern
computational tools in classical scientific topics and explore how
the VRE environment can accelerate the development and produce electronic
documents with significantly enhanced pedagogy.

In particular we will answer following questions:
\begin{compactitem}
\item When is a fully interactive worksheet required and when is
  a textbook with executable code cells sufficient?
\item How to assemble a classical monograph by reusing independently working
  building block of text and code?
\item What are best tools and practices for using a single source for
  producing printed and electronic (interactive) textbooks?
\item How to collaboratively write reusable course material?
\item How can we facilitate automatic testing of all code examples, plots, etc?
\item How can students can benefit from using VRE?
\end{compactitem}


\end{task}

\begin{task}[title=Demonstrator: Computational mathematics resources indexing service,
id=index-librorum-salvificorum,lead=UV,PM=2,partners={UB}] Beyond official documentation and
  tutorials, users of mathematical software and VREs learn from a wide
  array of sources: university courses, Q\&\ A sites, web searches,
  etc.  A simple web search on any major software component yields
  dozens of non-official tutorials and how-tos in many different
  languages. However, search engines mostly miss the relevant
  metadata: how does one find a tutorial on linear algebra in \PariGP,
  written at an undegraduate level, in French or Spanish?

This need has been felt by most communities at some point, and each
has come up with its own solution: most software components (e.g.,
\GAP, \PariGP, \Sage, \dots) simply link material from their official
page; \Sage has a wiki (\url{http://wiki.sagemath.org/}) referencing
additional resources, and used to host a large number of tutorial
worksheets on \url{http://sagenb.org/}; the recent introduction of
public projects in \SMC is sparking approximately the same phenomenon
that had previously happened with \url{http://sagenb.org/}; \IPython
host the Notebook Viewer service (\url{http://nbviewer.ipython.org/}),
which renders (without hosting) community-made notebooks; and teaching
institutions host or link their own collections of pedagogical
resources.

These collections are usually incomplete, limited in scope, hard to
search, and difficult to keep up-to-date.  What the community needs is a
community-curated, searchable, metadata-driven, multilingual, platform
agnostic indexing service whose goal is to reference and rank all the
community generated knowledge around a software component or VRE.

The goal of this task is to create the tool
(\delivref{dissem}{ils-tool}) powering such service, and to host a
(free) community-curated index for \TheProject related resources as a
demonstrator (\delivref{dissem}{ils-service}).

\end{task}




\end{tasklist}



%Raw material:
%\begin{compactitem}
%\item Documentation improvements: overview, cross links, overview of
%  recent improvements
%\item Thematic tutorials
%\item Collections of pedagogical documents\\
%  E.g. a complete collection of interactive class notes with computer
%  lab projects for the ``Algèbre et Calcul formel'' option of the
%  French math aggregation (starting from 2014-2015, only open-source
%  systems will be supported, and Sage is a major player).
%  % See http://nicolas.thiery.name/Enseignement/Agregation/ as a starter
%  % Math labs with Sage for first year students in France (L1): http://math.univ-lyon1.fr/~omarguin/
%\item Localization of the Sage user interface and key documents in
%  various European languages.
%\item Distribution of the documents either in the main distribution of
%  Sage or through the online repository (see collaborative tools).
%\item Massive online introduction course to Sage, drawing on the sage tutorial/notebooks.
%Could be ``First year Sage course in a box''.
%\item Taking the opportunity of Python courses to propose Sage as a natural extension
%for mathematics; an example is French's
%% The url macro eats the accented letters.
%% It doesn't just eat it, it pukes it back!
%``Classes pr\'eparatoires''
%%\footnote{\url{http://en.wikipedia.org/wiki/Classe_préparatoire_aux_grandes_écoles}},
%where Python has been recently selected as the language to learn programming\footnote{See
%the ``Annexe'' at
%\url{http://www.education.gouv.fr/pid25535/bulletin_officiel.html?cid_bo=71586}}.
%%\item \TODO{please expand!}
%\end{compactitem}

% Jeroen: About teaching: in Gent, Sage is already integrated in the
% courses (maybe you can add this, don't know if it's relevant)
% starting in the first year. It's good for the students because it
% helps in 2 ways: it helps them to understand the mathematics better
% and it helps them to learn basic down-to-earth programming (they
% also have a programming course in Java but that contains a lot of
% theory about complicated class structures)
% Same thing in Orsay
% More python centered but same in UZH
% We have also Sage @ Silesia from 1st semester (physics)

\begin{wpdelivs}
\begin{wpdeliv}[due=6,id=press-release-1,dissem=PU,nature=DEC,lead=PS]{Starting press release}\end{wpdeliv}
 \begin{wpdeliv}[due=12,id=workshops-1,dissem=PU,nature=R,lead=PS]{Community building: Impact of development workshops, year 1}\end{wpdeliv}
  \begin{wpdeliv}[due=12,id=ibook3a,dissem=PU,nature=DEM,lead=US]{Demonstrator: Problems in Physics with Sage v1} \end{wpdeliv}
  \begin{wpdeliv}[due=12,id=techno,dissem=PU,nature=R,lead=PS]{Review on emerging technologies} \end{wpdeliv}
   \begin{wpdeliv}[due=18,id=short-course,dissem=PU,nature=DEC,lead=USH]{Short Course: A short course for lecturers on using \TheProject for delivering mathematical education.}\end{wpdeliv}
\begin{wpdeliv}[due=24,id=workshops-2,dissem=PU,nature=R,lead=PS]{Community building: Impact of development workshops, year 2}\end{wpdeliv}
\begin{wpdeliv}[due=24,id=ils-tool,dissem=PU,nature=P,lead=UV]{Community-curated
     indexing tool (open source)} \end{wpdeliv}
      \begin{wpdeliv}[due=24,id=ils-service,dissem=PU,nature=DEM,lead=UV]{Community-curated
     indexing service for \TheProject} \end{wpdeliv}
      \begin{wpdeliv}[due=24,id=datascience-course,dissem=PU,nature=DEC,lead=USH]{Course material on using \TheProject in data science} \end{wpdeliv}
  \begin{wpdeliv}[due=24,id=dissem-1,dissem=PU,nature=R,lead=PS]{Impact of dissemination and training activities for years 1 and 2}\end{wpdeliv}
   \begin{wpdeliv}[due=30,id=ibook3b,dissem=PU,nature=DEM,lead=US]{Demonstrator: Problems in Physics with Sage v2} \end{wpdeliv}
   \begin{wpdeliv}[due=32,id=oommfnb-source-and-testing-setup,dissem=PU,nature=DEC,lead=USO]{Micromagnetic
     VRE code and documents source online} \end{wpdeliv}
 \begin{wpdeliv}[due=36,id=ibook1,dissem=PU,nature=DEM,lead=US]{Demonstrator: Nonlinear Processes in Biology  interactive book} \end{wpdeliv}
 \begin{wpdeliv}[due=36,id=workshops-3,dissem=PU,nature=R,lead=PS]{Community building: Impact of development workshops, year 3}\end{wpdeliv}
  \begin{wpdeliv}[due=36,id=lecture-notes,dissem=PU,nature=DEM,lead=USH]{Demonstrator: Interactive lecture notes and marking systems based on \TheProject.}\end{wpdeliv}
 \begin{wpdeliv}[due=30,id=ibook2,dissem=PU,nature=DEM,lead=US]{Demonstrator: Linear Algebra -  interactive book} \end{wpdeliv}
 \begin{wpdeliv}[due=44,id=ibook3c,dissem=PU,nature=DEM,lead=US]{Demonstrator: Problems in Physics with Sage v3} \end{wpdeliv}
 \begin{wpdeliv}[due=44,id=oommfnb-workshops,dissem=PU,nature=OTHER,lead=USO]{Micromagnetic
     VRE workshops delivered} \end{wpdeliv}
 \begin{wpdeliv}[due=24,id=ils-tool,dissem=PU,nature=P,lead=UV]{Community-curated
     indexing tool (open source)} \end{wpdeliv}
 \begin{wpdeliv}[due=24,id=ils-service,dissem=PU,nature=DEM,lead=UV]{Community-curated
     indexing service for \TheProject} \end{wpdeliv}
 \begin{wpdeliv}[due=18,id=short-course,dissem=PU,nature=DEC,lead=USH]{Short Course: A short course for lecturers on using \TheProject for delivering mathematical education.}\end{wpdeliv}
 \begin{wpdeliv}[due=24,id=datascience-course,dissem=PU,nature=DEC,lead=USH]{Course material on using \TheProject in data science.}
   \end{wpdeliv}
 \begin{wpdeliv}[due=36,id=lecture-notes,dissem=PU,nature=DEM,lead=USH]{Demonstrator:
     Interactive lecture notes and marking systems based on
     \TheProject.}\end{wpdeliv}
  \begin{wpdeliv}[due=47,id=ibook4,dissem=PU,nature=DEM,lead=USO]{Demonstrator:
      Computational Mathematics for Engineering} \end{wpdeliv}
 \begin{wpdeliv}[due=44,id=notebook-repo,dissem=PU,nature=DEM,lead=USH]{Demonstrator: Repository of interactive Notebooks in Machine Learning and Computational Biology based on \TheProject.}\end{wpdeliv}
 \begin{wpdeliv}[due=48,id=workshops-4,dissem=PU,nature=R,lead=PS]{Community building: Impact of development workshops, year 4}\end{wpdeliv}
 \begin{wpdeliv}[due=48,id=dissem-2,dissem=PU,nature=R,lead=PS]{Impact of dissemination and training activities for years 3 and 4}\end{wpdeliv}
\begin{wpdeliv}[due=48,id=press-release-2,dissem=PU,nature=DEC,lead=PS]{Ending press release}\end{wpdeliv}
\end{wpdelivs}


\end{workpackage}

%%% Local Variables:
%%% mode: latex
%%% TeX-master: "../proposal"
%%% End:

%  LocalWords:  workpackage dissem wphases wpobjectives wpdescription tasklist WPref nmag
%  LocalWords:  delivref linkedin organisation finalpressrelease organise wpdelivs github
%  LocalWords:  wpdeliv dissemination-of-oommf-nb-virtual-environment OOMMFNB taskref
%  LocalWords:  oommf-python-interface oommf-tutorial-and-documentation mumag magpar
%  LocalWords:  mumax micromagnum micromagnetic oommf-py-ipython-attributes summarising
%  LocalWords:  dissemination-of-oommf-nb-workshops localtaskref MMM-Intermag Fangohr
%  LocalWords:  maximise sagecell structdocs Algèbre Calcul formel eparatoires Annexe
%  LocalWords:  Jeroen
