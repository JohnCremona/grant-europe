\begin{draft}
\TOWRITE{PS (Work Package Lead)}{For WP leaders, please check the following (remove items
once completed)}
\begin{verbatim}
- [ ] have all the tasks in this Work Package a lead institution?
- [ ] have all deliverables in the WP a lead institution?
- [ ] do all tasks list all sites involved in them? 
- [ ] does the table of sites and their PM efforts match lists of sites for each task?
      (each site from the table is listed in all relevant tasks, and no site is listed
      only in the table or only at some task)
\end{verbatim}
\end{draft}



\begin{workpackage}[id=management,type=MGT,wphases=0-48!.2,swsites,
  title=Project Management,short=Management,
  lead=PS,
  PSRM=28,SARM=2,  
  USORM=2,LLRM=2,UVRM=2,UJFRM=2,UBRM=2,UORM=2, USHRM=2,
  UWRM=2, JURM=2, UKRM=2, USRM=2, ZHRM=1, SRRM=2, UWSRM=2]

\begin{wpobjectives}
%  The objectives of this work package are to undertake all project management activities,
%  including:
%  \begin{compactitem}
%  \item monitoring the overall progress of the project and the use of
%    resources;
%  \item ensuring the timely production of deliverables and other
%    project outputs;
%  \item reporting to the European Commission on financial matters;
%  \item preparing for and attending the annual project review
%    meetings; and
%  \item managing the project Advisory Board.
%  \end{compactitem}

Establish and maintain an effective contract, project, and operational management
approach, ensuring (i) an effective and timely implementation of the project, (ii) quality control
of the results, (iii) risk and innovation management of the project as a whole, as well as (iv)
timely and necessary interaction with the EC and other interested parties.

  % The objective of  is to undertake all project management
  % activities, including setting up joint infrastructure, organizing
  % meetings, and producing overview reports.
\end{wpobjectives}

\begin{wpdescription}
The project will be managed by UPS, which has profound experience in administrating and leading EU funded and national projects. The coordinator together with the WP leaders, will be responsible for monitoring WP status, coordination of work plan updates and annual internal progress reports. The project management structure  and roles of partners in the consortium are presented in the next section.
\end{wpdescription}

\begin{tasklist}
\begin{task}[title=Project and financial management,
id=project-finance-management,lead=PS,PM=33,
partners={LL,UV,UJF,UB,UO,USH,USO,SA,UW,JU,UK,US,ZH,SR,UWS}]
The task includes the following activities
  \begin{compactitem}
  \item Preparation and Distribution of the
    Consortium Agreement;
  \item Setup project website, intranet and
    communication procedures for effective communication;
  \item
    Organization of project review and progress meetings;
  \item
    Establishment and maintenance of external contacts (with the EC,
    other relevant national / EU projects, other academic and
    industrial stakeholders) to organize transfer of knowledge,
    present and
promote project results;
  \item Progress and Financial Reporting to the EC;
  \item Data and IPR Management will be managed in accordance with agreed rules stated in the
Consortium Agreement and in accordance with the Data management plan (Task X.Y).\TOWRITE{NT}{Fix the reference to the task}
  \end{compactitem}
\end{task}

\begin{task}[title=Quality assurance and risk management,id=project-quality-management,lead=PS,PM=15,partners={LL,UV,UJF,UB,UO,USH,USO,SA,UW,JU,UK,US,ZH,SR,UWS}]
A quality assurance plan will be established to ensure coherent and sufficient quality of the work
and results. The plan will be developed by UPS, involving all partners, and will be followed up
regularly. In addition, the project coordinator with support from the coordination team and quality review board will establish and review annually a risk management plan and self-assessment to ensure that technical barriers / potential risks are identified  and corrective measures are put into place on time (\delivref{management}{ipr}).
\end{task}

\begin{task}[title=Innovation management,
id=project-innovation-management,lead=PS,PM=10,
partners={LL,UV,UJF,UB,UO,USH,USO,SA,UW,JU,UK,US,ZH,SR,UWS}]
One of the most important criteria for success for the OpenDreamKit project is to bring the project results into use. Therefore, exploitation routes will be sought whenever possible. In
order to create a common understanding within the Consortium about how we can best usher
an idea all the way from conception to its realization and exploitation, the Coordinator will be responsible for
the preparation and realization of an Innovation Plan to assure that research activities meet the
required milestones and the outputs of innovation are fully aligned with the project objectives.
All research activities will go through three initial steps where the exploitation opportunity is
identified along with the main stakeholders for the exploitation opportunity and an IP owner.
\end{task}
\end{tasklist}

%
%  This workpackage will perform all the activities related to monitoring of progress
%  towards the project milestones shown on Page~\pageref{sec:milestones} and the
%  deliverables listed on Page~\pageref{sec:deliverables}, assuring the quality of the
%  deliverables, ensuring the collation and distribution of the required reports,
%  questionnaires and deliverables including the annual reports to the European Commission,
%  arranging project management meetings, tracking the project budget in terms of
%  expenditure and person-months, obtaining financial certificates as required, convening
%  project management meetings, ensuring that important project documents such as the
%  project contract and the consortium agreement are properly maintained and amended as
%  necessary, ensuring that contractual details are complied with, monitoring compliance
%  with the grant agreement, preparing for the annual review meetings, and reviewing
%  research results against the aims and objectives of the project. It also involves
%  managing and supporting the project Advisory Board, including supporting attendance at
%  project meetings, convening Advisory Board meetings, and obtaining feedback on the
%  project direction and results.


\begin{wpdelivs}
\begin{wpdeliv}[due=1,id=ca,dissem=CO,nature=R,lead=PS]{Consortium Agreement}
\end{wpdeliv}

\begin{wpdeliv}[due=1,id=tickets,dissem=PU,nature=DEC,lead=PS]{Establishing basic project infrastructure 
    (websites, wikis, issue trackers, mailing lists, repositories)}
\end{wpdeliv}

\begin{wpdeliv}[due=12,lead=PS,
id=ipr,dissem=CO,nature=R]{Internal Progress Reports, including risk management and quality assurance plan}
\end{wpdeliv}

\begin{wpdeliv}[due=18,lead=PS,
id=tickets,dissem=CO,nature=R]{Innovation Management Plan}
\end{wpdeliv}

\begin{wpdeliv}[due=36,lead=PS,
id=ipr2,dissem=CO,nature=R]{Internal Progress Reports, including risk management and quality assurance plan}
\end{wpdeliv}

\begin{wpdeliv}[due=45,lead=PS,
id=tickets,dissem=CO,nature=R]{Innovation Management Plan}
\end{wpdeliv}



%\begin{wpdeliv}[due=12,id=periodic-rep-1,dissem=PU,nature=OTHER]{First Annual Report (first year)}
% \end{wpdeliv}
%\begin{wpdeliv}[due=24,id=periodic-rep-2,dissem=PU,nature=OTHER]{Project Annual Report (second year)}
% \end{wpdeliv}
%\begin{wpdeliv}[due=36,id=periodic-rep-3,dissem=PU,nature=OTHER]{Project Annual Report (third year)}
% \end{wpdeliv}
%\begin{wpdeliv}[due=48,id=periodic-rep-4,dissem=PU,nature=OTHER]{Project Annual Report (fourth year)}
% \end{wpdeliv}

%Final report is not part of the deliveries.
%\begin{wpdeliv}[due=48,id=final-mgt-rep,dissem=PU,nature=OTHER]{Project Final Report}
% \end{wpdeliv}
\end{wpdelivs}
\end{workpackage}
%%% Local Variables: 
%%% mode: latex
%%% TeX-master: "../proposal"
%%% End: 

%  LocalWords:  workpackage wphases wpobjectives wpdescription pageref wpdelivs wpdeliv
%  LocalWords:  dissem mailinglists swrepository final-mgt-rep compactitem swsites
