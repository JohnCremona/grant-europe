\addtocounter{wpno}{1}
\begin{Workpackage}{\thewpno}
\wplabel{wp:x}
\WPTitle{\wpname{\thewpno}}
\WPStart{Month 1}
\WPParticipant{SA}{1}

\begin{WPObjectives}
  The objective of this work package is to develop and demonstrate a
  set of API's enabling components such as database interfaces,
  computational modules, separate systems such as GAP or Sage to be
  flexibly combined and run smoothly across a range of environments
  (cloud, local, server, ...).
\end{WPObjectives}

\begin{WPDescription}
  This work package includes work on:
  \begin{itemize}
  \item Portability
  \item Interfaces between systems
  \item Deployment and distribution
  \item HPC and Parallelism
  \end{itemize}
\end{WPDescription}

\begin{WPDeliverables}
\begin{itemize}
\item \ref{del:virtual_machines} (Month 12): Creation, deployment, and
  distribution of preconfigured virtual machines for Pari, Sage as a
  cloud service, in particular within the StratusLab infrastructure.
\item \ref{del:sage_cygwin} (Month 12): Fully functional one-click
  install Sage distribution for Windows with Cygwin 32bits.
  % Jean-Pierre: this should take a few months of work

  This 32bits version would work on Windows 64 bits, but more work
  would be required for a native 64 bits version.
  \TODO{Make this a second deliverable?}
  % Comments on this by Bill Hart
  % The big problems you will have on Windows 64 on Cygwin include:
  %
  % * anything with assembly language -- the ABI is different on Windows, so
  % it'll need rewriting, or you can incur a performance penalty by using
  % generic C fallback code
  % * the memory allocator on Windows is not so great
  % * bugs exposed due to being on a different platform, e.g. segfaults due to
  % off-by-one errors that were masked by the granularity of malloc on Linux
  % * build issues, due to identifying Cygwin and using the correct header
  % files, which are often different on Cygwin than linux
  % * issues with PATH vs LD_LIBRARY_PATH
  % * Windows has a case insensitive file system
  % * EOL issues
  % * Windows is not able to rapidly create and delete files, which some
  % libraries (esp. test code) calls for
  % * memory limitations (many people using Windows are using laptops with
  % limited memory, only a portion of which is realistically available to
  % Cygwin)
  % * autotools versions that don't support Windows (usually autotools has a
  % release that is used in all the distributions, which doesn't work correctly
  % on Windows, and this is followed up by a version which has all the Windows
  % patches)
  % * building takes forever on Windows. Mingw2 has now gotten parallel build
  % working on Windows and the speed is within a factor of 5 of Linux. But I'm
  % not sure the improvements have propagated to Cygwin yet.
  % * Cygwin 64 is new, contains quite a few bugs still, and things keep
  % changing with every version as they try to get things right.
  % * Although projects will likely accept patches for Windows, they are less
  % likely to maintain support themselves. I would like to think Singular would
  % be an exception to this. And obviously flint and MPIR work on Windows (even
  % with MSVC as of the next version of flint -- or now if you use our bleeding
  % edge repo version).
\item \ref{del:scscp_sage} Add support in Sage for the SCSCP interface
  protocol.
\item Some IPython/Jupyter deliverables here.
  \TODO{review what it can already do in term of choice of
    computational resource and storage back-end.}
\end{itemize}
\end{WPDeliverables}
Raw material:

Component Architecture
----------------------

Recomputation connection belongs here?

Need to dig into what Jupyter have done

Collaboration with unreliable (or restricted!) networking connections
(peer-to-peer, opportunistic syncing, 3rd world). This is technically
interesting, and gets in support for non-networked working. Not sure
if it belongs here or not.

- Self adaptation to the environment, better schemes for automatically
  selecting appropriate algorithms / components for a given task.

- Modularisation:
  - common architecture for module maintenance and distribution
    (related to point 1 above)
  - sharing experience and best practices
  - modularization of Sage
  - refactorization of GAP's package mechanism; namespaces?

- Portability
  Port to windows (GAP, Sage, Singular)
  Multiplatform test infrastructure

- Security concerns

- Parallelism

- 


Goal: Fostering collaborations/integration between components in an open source ecosystem
=============================================================================

- How to make systems "cooperate" rather than "predate each other".
- E.g. reduce the version issues
- Pushing Python bindings upstream

- How to make it easy to develop simultaneously two interdependent
  components (e.g. Sage+Singular)

- Foster communication

- Social aspect:
  Credit, Citations, Recognition
  Funding

Documentation system
====================

In which package?

Improvements to Sphinx

Sage heavily customizes the Sphinx documentation system, hacking deep
in it in some cases, with quite some duplication in some cases.
Refactor the whole thing, generalizing and contributing back upstream
as much as possible (e.g. parallel compilation).

\end{Workpackage}
