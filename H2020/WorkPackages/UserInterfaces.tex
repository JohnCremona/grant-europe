\TOWRITE{NL}{Proofread WP 4 User Interfaces pass 1}
\TOWRITE{ALL}{Proofread WP 4 User Interfaces pass 2}
\begin{draft}
\TOWRITE{HPL/MRK (Work Package Lead)}{For WP leaders, please check the following (remove items
once completed)}
\begin{verbatim}
- [ ] do all tasks list all sites involved in them?
- [ ] does the table of sites and their PM efforts match lists of sites for each task?
      (each site from the table is listed in all relevant tasks, and no site is listed
      only in the table or only at some task)
\end{verbatim}
\end{draft}

\begin{workpackage}[id=UI,wphases=0-48,swsites,
  title=User Interfaces,
  lead=SR,
  PSRM=26,  % Sage-Jupyter interface, sphinx documentation dynamic documentation and exploration system
  UVRM=2,   % Sage-Jupyter interface
  JURM=22,  % Jacobs: active documents
  USHRM=6, % Supporting reproducible data science and sharing of models
  LLRM=4, % Jupyter 
  SARM=18, % GAP
  UKRM=2, % Singular
  UBRM=28,  % Pari
  USORM=16, % Southampton, micromagnetic VRE some contribution (1 month) to 3d visualisation
  SRRM=28,
  USRM=4, % University of Silesia, 3d without subcontracting
  swsites]    % rotate partner logos so that table fits on page.

\begin{wpobjectives}
  The objective of this work package is to provide modern, robust,
  and flexible user interfaces for computation, supporting real-time
  sharing, integration with collaborative problem-solving,
  multilingual documents, paper writing and publication, links to
  databases, etc.
\end{wpobjectives}

\begin{wpdescription}
  Project \Jupyter is describes in Section~\label{sec:jupyter}. In
  this work package, we will add new functionality to the \Jupyter
  Notebook that fosters excellence in computational science and
  research. In particular, we will push towards reproducible and
  effective science by allowing to create structured documents (such
  as reports, books, theses) from Notebooks, and by allowing those
  notebooks to be re-executed as self-contained regression tests. We
  will also unify the notebook infrastructure used in \Sage with
  \Jupyter, push forward dynamic documentation exploration
  capabilities, and work towards concurrent multi-user editing of
  Notebooks.

  The last tasks in this work package are focused on an
  \emph{application of the \Jupyter notebook technology} for education
  and research.

\TOWRITE{Hans-Petter / Marcin?}{Say something about notebook for
  education and planned demonstrators [notebooks]?}

To demonstrate the power of the \TheProject environment to accelerate
computational science, deliver better value for money and make
computational science more robust, we will put together a
micromagnetic VRE
(\ref{sec:introduction-micromagnetic-vre-demonstrator}) as a
demonstrator.

\TOWRITE{POD,JC}{Somewhere in this package there needs to be a mention of the experience acquired with Knowls, developed for LMFDB. -- POD}
\end{wpdescription}

\begin{tasklist}
\begin{task}[title=Uniform notebook interface for all interactive
    components,id=ipython-kernels,lead=PS, partners={SR,UK,USH,USO,LL,SA,UV}]
  In this task, we will implement \Jupyter interfaces for the
  interactive computation components of \TheProject, including \GAP,
  \PariGP, \Sage, and Singular. A first release
  \delivref{UI}{ipython-kernels-basic} will focus on basic functionality,
  and a second release \delivref{UI}{ipython-kernels} will cover advanced
  features like 3D graphics or transparent documentation browsing (as
  live worksheets whenever relevant).

  % Note from William: my student Andrew Ohana just mostly did
  % something like that for IPython, but then stopped.  Anyway, it's
  % very do-able based on a summer project from another student and a
  % bunch of work I did with THREE.js for SMC.

  \Sage itself will require a specific treatment as it already has a
  notebook interface. Its development started about at the same time
  as the \Jupyter notebook, with similar target features but a
  different agenda: the \Sage notebook had to be available very quickly
  to solve pressing needs of the \Sage community; instead the \Jupyter
  notebook was to take its time and build robust foundations from the
  ground up. The two projects have exchanged a lot, and the \Jupyter
  notebook, which benefits from a much larger user base and thus
  developer pool, has mostly caught up with the \Sage notebook in terms
  of functionality. It is thus time for the \Sage community to outsource
  this key but non disciplinary component and phase out the \Sage
  notebook in favor of the \Jupyter notebook.

  % In charge: Jupyter dev + dev in Orsay + community?
  The \Sage and \Jupyter convergence \delivref{UI}{ipython-kernel-sage} will
  require:
  \begin{compactitem}
  \item Robust migration path and tools for \Sage worksheets,
  \item Support for math, 2D, and interactive 3D scene visualization,
    % \item Bundling of the \Jupyter notebook and its dependencies within
    %   the Sage distribution. DONE
  \item Import and export of ReST documents, with full support for
    \Sage's specific roles (math, ...),
  \item Support for remote \Sage kernel, typically on the cloud, or
    running with a different Python version (\Sage as a library),
  \item A migration path for interactive widgets implemented with
    \Sage's \texttt{@interact} functionality.
  \end{compactitem}

  Joint meetings and visits between the developers of \Jupyter and of
  the computing components will be a key asset for this task.

\end{task}

\begin{task}[id=notebook-collab,title=Notebook improvements for collaboration,lead=SR, partners={PS,USH,JU,USO}]
  In this task, we will further improve tools for collaboration between
  authors of shared \Jupyter notebooks.

  Version control tools, such as Git and Mercurial, have become an integral part of open and
  collaborative science and software. Version control tools allow reviewing proposed changes via
  different tools, and resolving conflicting changes with merge tools. \Jupyter notebook documents,
  being text files, are relatively well suited to tracking in version control. However, being
  structured JSON data, diffing and merging are difficult. Tools shall be developed to provide
  better support for visual diffing and merging of Notebook documents, and integrated into existing
  version control workflows \delivref{UI}{jupyter-collab}.

  Given the interactive nature of \Jupyter notebooks, live collaboration, where multiple authors
  work on the document simultaneously, as in Google Docs, is particularly desirable. The addition
  of potentially shared execution adds both value and challenge to live collaboration. Various
  projects have added some amount of live sessions from outside (\SMC, Colaboratory), but
  outside the core project. There are many different aspects of collaboration to explore,
  including shared or separate execution for authors collaborating on a single notebook,
 \delivref{UI}{jupyter-collab} to indicate not only authorship,
  but which author triggered which execution, and other challenges.
  Various avenues for live session collaboration will be explored for integration into \Jupyter itself
  \delivref{UI}{jupyter-live-collab}.
\end{task}




\begin{task}[id=notebook-verification,title=Reproducible Notebooks,lead=SR, partners={PS,USO}]
  In this task, we will develop tools that allow to re-execute
  notebook documents with automated regression testing. The computed
  output will be compared against the stored output, and deviations
  reported as assertion errors.

  Notebooks are used in a variety of contexts, like training and
  teaching material (tutorials, documentation, books) or computer
  experimentation logbooks, where reproducibility is critical: the
  notebooks shall remain functional and correct in the long run, even
  when the underlying computational software or infrastructure changes
  over time or across platforms.

  This task is a critical step toward reproducibility, allowing the
  notebook author to get an immediate notice when, e.g., a backward
  incompatible change occurs. It becomes even possible to anticipate
  such situations upstream by including important notebooks directly
  in the automated test suite of the computational software, giving an
  easy way for casual users to contribute regression tests.

  Technically speaking, \Jupyter notebooks store outputs as rich mime-type structures,
  with JSON metadata. Using this metadata, it will be possible to express
  expectations of output, allowing more flexible and powerful tests
  than direct text comparison \delivref{UI}{jupyter-test}.
  Prior work has been done in \Sage for ReST files, e.g. \lstinline{sage -t notebook.rst},
  and this model will be extended to notebooks.
\end{task}

\begin{task}[id=sage-sphinx, title=Refactor \Sage's \Sphinx documentation system, lead=PS,PM=6, partners={SR,UV}]
  \Sage, like \Python and many other \Python based projects, uses the
  \Sphinx documentation system. Due to particularly stringent needs,
  many layers of customization and adaptations have accumulated over
  the years, in particular for proper scaling to the sheer size of the
  Sage documentation (13k pages just for the reference manual).

  A deep refactorization (\delivref{UI}{sage-sphinx}) is critically needed to get rid of multiple
  duplication, and foster sustainability by outsourcing back to \Sphinx
  all generic aspects (parallel compilation, index generation, ...).
  % In charge: dev in Orsay or Logilab + visit of Sphinx dev  + FH
\end{task}

\begin{task}[id=dynamic-inspect,title=Dynamic documentation and exploration system,lead=PS, partners={SR,USO,UV}]
  Introspection has become a critical tool in interactive computation,
  allowing user to explore on the fly the properties and capabilities
  of the objects under manipulation. This becomes particularly acute
  in systems like \Sage where large parts of the class hierarchy is
  built dynamically, and static documentation builders like \Sphinx
  cannot anymore render all the available information.

  In this task, we will investigate how to further enhance the user
  experience. This will include:
  \begin{compactitem}
  \item On the fly generation of Javadoc style documentation, through
    introspection, allowing e.g. the exploration of the class
    hierarchy, available methods, etc.
  \item Widgets based on the HTML5 and web component standards to display
    graphical views of the results of SPARQL queries, as well as populating data
    structures with the results of such queries,
  \item \delivref{UI}{ipython-advanced-interacts} (Month 36)
    Exploratory support for semantic-aware interactive widgets
    providing views on objects of the underlying computational or
    database components. Preliminary steps are demonstrated in the
    \texttt{Larch Environment} project (see demo video on
    \url{http://www.larchenvironment.com/}) and
    \software{sage-explorer}
    (\url{https://github.com/jbandlow/sage-explorer}). The ultimate
    aim would be to automatically generate \LMFDB-style interfaces.
  \end{compactitem}
  Whenever possible, those features will be implemented generically
  for any computation kernel by extending the \Jupyter protocol with
  introspection and documentation queries.
  % In charge: \Jupyter dev + dev in Orsay + NT?
\end{task}

\begin{task}[title=Structured documents,id=structdocs,lead=JU,PM=24,lead=JU,partners={SR,USH}]

  \Jupyter notebook is basically a sequence of cells that contain either text or
  a program. The complex documents, such as books, articles or reports, require
  a richer description that covers the the structure of the document and the
  semantics of its elements. This task will investigate this problem and try to
  find a way to write these documents using the breakthroughs achieved in the
  other tasks to this workpackage.

  Several technical complementary options can be explored:
  \begin{compactitem}

  \item  MathHub.info is a portal for reading and interacting with
    "active documents" (i.e. documents that have an additional semantic
    layer that supports semantic services like definition lookup,
    type-inference, unit conversion,\ldots)

  \item  \Jupyter notebooks are essentially ``programs with
    documentation' and lack the semantical structure needed by complex
    documents.

  \item  sTeX is a semantic variant of LaTeX that can be transformed into
    OMDoc/MMT, which is the native knowledge representation format for
    active documents and machine-actionable knowledge about math and
    symbolic programs.

  \end{compactitem}

  After gathering the needs and the requirements for the writing of complex
  documents in the mathematical field, we will study these various options,
  design and build a solution that meets the expectations. This work will be
  done through an iterative process that tries to enhance little-by-little the
  already existing softwares.

\end{task}

\begin{task}[id=mathhub,title=Active Documents Portal,lead=JU,PM=12]
  We will extend the existing \url{http://mathhub.info} system to a portal for
  interacting with active/structured documents (see \localtaskref{structdocs}) and run it
  for internal use in the \TheProject first and later for general use outside the
  project. \url{MathHub.info} already provides very basic sTeX editing and versioning. For
  \TheProject we will need to extend it with the computational side based on the
  integrated format from \localtaskref{structdocs}.
\end{task}



\begin{task}[title=Visualization system for 3d data in web-notebook
,id=vis3d,lead=SR, partners={US,PS,USO}]

The \Jupyter notebook is a very attractive environment for building
user interfaces for research, but the current support for inline
visualization is limited to curve plots and 2d scalar fields. Many
scientific simulations need visualization of 3d scalar and vector
fields, as shown in figure~\ref{fig:3d-plots}. 
The amount of data can be tremendous, especially in
time-dependent problems computed in a distributed fashion over
large-scale computational clusters. Interactive inspection of such
simulations can be a valuable tool which accelerates research. For
inspection, one does not need to transfer and gather the full dataset
at each time step - getting selected computed fields on user request
or preprocessing certain quantities like cross sections with some
predefined frequency will mostly suffice.

In this task we will first investigate available technologies for fast
in-browser visualization techniques for the typical structures to be
displayed (isosurfaces, streamlines, vector fields, cross sections,
etc.).  There exists several solutions which can provide basis for
further development. Perhaps the most known, and also advanced is
\href{http://threejs.org/}{three.js} which provides basis for 3d
visualization in a web browser. Three.js is WebGL based, but it also
provides canvas based rendering for system which do not support
WebGL. It has already been experimentally deployed in Sage Cell Server
and SMC projects. Another promising technologies are visualization
libraries using exclusively OpenGL. They can be deployed in browser based
system by using of the WebGL API which is a restricted subset of the
regular OpenGL API. It can by accomplished if visualization is done
purely in GPU. \href{http://vispy.org/}{VisPy} and
\href{http://glumpy.github.io/}{glumpy} projects have found GPU-only
solutions for common visualization objects (lines, arrows, markers,
text, iso-lines, iso-surfaces, text, etc) where data does not exit the
GPU. Furthermore, the vispy project already offers an experimental
interface with Jupyter notebook that could be extended to cope with
our specifications. Through this tight collaboration with the authors,
\TheProject could benefit from both dedicated and state-of-the art
visualization techniques.
\end{task}


\begin{task}[title=Visualization of 3d fluid dynamics data in web-notebook
,id=cfd-vis,lead=SR, partners={US,PS,USO}]

We propose to let computational fluid dynamics (CFD) be a driving
application for the development of 3d visualization in \Jupyter
notebooks (\taskref{UI}{vis3d}) since CFD is one of the most demanding
cases of scientific visualization. The the same time this task will be
demonstrator for (\taskref{UI}{vis3d}). 

Successfully handling CFD makes the tool immediately applicable to a
range of other fields such as heat transfer, electromagnetics,
material science, and 3d algebraic structures in
mathematics. Simulations are started inside the notebook and executed
on HPC clusters. This approach will significantly lower the threshold
for using parallel computing codes that can be hard to install
correctly on local workstations. Such use cases with 3d visualization
will greatly extend the potential applications of the \Jupyter
notebook concept throughout science and engineering.

As example code for a 3d live web notebook with fluid dynamics
simulations, we will use the Lattice Boltzmann solver which is under
development at the University of Silesia:
\href{http://sailfish.us.edu.pl/}{Sailfish}.  This code is an advanced
Lattice Boltzmann solver designed from the ground up for distributed
system of GPU compute clusters. It is implemented predominantly in
Python, and it uses run-time code generation techniques to
automatically build optimized code for CUDA and OpenCL devices. Since
running Sailfish requires specialized hardware, it is reasonable to
use it on dedicated HPC installations.


\end{task}


\begin{task}[id=mws,title=Math Search Engine,lead=JU,PM=10]
  \TODO{MK: write along the following lines We already have a search engine, therefore we
    only need to build harvesters -- i.e. programs that generate keywords and Formula
    URL/pairs from the contents -- \delivref{UI}{mws} and \delivref{UI}{notebooksearch}. For
    \delivref{UI}{cassearch}, this is also true in principle, but the formulae in code can
    take many more forms and the notion of a hit URL is not as clear.}
\end{task}

\begin{task}[id=simulagora,title=Simulagora collaboration,lead=LL,PM=2]
  Comparing the collaboration workflow of Simulagora, SageMathCloud and GitHub.
\end{task}

\begin{task}[lead=UB,title=Common option system for various displays in Sage,id=Sage-display,PM=12,partners={UB}]
  Given a mathematical object, it often has various possible representation on a
  computer. From raw text to \LaTeX, from simple 2d picture to
  a complicated 3d animation.

  In this task, we provide a uniform option system for displaying object within \Sage (raw text, \LaTeX,
  tikz, matplotlib, jmol, tachyon, \ldots). We will implement some of the missing display and will benefit
  of the work done in task~\taskref{UI}{cfd-vis}.
\end{task}

\begin{task}[lead=USO,title=Case study: micromagnetic VRE built from
  \TheProject,id=oommf-py-ipython-attributes,PM=6,partners={SR,USH},wphases=9-15]
  % 6 person months

  In this task, we use the \TheProject architecture to assemble a
  virtual research environment software tailored for the large
  micromagnetic research community
  (\ref{sec:introduction-micromagnetic-vre-demonstrator}).

  The micromagnetic VRE will be based on the \Jupyter Notebook, the
  Python interface to the micromagnetic simulation library OOMMF
  (\taskref{component-architecture}{oommf-python-interface}),
  and the additional features added to \Jupyter in this work
  package. 

  The \Jupyter Notebook environment allows to host, execute and
  document the Python-based OOMMF simulation in an executable
  document. In this interactive environment, objects can be displayed
  using various representations, including, for example, textual
  representation (i.e. strings), bitmap images and SVG (vector
  graphics) files. We will create functionality so that magnetisation
  vector field objects can be presented as a rendered 3d and 2d-view
  of the magnetisation field (Fig.~\ref{fig:3d-plots}), and similar
  features for scalar fields such as field components and energies for
  static and time dependent data (linking to
  \localtaskref{cfd-vis}). This allows computational steering and the
  interactive exploration of the behavior of magnetic nanostructures.

  Beyond that, the \Jupyter Widgets allow the creation of graphical
  user interface (GUI) elements, and we will generate code to
  display these widgets on demand to (i) set up micromagnetic
  simulations using a GUI, and (ii) assist in common post-processing
  simulation results. Recent pilot work has shown that it is possible
  to make \Jupyter Widgets interact with the Python interpreter
  session and this allows to activate a GUI-like (widget based)
  interface when desired but to quickly return to the interpreter
  prompt, taking forward the results (data) from the GUI session
  \cite{IPython-widget-GUI-demo-youtube-2014} and providing a
  continuous path from scripting to GUI. We believe that having the
  ability to mix and match GUI-based and command driven analysis
  combines the best of both approaches, caters for users' preferences,
  and provides significant additional value.

  The deliverable for this task is the open source micromagnetic VRE
  software (\delivref{UI}{oommf-nb}).
\end{task}

\begin{task}[lead=UB,title=Python/Cython bindings for Pari,PM=16,id=pari-python]
  \Pari is a C-library and GP is its standalone interpreter. Partial
  Python/Cython bindings are provided by Sage.

  The task aims to develop an independent Python/Cython library that would provide
  bindings for \PariGP and which would tightly be developed within the \PariGP team.

  The deliverable for this task is \delivref{UI}{pari-python-lib}.
\end{task}

\begin{task}[lead=USO,title=Demonstrator: micromagnetic VRE notebooks,
  id=oommf-tutorial-and-documentation, PM=6, partners={SR,PS},wphases=15-21]
  % 5 person months + 1 month co-investigator [Ian Hawke's experience]

  The purpose of the micromagnetic VRE
  (\localtaskref{oommf-py-ipython-attributes}) is to enable excellent
  computational research. To maximise the value of this grant's investment for the
  community, we will not carry out micromagnetic research but instead
  produce a set of executable notebooks using the micromagnetic VRE
  to demonstrate its power and applicability.

  We will create executable notebook documents
  (\delivref{UI}{oommf-nb-documentation}) within the micromagnetic VRE
  including (i) a new tutorial on computational micromagnetics with
  OOMMF, (ii) the complete documentation of the \texttt{OOMMF-Py}
  library (\taskref{component-architecture}{oommf-python-interface}),
  and (iii) a set of typical micromagnetic case studies. The tutorial,
  in terms of content, will take guidance from the tutorial provided
  for Nmag \cite{Nmag-tutorial-url} and will introduce the additional
  features of the \Jupyter-driven micromagnetic VRE. We expect this
  substantial and executable documentation of the micromagnetic VRE to
  become the standard resource that introduces researchers to
  computational micromagnetics, in particular through the online
  portal (\localtaskref{oommf-nb-ve}).

  %% This block is about the benefits of using the notebook. It should
  %% go somewhere else in more generic form:
  %The output of this activity will deliver multiple benefits:
  %providing a systematic introduction to \texttt{OOMMF-py} suitable for both
  %those users (i) new to micromagnetic modelling and those (ii) new to
  %the \texttt{OOMMF-py} interface. Because the documentation is developed in an
  %\Jupyter notebook, the documents are executable. For new learners
  %this is a great simplification because they can skip through the
  %given document and execute the given examples there and then: at the
  %moment, this is a process of manually writing a script, or locating
  %it in the directory structure of files, then executing this,
  %subsequently opening and processing the data files, etc. In the new
  %model, this end-to-end simulation will start from specifying the
  %material parameters in the notebook (all of this is given in the
  %tutorial), to running the simulation in the notebook to processing
  %of computed data while the simulation runs (or subsequently) in the
  %notebook; thus providing one virtual research environment, with all
  %the associated benefits of making best use of the scientist's time
  %using the tool and environment.

\end{task}

\begin{task}[lead=USO,id=oommf-nb-ve,title=Online portal for
  micromagnetic VRE demonstrator,PM=3, partners={SR,JU},wphases=21-24]

  % 3 person months
  Recently, a TeMPorary \Jupyter NoteBook (TMPNB) has been made
  available (at \href{http://tmpnb.org}{http://tmpnb.org}) that allows
  anybody to open this URL and use their very own \Jupyter notebook
  for quick calculations and tests online. We will provide such a
  portal (\delivref{UI}{oommf-nb-tmp}) which provides the micromagnetic VRE for anonymous use. This
  service allows users to execute the demonstrator tutorial and
  documentation notebooks
  (\localtaskref{oommf-tutorial-and-documentation}) and run the
  calculations in real time on the web server, without having to
  install any software on their own machine.  This web service will be
  based on Docker \cite{Docker} virtualisation technology and we will
  make available the scripts to create VirtualBox \cite{Virtualbox}
  images, and Docker containers. The same virtual machine images can
  also be used for Cloud hosted computing services.

  We request \euro{6000} to purchase a machine to provide these
  services (shared memory, 64 cores, 128GB RAM, Solid-state drive (SDD)
  to make the system more responsive). 
  %This machine will also support
  %the regression testing and continuous integration (see task
  %\taskref{dissem}{dissemination-of-oommf-nb-virtual-environment}).
  %Setup and
  %maintenance of the machine is part of this work task.
\end{task}



%<<<<<<< HEAD
%=======
%As example code for a 3d live web notebook with fluid dynamics
%simulations, we will use the Lattice Boltzmann solver which is under
%development at the University of Silesia:
%\href{http://sailfish.us.edu.pl/}{Sailfish}.  This code is an advanced
%Lattice Boltzmann solver designed from the ground up for distributed
%system of GPU compute clusters. It is implemented predominantly in
%Python, and it uses run-time code generation techniques to
%automatically build optimized code for CUDA and OpenCL devices. Since
%running Sailfish requires specialized hardware, it is reasonable to
%use it on dedicated HPC installations.
%
%In this task we will first investigate available technologies for fast
%in-browser visualization techniques for the typical structures to be
%displayed (isosurfaces, streamlines, vector fields, cross sections,
%etc.).  There exists several solutions which can provide basis for
%further development. Perhaps the most known, and also advanced is
%\href{http://threejs.org/}{three.js} which provides basis for 3d
%visualization in a web browser. Three.js is WebGL based, but it also
%provides canvas based rendering for system which do not support
%WebGL. It has already been experimentally deployed in Sage Cell Server
%and SMC projects. Another promising technologies are visualization
%libraries using exclusively OpenGL. They can be deployed in browser based
%system by using of the WebGL API which is a restricted subset of the
%regular OpenGL API. It can by accomplished if visualization is done
%purely in GPU. \href{http://vispy.org/}{VisPy} and
%\href{http://glumpy.github.io/}{glumpy} projects have found GPU-only
%solutions for common visualization objects (lines, arrows, markers,
%text, iso-lines, iso-surfaces, text, etc) where data does not exit the
%GPU. Furthermore, the vispy project already offers an experimental
%interface with Jupyter notebook that could be extended to cope with
%our specifications. Through this tight collaboration with the authors,
%\TheProject could benefit from both dedicated and state-of-the art
%visualization techniques.
%
%\end{task}
%
%
%\begin{task}[id=mws,title=Math Search Engine,lead=JU,PM=10]
%  \TOWRITE{MK}{write along the following lines: We already have a search engine, therefore we
%    only need to build harvesters -- i.e. programs that generate keywords and Formula
%    URL/pairs from the contents -- \delivref{UI}{mws} and \delivref{UI}{notebooksearch}. For
%    \delivref{UI}{cassearch}, this is also true in principle, but the formulae in code can
%    take many more forms and the notion of a hit URL is not as clear.}
%\end{task}
%
%\begin{task}[id=simulagora,title=Simulagora collaboration,lead=LL,PM=2]
%  Comparing the collaboration workflow of Simulagora, SageMathCloud and GitHub.
%\end{task}
%>>>>>>> 8f0d5df6e3fe2a851bb20d5941c216eb743b79cd
%
\end{tasklist}

\begin{wpdelivs}
  \begin{wpdeliv}[id=adstex,due=6,nature=R,dissem=PU,lead=JU]
    {Active Documents based on sTeX}
  \end{wpdeliv}
    \begin{wpdeliv}[id=mws,due=9,nature=OTHER,dissem=PU,lead=JU]
      {Full-text search (formulae + Keywords) over LaTeX-based documents
        (e.g. the arXiv subset)}
    \end{wpdeliv}
    \begin{wpdeliv}[id=mathhub-editing,due=12,nature=DEM,dissem=PU,lead=JU]
      {Distributed, Collaborative, Versioned Editing of Active Documents in MathHub.info}
    \end{wpdeliv}
  \begin{wpdeliv}[due=14,id=ipython-kernels-basic,dissem=PU,nature=OTHER,lead=PS]
      {Basic \Jupyter interface for GAP, \PariGP, \Sage, Singular}
  \end{wpdeliv}
  \begin{wpdeliv}[due=12,id=ipython-kernels,dissem=PU,nature=OTHER,lead=PS]
      {Full featured \Jupyter interface for GAP, \PariGP, Singular}
  \end{wpdeliv}
  \begin{wpdeliv}[due=24,id=pari-python-lib,dissem=PU,nature=OTHER,lead=UB]
	  {Python/Cython bindings for \PariGP}
  \end{wpdeliv}

  \begin{wpdeliv}[due=12,id=ipython-kernel-sage,dissem=PU,nature=DEM,lead=PS]
      {\Sage notebook / \Jupyter notebook convergence}
  \end{wpdeliv}
    \begin{wpdeliv}[due=15,id=oommf-nb,dissem=PU,nature=OTHER,lead=USO]
      {Micromagnetic VRE software completed}
    \end{wpdeliv}

    \begin{wpdeliv}[id=notebooksearch,due=18,nature=OTHER,dissem=PU,lead=JU]
      {Full-text search (Formulae + Keywords) over Notebooks}
      We want to be able to search over the Jupyter-based Notebooks from
      \localtaskref{structdocs}.
\end{wpdeliv}
    \begin{wpdeliv}[id=adcomp,due=18,nature=DEM,dissem=PU,lead=JU]
      {In-place computation in active documents (context/computation)}
    \end{wpdeliv}

  \begin{wpdeliv}[due=18,id=jupyter-test,dissem=PU,nature=OTHER,lead=SR]
      {Facilities for running notebooks as verification tests}
  \end{wpdeliv}

  \begin{wpdeliv}[due=12,id=jupyter-collab,dissem=PU,nature=OTHER,lead=SR]
      {Tools for collaborating on notebooks via version-control}
  \end{wpdeliv}
    \begin{wpdeliv}[due=21,id=oommf-nb-documentation,dissem=PU,nature=DEM,lead=USO]
      {Micromagnetic VRE tutorial and documentation notebooks}
    \end{wpdeliv}
    \begin{wpdeliv}[id=jupyter-import,due=24,nature=DEM,dissem=PU,lead=JU]
      {Notebook Import into MathHub.info (interactive display)}
    \end{wpdeliv}
    \begin{wpdeliv}[due=24,id=oommf-nb-tmp,dissem=PU,nature=DEC,lead=USO]
      {Demonstrator online portal available}
    \end{wpdeliv}
  \begin{wpdeliv}[due=24,id=vis3d,dissem=PU,nature=OTHER,lead=SR]
      {\Jupyter extension for 3d visualisation}
  \end{wpdeliv}
  \begin{wpdeliv}[due=36,id=cfd-vis,dissem=PU,nature=OTHER,lead=SR]
      {Computational Fluid dynamics visualization in web notebook}
  \end{wpdeliv}

   \begin{wpdeliv}[id=cassearch,due=30,nature=OTHER,dissem=PU,lead=JU]
      {Formula search in CAS programs and Software Modules}
    \end{wpdeliv}
  \begin{wpdeliv}[due=36,id=jupyter-live-collab,dissem=PU,nature=OTHER,lead=SR]
      {Exploratory support for live notebook collaboration}
  \end{wpdeliv}
  \begin{wpdeliv}[due=24,id=sage-sphinx,dissem=PU,nature=OTHER,lead=PS]
      {Refactorization of \Sage's \Sphinx documentation system}
  \end{wpdeliv}
  \begin{wpdeliv}[due=36,id=ipython-advanced-interacts,dissem=PU,nature=DEM,lead=PS]
      {Exploratory support for semantic-aware interactive widgets providing views on objects
      represented and or in databases}
  \end{wpdeliv}
\begin{wpdeliv}[id=nbad-search,due=42,nature=OTHER,dissem=PU,lead=JU]
  {Search from Notebooks/Active Documents Interface} Often it is important to have some
  local context to inform search, therefore search from the notebook/active documents
  interface is an interesting and useful development target.
\end{wpdeliv}
% communication with live computing process
% post simulation data analysis module
% visualization of vector and scalar fields
% editor for geometry and boundary conditions  on regular meshes


  % Shared \Jupyter sessions embedded in voice-over-IP or
  % teleconference calls or reciprocally.
  %
  % NOTE: This task is probably outdated by appear.in which makes
  % video-conferencing in the browser trivial
  %
  % \delivref{ipython-collaborative}
  % Eugen Dedu:
  % I think such a module can be thought of as a screen-capturing
  % module, i.e. allow Ekiga to capture the screen of a Sage user (this
  % is currently not possible).  This is not a difficult task to do.
  % Julien Puydt: ekiga can do that since something like 2008 with my
  % experimental gstreamer plugin, and I shall be able to present
  % interesting sample code to the ekiga-devel mailing-list in something
  % like two-three weeks (after I'm done with my students), which will
  % hopefully be part of the next version.
  %
  % But as Nicolas noted in his answer, some kind of interactive session
  % where people can share a sage session would be better.
  %
  % I think the feature decomposes in the following pieces:
  % - IPython should have a way to share sessions between several
  % participants using an open and standard protocol ;
  % - ekiga should implement it.
  %
  % In my opinion ekiga, because of its dependency on ptlib and opal
  % libraries and the use of complex protocols like SIP and H323, needs
  % highly technical people.  Students cannot help much, but engineers
  % are appropriate.
  \end{wpdelivs}
\end{workpackage}

%%% Local Variables:
%%% mode: latex
%%% TeX-master: "../proposal.tex"
%%% End:

%  LocalWords:  workpackage wphases Jupyter OOMMFNB wpobjectives wpdescription TOWRITE
%  LocalWords:  Paderborn IPython KBase Hackathon Quantopian Logilab Enthought Authorea
%  LocalWords:  emph Jupyther nanostructures tasklist delivref THREE.js texttt diffing
%  LocalWords:  notebook-collab jupyter-collab Colaboratory jupyter-live-collab Javadoc
%  LocalWords:  notebook.rst Knowls structdocs localtaskref Needs.rst CTypes Cython Nmag
%  LocalWords:  oommf-python-interface OOMMF-py-raw micromagnetic oommf-py magnetisation
%  LocalWords:  Micromagnetic-Standardproblem-3 oommf-py-ipython-attributes vispy taskref
%  LocalWords:  oommf-nb IPython-widget-GUI-demo-youtube-2014 oommf-nb-documentation Dedu
%  LocalWords:  oommf-tutorial-and-documentation modelling micromagnetics oommf-nb-ve mws
%  LocalWords:  TeMPorary oommf-nb-tmp oommf-nb-virtual Virtualbox Cloudhosted dissem
%  LocalWords:  dissemination-of-oommf-nb-virtual-environment wpdelivs wpdeliv Eugen tikz
%  LocalWords:  Ekiga Puydt gstreamer ekiga-devel ptlib compactitem refactorization numpy
%  LocalWords:  cfd-vis paraview ldots electromagnetics isosurfaces notebooksearch adstex
%  LocalWords:  cassearch simulagora Simulagora mathhub-editing adcomp nbad-search glumpy
%  LocalWords:  swsites visualisation introduction-micromagnetic-vre-demonstrator mathhub
%  LocalWords:  matplotlib jmol maximise virtualisation
