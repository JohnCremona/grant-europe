\addtocounter{wpno}{1}
\begin{Workpackage}{\thewpno}
\wplabel{wp:x}
\WPTitle{\wpname{\thewpno}}
\WPStart{Month 1}
\WPParticipant{SA}{1}

\begin{WPObjectives}
  The objective of this work package is to provide a modern, robust,
  and flexible interface for computation, supporting real-time
  sharing, integration with collaborative problem-solving,
  multilingual documents, paper writing and publication, links to
  databases etc.
\end{WPObjectives}

\begin{WPDescription}
  \TODO{What is a notebook interface}

  \WRITEME{IPython people}{improve this draft presentation of }

  IPython is a leading notebook interface in the world of interactive
  computations, and use massively by biologists, physicists, \TODO{and
    outside academia!}. Originally tailored for Python, it has been
  language agnostic, and can communicate through a standardized
  interface to various computation kernels \TODO{cite a few}. It can
  transparently run kernels locally or remotely (e.g. on the cloud),
  and has built in support for HPC.


  .. TODO:: include here everything about this topic in Needs.rst
\end{WPDescription}

\TODO{Wherever relevant, create tickets with details, and refer to
  them here.}

\begin{WPDeliverables}
  \begin{itemize}
  \item \ref{del:ipython_GAP} (Month ???) IPython notebook interface for GAP.
    % In charge: IPython dev + GAP dev
  \item \ref{del:ipython_GAP} (Month ???) IPython notebook interface for Singular.
    % In charge: IPython dev + Singular dev
  \item \ref{del:ipython_GAP} (Month ???) IPython notebook interface for Pari.
    % In charge: IPython dev + Pari dev
  \item \ref{del:ipython_sage} (Month 24\TODO{earlier?}): Sage
    notebook / IPython notebook convergence:\\
    The development of the Sage and IPython notebooks started about at
    the same time in 200? with similar target features but a different
    agenda: the Sage notebook had to be available very quickly to
    solve pressing needs of the Sage community; instead the IPython
    notebook was to take its time and build robust foundations from
    the ground up. The two projects have exchanged a lot, and the
    IPython notebook, which benefits from a much larger user base and
    thus developer pool, has mostly caught up with the Sage notebook
    in terms of functionality. It's thus time for the Sage community
    to outsource this key but non disciplinary architecture element
    and phase out the Sage notebook in favor of the IPython notebook.

    \begin{enumerate}
    \item Complete support for using the IPython notebook as a user
      interface for Sage, including:
      \begin{itemize}
      \item Support for Math, 2D, and 3D output.
      \item Bundling of the IPython notebook and its dependencies within
        the Sage distribution.
      \item Support for remote Sage kernel, typically on the cloud, or
        running with a different Python version (Sage as a library).
      \item One click access to the Sage documentation, as live
        worksheets.
      \end{itemize}
    \item Robust migration tools for Sage worksheets.
    \item Import (and export?)  of ReST documents, with full support for
      Sage's specific roles (math, ...)
    \item Support for interactive widgets implemented with Sage's
      \texttt{@interact} functionality.
    \end{enumerate}
    % In charge: IPython dev + dev in Orsay + community?
    \TODO{Should the four items above be merged in a single deliverable?}
  \item \ref{del:ipython_usability} (Month 24) IPython usability
    \begin{itemize}
    \item Multilingual notebooks?
    \item Improved 2D/3D graphics: maybe architecture for integrating
      VPython, vispy, ...?
    \end{itemize}
    % In charge: IPython dev
  \item \ref{del:ipython_advanced_interacts} (Month 36) Exploratory
    support for interactive widgets as demonstrated in the \texttt{Larch
      Environment} project (see demo vidéo on
    \url{http://www.larchenvironment.com/})
    % In charge: IPython dev
  \item \ref{del:ipython_docking} (Month ???) Heavyweight (e.g. QT
    based and not web based) user interface with docking support in
    the style of \texttt{Spyder}
    \url{https://code.google.com/p/spyderlib/}. Can possibly be
    implemented by extending the QT IPython console, or by letting
    \texttt{Spyder} use the IPython protocol (\TODO{If that's not yet
      the case}).
  \item \ref{del:ipython_dynamic_doc} (Month ???) Dynamic
    documentation system: java API style documentation, allowing
    e.g. the exploration of the class hierarchy, generated on the fly
    by introspection.  If possible this should be implemented
    generically for any computation kernel by extending the IPython
    protocol with introspection and documentation queries.

    Rationale: in Sage, large parts of the class hierarchy is built
    dynamically, hence static documentation builders like Sphinx can't
    render all the available information.
    % In charge: IPython dev + dev in Orsay + NT?
  \item \ref{del:ipython_test} (Month ???) Support for tested
    notebooks.\\
    The writer specifies the expected outputs, e.g. in text format,
    and can check at any point that the full execution of the notebook
    yields exactly the expected output, as can be done with e.g. ReST
    files in Sage: \lstinline{sage -t notebook.rst}
  \item \ref{del:ipython_structured_documents} (Month ???) Support for writing
    interactive structured documents, and in particular papers, books,
    experimentation log books and reports, presentations, course
    notes, etc, with the following features:
    \begin{itemize}
    \item Static printed/PDF/HTML version and interactive version.\\
      Achieved by either importing or exporting document files in some
      standard format (LaTeX, ReST, Markdown, ...).
    \item Tests (see above).
    \item Native folding support for sections and the like, with mouse
      and keyboard commands.
    \item Collaborative edition.
    \item Version control.
    \end{itemize}
  \end{itemize}
  % TODO: logilab: inclusion of database queries and views
\end{WPDeliverables}
\end{Workpackage}
