\begin{draft}
\TOWRITE{DP (Work Package Lead)}{For WP leaders, please check the following (remove items
once completed)}
\begin{verbatim}
- [ ] have all the tasks in this Work Package a lead institution?
- [ ] have all deliverables in the WP a lead institution?
- [ ] do all tasks list all sites involved in them? 
- [ ] does the table of sites and their PM efforts match lists of sites for each task?
      (each site from the table is listed in all relevant tasks, and no site is listed
      only in the table or only at some task)
\end{verbatim}
\end{draft}



\begin{workpackage}[id=social-aspects,wphases=0-48,
  title=Social Aspects,
  lead=UO,
  UORM=27,USHRM=8, USORM=6] 

%\TOWRITE{DP/UM}{workpackage Social Aspects}
% At Nicolas' request I'm having a go so there is something here
% SL

\begin{wpobjectives}

The processes by which mathematical knowledge and mathematical
software are developed, validated and applied are quite
distinctive. In other sciences, the universe provides ``ground truth''
and the scientific texts or theories can be validated against that by
experiment. In mathematics the text is the ground truth. The
traditional model of mathematical research is a mathematician, or a
small group of mathematicians, standing around a blackboard, producing
a proof they would ``clean up': removing all traces of the process
by which it had been discovered and then submit the ``clean'' text to
their peers for review.

Mathematicians have adopted new technology is a range of ways: email
and shared documents are used to collaborate on problem-solving and
writing; larger ``crowdsourcing'' \cite{polymath_SIAM, PolymathBlog},
arrangements pull together diverse experts; symbolic computation
tackles huge routine calculations; and computers check proofs that are
just too long and complicated for any human to comprehend. This
technology is both revealing (since email messages, version control
systems and bulletin boards can be analyzed) and altering the ways in
which mathematicians collaborate.

In an EPSRC funded project ``The Social Machine of Mathematics''
Martin and others are bringing rigorous methods from the social 
sciences together to study these processes. This brings up the following
objectives: 
\begin{compactitem}
\item to bring the insights of this and similar projects into the
  design of \TheProject VRE, ensuring that it supports the ways in
  which mathematicians really work, and not just the way software
  developers or indeed mathematicians, think they do;
\item to extend this work to study the collaborative processes of free
  open source (mathematical) software development to produce
  guidelines for best practice and ideas for how existing processes
  can extend to a ``system of systems''.
\end{compactitem} 
\end{wpobjectives}

\begin{wpdescription}


``Crowdsourcing'' -- fine grained collaborative development of ideas,
  proofs or software is a common theme to both objectives. The purpose
  of a VRE is to allow effective crowd-sourcing of computationally
  supported mathematical theorems, while free software development in
  inherently a collaborative process, and we wish to study the best
  ways of allowing it to scale.

In a sense mathematics has been a crowdsourced endeavour since at
least the foundation of the Royal Society in the seventeenth century. 
The first scientific journals were published collections of letters
received, posing questions and observations and offering solutions. 
Although limited to the speed of physical post, this model had much in
common with the public email lists that underpinned collaborative
software development in the 1990s.

In recent years, the internet and critical tools such as distributed
version control have supported much more widespread and finer-grained
crowdsourcing in software development, and even more recently in
mathematics in online mathematics communities, such as Math-overflow
\cite{mathoverflow} and in Polymath Projects \cite{polymath_SIAM,
  PolymathBlog}.  Effectively supporting and encouraging ``Mutual
crowdsourcing'' is the main driving force in developing and
maintaining of any large-scale open-source virtual research
environment.

In this workpackage we will take on board and continue the study of
crowdsourcing from the ongoing project in Oxford, and broaden it to
study software development more deeply.

\end{wpdescription}

\begin{tasklist}
\begin{task}{title=Social Science Input to
    Design,id=social-input,lead=UO,
    partners={UO}}
The purpose of this task is to ensure that the design of \TheProject
VRE reflects the lessons learned by social scientists stufying the
ways in which mathematicians actual collaborate and work. Since UO and
Martin in particular are already central in the community
working in this area, we are well placed to ensure that this happens. 

As soon as the project begins, team members at UO will combine their
own work with a review of the public literature, and identify and
meet with key research groups in this area, to distill relevant
current knowledge for use in the design phases of other parts of the
project. They will  present the lesson
learned at project meetings and workshops and deliver it as a report
in month 3.

After that, they will monitor the further development of this area and
ensure that any new insights are communicater promptly to the rest of
the project. This will be synthesised for archival purposes into two
further reports in months 24 and 42. 

\end{task}

\begin{task}{title=Implications of VREs for Publication,id=social-output,lead=UO,partners={UO}}

A key aspect of the \TheProject VRE is support for the full life-cycle
of mathematical research, up to, and after publication of
results. While it is necessary to support established models of
publication which are central to mathematical practice and academic
life, it is also appropriate to explore whether new models for the
distribution of mathematical results might be enabled by VRE
technologies and more effective for new forms of mathematical results.

The current model for distribution of scientific output stems from an
era when the printing press was dominant. The process has become
formalized through peer review and publication of journals. The PDF
format wide;y used for distribution of documents reflects the
\textit{status quo}, that a
scientific paper is a written as if for printing and remains an
unchanging document. In scientific blogging we are seeing that more
rapid propagation of ideas can occur when the constraints of the
printed format are relaxed, however, these routes lack the formalization
that ensures (usually) fair attribution of ideas and commentary. 

We will prototype and evaluate tools and ideas for distribution of
scientific knowledge that don't rely on a static format and allow for
the full spectrum of scientific debate.
The tools will encourage
proper credit attribution through encouraging sharing of attribution
for ideas, software and data. This will intregrate with work in
WorkPackage ?? concerning attribution and citability for mathematical databases.

Current ideas for tools we might prototype include live posters for
distribution of knowledge, designed for integration with either large
touch screens or smaller tablets \localdelivref{social-poster} and
augmentations to the Jupyter project with facilities for providing comment
on notebooks to encourage debate on mathematical and computational
ideas \localdelivref{jupyter-comment}.
\end{task}


\begin{task}[title=Survey on software development,id=datacollection]
We will survey the data needed to assess development models of
large-scale academic open-source projects, such that the probable
correlation between the size of the atomic contribution vs. the speed
of the contribution making it into the code, and collect appropriate
statistical data. The latter will require non-trivial amount of
programming work, even only for the test system, \Sage.
\end{task}

\begin{task}[title=Mechanism Design for free software development,id=decisionmaking]


While crowdsourced open source software development has become an
incredibly powerful force in recent years, it still has limitations. 
Open source projects tend to be fragile, in the community sense, and
suffer from disagreements that ultimately result in ``forks'' and the
resulting repetition of effort. We will analyse this in a setup of
cooperative game theory, and try to design a finely tuned systems of
incentives and rewards for contribution, to increase the stability of
the community and its useful output.

We will focus on three areas: prioritisation of bugfixes and feature
requests to achieve reliable and useful systems; effective cooperation between multiple collaborative
projects and and making decisions about
the strategic direction of the system.  

We will use prioritisation as a testbed to identify the appropriate incentives for all
participants which encourage sustained development
of the most important parts of the system.
To this end, we will use
ideas from the burgeoning field of mechanism design \cite{AGTbook} and
in particular on recent research on crowdsourcing in algorithmic
mechanism design \cite{crowds}.  While doing so, we will apply
outcomes to a case study system --- \Sage.  We will apply preference
and opinion aggregation techniques \cite{pref-aggr} to develop a
community prioritisation scheme for \Sage bugs and features requests,
which are presently being maintained on the \Sage TRAC server
\cite{trac-sagemath} and implement them as a TRAC \cite{Trac} add-on.
 Trusting results of computer computations is crucial for
usability; channels for communicating bug reports and fixes need to be
carefully analysed from social point of view.  Commercial closed
source computer algebra and other computational systems often fail to
react to bug reports in a timely manner, and seemingly are falling
into the short-sighted trap of hiding bugs from potential and current
users \cite{misfort}, Open source systems are only marginally better
in this way, as recent computer security scares, such as the one
around Bash \cite{shellshock}, indicate.  A game-theoretic analysis of
this situation will be attempted.


A key strength of free and open-source software models is the ability
to build upon pre-existing software. \GAP, \PariGP, \Singular and
especially \Sage have made heavy use of this. Problems arise over
time, however, as the priorities of the developers of the systems
diverge. Bugs reported by so-called ``downstreanm'' systems may not be
given the same priority, or even believed, as bugs reported by direct
users of the ``upstream'' system, for instance, or incompatible
changes made that are acceptable to the direct user community, but not
to a dependent system. We will explore how sociological insights can
be used to reduce these problems.


Problems with decision making arise because 
development of open-source academic software is task-driven,
where tasks (also known as tickets) are posted on a website, and their
priorities are set in an ad hoc manner.  This latter might be
good enough for simple bug fixing, for more elaborate task this often
leads to delays etc.  We would like to investigate an voting-driven
approach, where the priorities are being voted on by the developer
community, and possibly the people who completed tasks are
incentivised in some form (e.g. by ``karma points'', as on
MathOverflow).
\TODO{This is still a bit of a mess}

\end{task}

\begin{task}[title=Evaluation of Micromagnetic VRE,lead=USO,PM=6,
id=oommf-nb-evaluation,partners={UO,PS}]
  % 4 person months, 1 person month investigator time
  We will use the micromagnetic VRE demonstrator
  (\taskref{UI}{oommf-tutorial-and-documentation}), its dissemination
  workshops \linebreak(\taskref{dissem}{dissemination-of-oommf-nb-workshops})
  and interactions with its users and contributors in the
  micromagnetic community to evaluate, reflect and report on the project,
  taking into account technical and social aspects.

  A survey will be developed and used to gather user input and
  feedback on usefulness of the provided capabilities, with particular
  focus on the capabilities of the micromagnetic VRE to (i) enable new
  and better science, to (ii) allow to make progress effectively, to
  (iii) carry out computational science reproducibly, to (iv)
  collaboratively enable trust and to (v) become a self-sustained
  project from community contributions. Amongst other channels, we
  will target attendees of the micromagnetic VRE dissemination
  workshops (\taskref{dissem}{dissemination-of-oommf-nb-workshops}) to
  gather data.

  All results and insights will be summarised in a public document
  (\localdelivref{oommf-nb-evaluation}) and reported at appropriate
  workshops and conferences to share the lessons learned from this
  \Jupyter-based VRE for micromagnetics. We will create a manuscript
  for journal publication, summarising the demonstrator project and
  this evaluation. An important point of this publication is to
  provide a reference that can be cited by publications making use of
  the new micromagnetic VRE, to allow tracking of uptake and
  development of this VRE beyond the life time of this H2020 project.
\end{task}



\end{tasklist}

% Things to investigate?
% - User surveys. Cf. https://groups.google.com/d/msg/sage-devel/v8Kfky4p6D4/_xRM0bggCo8J
% - The discussion about Code of Conducts and the like

\begin{wpdelivs}
%   \begin{wpdeliv}[due=12,id=social-...,dissem=PU,nature=??]
%       {...}
% \end{wpdeliv}
\begin{wpdeliv}[due=3,id=social-report,dissem=PU,nature=R,lead=UO]
 {Report on relevant research in sociology of mathematics and lessones
   for design of \TheProject VRE}
\begin{wpdeliv}[due=24,id=social-report-two,dissem=PU,nature=R,lead=UO]
 {Report on relevant research in sociology of mathematics and lessones
   for design of \TheProject VRE}
\begin{wpdeliv}[due=42,id=social-report-three,dissem=PU,nature=R,lead=UO]
 {Report on relevant research in sociology of mathematics and lessones
   for design of \TheProject VRE}

 \begin{wpdeliv}[due=36,id=social-poster,dissem=PU,nature=DEM,lead=USH]
   {Demonstrator: Jupyter Notebook Live Poster} 
\end{wpdeliv}
 \begin{wpdeliv}[due=24,id=social-poster,dissem=PU,nature=DEM,lead=USH]
   {Demonstrator: Mechanism for comment on posted Jupyter notebooks.} 
\end{wpdeliv}
\begin{wpdeliv}[due=42,id=social-publishing-report,dissem=PU,nature=R,lead=USH]
{Report on new publication mechanisms, including evaluation of
  demonstrator projects}

\TOWRITE{ Some deliverables from the mechanism design stuff}


 \begin{wpdeliv}[due=48,id=oommf-nb-evaluation,dissem=PU,nature=R,lead=USO]
      {Micromagnetic VRE environment evaluation report}
\end{wpdeliv}
\end{wpdelivs}
\end{workpackage}
%%% Local Variables:
%%% mode: latex
%%% TeX-master: "../proposal"
%%% End:

%  LocalWords:  workpackage wphases TOWRITE wpobjectives analyse wpdescription AGTbook
%  LocalWords:  mathoverflow Sagemath pref-aggr prioritisation trac-sagemath Trac misfort
%  LocalWords:  analysed shellshock tasklist datacollection decisionmaking incentivised
%  LocalWords:  OOMMFNB taskref oommf-python-interface oommf-tutorial-and-documentation
%  LocalWords:  micromagnetic dissem dissemination-of-oommf-nb-virtual-environment texttt
%  LocalWords:  dissemination-of-oommf-nb-workshops summarised delivref wpdelivs wpdeliv
%  LocalWords:  oommf-nb-evaluation compactitem recomputation-style phenomenom
%  LocalWords:  localdelivref
