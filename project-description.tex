\documentclass[a4,12pt]{amsart}

\usepackage[utf8]{inputenc}
\usepackage[T1]{fontenc}
\usepackage[american]{babel}
\usepackage{amssymb,amsbsy,amsmath,amsfonts,amssymb,amscd}
\usepackage{graphicx}
\usepackage{hyperref}
\usepackage{xspace}
\usepackage{eurosym}
\usepackage[left=2.5cm, right=2.5cm, top=2.5cm,bottom=2cm]{geometry}

\newcommand{\sage}{\href{http://www.sagemath.org/}{\texttt{Sage}}\xspace}
\newcommand{\sagecombinat}{\href{http://wiki.sagemath.org/combinat/}{\texttt{Sage-Combinat}}\xspace}
\usepackage{color}
%\newcommand{\TODO}[2][To do: ]{{\textcolor{red}{\textbf{#1#2}}}}
\newcommand{\TODO}[2][To do: ]{}

\usepackage{enumitem}
\setitemize{leftmargin=4ex, noitemsep,topsep=0pt,parsep=0pt,partopsep=0pt}

%%%%%%%%%%%%%%%%%%%%%%%%%%%%%%%%%%%%%%%%%%%%%%%%%%%%%%%%%%%%%%%%%%%%%%%%%%%%%%
% A wider quote
\renewenvironment{quote}{%
  \list{}{%
    \leftmargin0.5cm   % this is the adjusting screw
    \rightmargin\leftmargin
  }
  \item\relax
}
{\endlist}

\newcommand{\CS}{computer science}
\newcommand{\MATH}{mathematics}

%%%%%%%%%%%%%%%%%%%%%%%%%%%%%%%%%%%%%%%%%%%%%%%%%%%%%%%%%%%%%%%%%%%%%%%%%%%%%%
\title[SageMath in Europe]{SageMath in Europe: Mutualized development\\
  of computer exploration environment\\
  for research in mathematics
}
% Titre français:
% Développent logiciel mutualisé pour la recherche en combinatoire et au delà
\date{}
%%%%%%%%%%%%%%%%%%%%%%%%%%%%%%%%%%%%%%%%%%%%%%%%%%%%%%%%%%%%%%%%%%%%%%%%%%%%%%

\begin{document}
\ \vspace{-1.5cm}
\hrule
\medskip
\maketitle
\ \vspace{-1cm}
\hrule
%\subsection*{Long title}

%\TODO{From the application guide: maximum 150 caractères}

%\textbf{Mutualized software development\\for research in combinatorics and beyond}

%\textbf{Sage-Combinat: fostering code sharing in combinatorics and beyond, for computer exploration and research}
% Keywords: in the French community?
% Research on algorithms, benchmarking, ...
% Sage-Combinat: un peu restrictif
% Mission: “To improve the open source mathematical system \sage as an extensible toolbox for computer exploration in combinatorics and beyond, and foster code sharing between researchers in this area”.

\section{Summary}

\subsection{Context and scientific theme}

Computer exploration is an instrumental tool for research in
mathematics; in some cases, computers have even been used for proving
results like the famous four color theorem, or the alternating
sign matrices conjecture.
%\TODO{the analogue of telescopes for astrophysicists}.
The 90's have seen the creation of many specialized open source
packages, like GAP for group theory, Maxima for symbolic computation,
or Pari/GP for number theory.  Yet, specialized packages are
insufficient in many areas, like combinatorics and geometry, where one
needs to handle simultaneously a wide variety of objects and features.
%which only a general purpose software system can provide and support.
However, not being open source and lacking basic computer science
paradigms, general purpose systems like \texttt{Maple} or
\texttt{Mathematica} impede the pooling of the software development
efforts.

To tackle this situation, William Stein started in 2005 the
\textbf{\sage{} project, a free open-source mathematics software
  system}. It provides a full featured \textbf{research environment},
including a rich notebook interface for interactive computations and
visualization, collaborative work on the cloud, or intensive batch
computations. \sage{} builds on the \textbf{Python and Scientific
  Python stack}, enriched by dozens of open-source mathematical
packages, in particular \textbf{GAP, Singular, PARI/GP}, and a native
mathematical library.

\sage{} is developed by a large international community of more than 300
researchers and teachers in maths, computer science, and physics,
using modern collaborative tools for agile development.  A strong
emphasis is put on collaborative development and training by regularly
organising week-long workshops all around the world. \sage{} fosters
inter-disciplinary research by promoting code sharing and reuse in
widely different contexts, including \textbf{physics and computer
  science}. It is \textbf{particularly beneficial for young
  researchers}.

%, giving them access to a platform with a wide array of
%features, and allowing them to focus on those computational aspects
%directly related to their own research topic.

\subsection{Challenges and originality of the solution}

So far the \sage{} ecosystem has been running on a tiny budget for its
size. Yet, long term critical non mathematical features like
portability, modularization, packaging, user interfaces, large data,
parallelism, or outreach toward related software, have been lagging
behind. Indeed they cannot be implemented as a side product of
research projects, and \textbf{need to be assigned to full time
  research engineers}. Regular funding is also needed to better
structure the \sage{} community in Europe and support its upcoming
major widening through training and development workshops, exchanges,
...
%
% Up to now, this has been funded as an aside of research projects, or
% through the generosity of labs.
%

\textbf{To scale to mainstream and assume its leadership}, \sage{} needs
direct funding in addition to indirect funding through research
projects. Such funding already exists in the United States, e.g. with
large % (500k\$)
NSF Computational Mathematics grants; the boosting effect on the
community is already shining (workshops, ICERM semester, many new
recruits, ...).

\medskip
\textbf{The purpose of this proposal is to trigger a similar boost in
  Europe.}

\subsection{Outreach}%Les retombées scientifiques et sociétales}

\sage{} already plays a fundamental role in hundreds of scientific
publications and triggers cross-fertilization across disciplines. Its
usage for \texttt{higher education} is spreading. Those benefits will
be boosted to the measure of \sage{}.  The project will further
contribute back specialized and general purpose tools to the
Scientific Python ecosystem which is used intensively by both the
academic and industrial worlds.


\clearpage

\section{Context and scientific theme}

% \TODO{From the application guide: 2 à 3 pages maximum Le contenu de
%   cette section permet une appréciation selon le premier critère
%   d’évaluation (Intérêt des objectifs scientifiques et
%   technologiques).  Donner les informations permettant d’évaluer
%   l’intérêt des objectifs scientifiques et technologiques, décrire
%   brièvement la méthodologie et/ou la capacité à générer des
%   résultats, le potentiel d’avancée dans le domaine, l’ambition, la
%   nouveauté, le potentiel de rupture.}

Since the 50's computer exploration has been an instrumental tool for
research in mathematics; in some cases, computers have even been used for
proving results like the famous four color theorem, or the
alternating sign matrices conjecture.
%\TODO{the analogue of telescopes for astrophysicists}.
To support this, much efforts have been spent on the software
development. As the sophistication of the required computations
increased, supported by the boom of the available computational power,
it was progressively realized that single topic/single person software
projects, or even projects at the level of a single research team
simply did not scale. It was essential to share the software
development at the level of a research community, both in term of
breath of features and sharing of development efforts. This lead
to the apparition in the 90's of many specialized open source
packages, like GAP for group theory, Maxima for symbolic computation, or
Pari/GP for number theory.

Yet, this proved insufficient in many areas like combinatorics, as
computer exploration often requires simultaneously a wide variety of
objects and features which only a general purpose software system can
provide and support. At the same time, the then existing general
purpose systems like \texttt{Maple} or \texttt{Mathematica} prevented
the scaling of existing packages because they lacked basic computer
science paradigms; even more important, they were not open source.

%\subsection*{Originality of the Sage project}

To tackle this situation, William Stein started in 2005 the \textbf{\sage{}
project, a free open-source mathematics software system}. This project
soon crystallized a large international community of contributors; now
more than 300 researchers and teachers in maths and computer
science. It already played a fundamental role in hundreds of
scientific publications, and its usage for \texttt{higher education}
is spreading.

Building on the very active Python and Scientific Python stack
combined with the power of dozens of open-source mathematical packages
(GAP, Singular, or FLINT to name but a few), \sage{} provides a full
featured research environment including a rich notebook interface for
interactive computations and visualization, as well as support for
intensive batch computations, for example on the cloud. \sage{} is also
the computational workhorse behind sister database projects like
\textsc{LMFDB}, the online database of L-functions, modular forms, and
related objects, or the \textsc{EFD}, the explicit formulas database,
which is an inventory of multiplication formulas for elliptic curve cryptography.

Thanks to a high level programming framework, most of the code can be
written generically and reused in widely different contexts, including
\textbf{physics and computer science}. The induced cross-discipline
interactions both between developers and with users
are a regular source of cross fertilization, and already lead to
several joint research projects.

\sage{} \textbf{is particularly beneficial for young researchers},
giving them access to a platform with a wide array of features, and
allowing them to focus on those computational aspects directly related
to their own research topic. With, at the end, the possibility and
incentives to contribute back their developments to the community
\emph{and get credit for it}.

Collaborative software development involves lots of interactions
within the community, around software and user interface design,
algorithmic, code review, governance, and maintenance. Those
interactions are organized using modern collaborative tools for open
source agile development (distributed version control, mailing lists,
ticket server, testing tools, ...), and following best practices. A
strong emphasis is put on training, in particular at the occasion of
numerous week-long workshops like the upcoming \emph{Sage Days 60} in
India.

\section{Scaling further toward 2020}

After 10 years of existence, \textbf{\sage{} has proven its viability
  and leadership}, surviving the many hurdles faced by new open source
software projects, reaching a comfortable community size, and proving
the adequacy of its ``developed by users, for users'' model.  Yet
\sage{} has not yet been systematically adopted, and still \textbf{much
  good research time is wasted developing and redeveloping tools by
  lack of sharing and coordination}.

By design, the development of \sage{} has been consciously designed to
run on a small budget. This was a key feature toward its independence
and its early wide adoption across research teams and international
frontiers. In particular, it is ``developed by users, for users'':
most if not all of the development is achieved by the researchers
themselves and justified w.r.t. their funding institutions by the
immediate application to their own research projects. Core development
is taken care of by permanent researchers that can afford to invest
time over the long run (10 years). Coordination is achieved
electronically using modern collaborative development tools
(distributed version control, ticket server, mailing lists, ...).

So far, \sage{} has been running viably on a small budget for its size
(300 developers), thanks to its ``developed by users for users''
model. Yet, long term non mathematical features have been lagging
behind because they cannot be split in pieces small enough to be
implemented as a side product of research projects: portability
(including native Windows support), build infrastructure and
modularization, packaging and distribution, improved user interfaces,
core support for parallelism, large data support, ... Those features
are critical to let \sage{} scale to mainstream and assume its
leadership. To this end, \textbf{the development model needs to be
  backed up by full time research engineers}.

It is also vital to organize regular physical meetings (missions,
invitations, workshops) for fast coordination; this becomes even more
critical when it comes to train masses of new developers and users.
Finally, productive software development requires powerful hardware for
compilation, regression testing, continuous integration,
benchmarking.
%
Up to now, this has been funded as an aside of research
projects (\TODO{cite them}), or through the case by case generosity of
labs, taken on their recurrent budget (e.g. developers meeting and
missions funded by the LITIS in Rouen, the LRI and LMO in Orsay,
... \TODO{expand to the European scale}), but it has reached its
limits, with much time wasted on patching together little pieces of
funding.
%



\textbf{To scale further, \sage{} needs direct funding in addition to
  indirect funding through research projects.} Such funding already
exists in the United States, e.g. with large (500k\$) NSF
Computational Mathematics grants; the boosting effect of those on the
community is already shining there (workshops, ICERM semester, many
new recruits, ...).

\medskip
\textbf{The purpose of this proposal is to trigger a similar boost in
  Europe.}

\section{Outreach}

\TODO{Insert this somewhere: Dissemination: Sharing code increases our
  visibility, spreads our research results to other scientific areas,
  and benefits education.}

\subsection{Teaching}

We plan to integrate \sage{} into the curriculum of the courses at our
universities. Many undergraduate and graduate courses in number theory
and combinatorics already involve computational components. Moreover,
since this year, Sage plays an important role in the Agrégation. With
the dwindling resources at the universities, a free and open-source
alternative to the currently used systems % such as {\sc Mathematica}
%and {\sc MatLab}
is most welcome. Since \sage{} is based on Python, one of the top five
programming languages, this also gives the students valuable
education.  Inspired students can furthermore look at the algorithms
or even contribute back to \sage{}.

\TODO{Reuse stuff from the NSF grant}

\section{Consortium}

By design, \sage{} is developed by a community of individually involved
researchers, transversely to the institutions. They form a geographically
dispersed group of people, whose fields of expertise cover a wide continuum of
research themes in discrete mathematics, dynamical systems, geometry, algebra,
or number theory, with deep connections to neighbor fields like symbolic
computations or statistical physics.
%
% With that respect, the french situation an isolated case.
%
% TODO : reutiliser l'argument de dispersion pour le dev d'infrastructures
% collaboratives a distance.


In France, a core team has been closely collaborating around Sage for a long time.
%Delecroix, Cohen, Labbé, Lelièvre, Hivert, Monteil, Mezzarobba, Saliola,
%Thiéry.
In particular, they co-organized and participated to a stream of events that
included (international) developer meetings, user group meetings, training
sessions (Burkina Faso, France, Québec), and development of educational
material.

The \textsc{\sagecombinat} project, a strong international community of
collaborators founded in 2000 that focusses on combinatorics, will be involved
in the project, including a large number of PhD students and postdocs.

The European consortium, in building, aims at scaling this dynamics and
experience in both geographical and thematic directions.
%
We already got positive feedback and engagement of researchers from Austria
(Linz, Wien: \CS), France (Rouen, Bordeaux, Marne, Saclay: \CS ; Lyon,
Marseille: math ; Paris: math, \CS, physics), Germany (Hannover, Berlin, Jena:
math), Switzerland (Zürich: math), Great Britain (Oxford, Bristol, Warwick: math and \CS),
Ireland (Dublin: physics), Québec (Montréal, joint lab CRM/CNRS: \CS, math).

\section{Actions}
\subsection{Collaborative tools}

\begin{itemize}
\item Development of an online repository for indexing and sharing
  notebooks and other Sage resources.
  %\TODO{Be clearer: we want to get rid of notebooks over the
  %  filesystem and .sws tarball}
\item Workflow improvements: continuous integration, ...
\item Collaborative interactive tools: e.g. embedding live shared
  \sage sessions (and more generally IPython) in voice-over-IP or
  teleconference calls.
  % Eugen Dedu:
  % I think such a module can be thought of as a screen-capturing
  % module, i.e. allow Ekiga to capture the screen of a Sage user (this
  % is currently not possible).  This is not a difficult task to do.
  % Julien Puydt: ekiga can do that since something like 2008 with my
  % experimental gstreamer plugin, and I shall be able to present
  % interesting sample code to the ekiga-devel mailing-list in something
  % like two-three weeks (after I'm done with my students), which will
  % hopefully be part of the next version.
  %
  % But as Nicolas noted in his answer, some kind of interative session
  % where people can share a sage session would be better.
  %
  % I think the feature decomposes in the following pieces:
  % - IPython should have a way to share sessions between several
  %   participants using an open and standard protocol ;
  % - ekiga should implement it.

  % Eugen Dedu:
  % In my opinion ekiga, because of its dependency on ptlib and opal
  % libraries and the use of complex protocols like SIP and H323, needs
  % highly technical people.  Students cannot help much, but engineers
  % are appropriate.

%\item \TODO{please expand!}
\item Sharing best practices and tools between sister projects (GAP, ...)
\end{itemize}

%\TODO{lmonade has very similar objectives but uses the gentoo prefix whereas Linux distributions use very different packaging systems:
%\begin{itemize}
%\item gentoo prefix (gentoo)
%\item pacman (arch),
%\item yum (redhat),
%\item apt (debian),
%\item easy\_install
%\item Python index packaging (pip)
%\end{itemize}}

\subsection{Modularization of the Sage distribution}
Separation of the different components of Sage (communication with
third-party softwares, build system, Sage native code). This is a
prerequisite for easier packaging and integration in standard Linux
distributions and lmonade, native integration within the IPython
notebook and other interfaces (larchenv, Spyder, ...) and
collaboration with sister projects.

\subsection{Deployment}
\begin{itemize}
\item Better Windows support
\item Live USB keys
\item Creation, deployment, and distribution of preconfigured virtual
machines for Sage as a cloud service, in particular within the
StratusLab infrastructure.
\end{itemize}
 % http://sagedebianlive.metelu.net/

\subsection{Interfaces with other computation systems}

\begin{itemize}
\item Support for \href{http://www.symbolic-computing.org/}{SCSCP}
%\item \TODO{expand}
\item foster collaboration with upstream libraries by sharing the
  development and maintenance of the interfaces (typically as
  standalone Python bindings)
\end{itemize}


\subsection{Community animation and training}

\begin{itemize}
\item Sage Days (training workshops and developers meetings)
  \begin{itemize}
  \item \TODO{<put suggestions of locations / dates here>}
  \end{itemize}
\item Exchange visits for students, researchers, ... in between
  partner institutions, and with developers worldwide
\item Strengthening of collaborations with sister projects: GAP,
  Linbox, Singular, PARI/GP, IPython, Scientific Python, R, to name but a
  few.
\item \TODO{please expand}
\end{itemize}



\subsection{Math features}

\begin{itemize}
\item Foundation work: categories, morphisms, coercion, ...
\item Generic parallelism tools for combinatorics
\item Representation theory of monoids and algebras
\item Ore algebras and applications to combinatorics
\item Further development of combinatorial species
\item Automatons and combinatorics on words
  % Interface with Vaucanson, ...
  % https://github.com/coulbois/sage-train-track
  % https://www.lrde.epita.fr/wiki/Vaucanson
  % More support for group theory (especially free group automorphisms, mapping class group of surfaces, braid groups)
\item Symbolic dynamical systems (substitutive systems, Bratelli diagrams,
  sofic shifts, tilings, interval exchange transformations)
\item Manifolds (symbolic computations in charts, differential equations, etc)
  % http://sagemanifolds.obspm.fr/
\item Intersection theory in algebraic geometry
  % http://www.math.sciences.univ-nantes.fr/~sorger/chow_en.html
\item Complexes of modules, resolutions.
\item Graded commutative differential algebras and applications to rational homotopy.
\item Standard bases for modules over path algebras
\item Modular cohomology and Ext algebras for basic algebras
  % going beyond the optional spkg http://sage.math.washington.edu/home/SimonKing/Cohomology/
\item ...
%\item \TODO{please expand!}
\end{itemize}

\subsection{Parallel Computing}

\begin{itemize}
 \item Transparent integration of Ipython capabilities for cluster computing.
 \item Customization of blas/lapack implementation to take advantage of GPU computing.
 \item Implementation of a transparent abstraction over mpi.
 \item Develop or integrate existing solutions for MapReduce operations over big data.
\end{itemize}


\subsection{Singular}

- Interfaces: libSingular, pySingular?, GAP-Singular, Singular-Sage
- FLINT development (key component for several systems)
- Parallelism, portability, ...
- User interface: IPython notebook, 3D visualization of algebraic surfaces, ..
- Sharing best practices for software development
- Moving code from Sage into Singular when relevant
- Some demonstrators of cross-disciplinary/cross-software calculations


\section{Tentative budget}

\begin{tabular}{|l|r|}\hline
  Research engineer (5 pers for 5 years)                       & 1250k\euro          \\\hline
  Postdoc           (2 pers for 5 years)                       &  400k\euro\\\hline
  Invitations and missions (national / international), student internships & 100 k\euro          \\\hline
  Sage Days and developers meetings (1+2 per year)             & 100k\euro    \\\hline
  Subcontracting, participation to the Systematic cluster (GT Logiciel Libre)  & 50 k\euro\\\hline
  Hardware                                                     & 100 k\euro\\\hline
  Total (including 4\% overhead)                               & 2000 k\euro\\\hline
\end{tabular}


% \TODO{From the application guide: (1 à 2 pages maximum) Le contenu de
%   cette section permet une appréciation selon le troisième critère
%   d’évaluation (Cohérence de la pré-proposition par rapport aux
%   objectifs du projet) Donner les informations relatives aux
%   compétences requises pour mener le projet objet de la
%   pré-proposition, en précisant l’identité du ou des scientifique(s)
%   impliqué(s), l’identification des Partenaires 4 auxquels il(s)
%   est(sont) rattaché(s) et 2-3 références pertinentes dans le domaine
%   en lien direct avec la pré-proposition (publications, faits
%   marquants de R\&D, brevets, prix scientifiques, produits, procédés,
%   licences, services...), et tout autre élément permettant de juger de
%   la qualité des déposants et du consortium le cas échéant.}

% The consortium is built around a core team that has been closely
% collaborating around Sage in France and Québec for a long time:
% Delecroix, Cohen, Labbé, Lelièvre, Hivert, Monteil, Mezzarobba,
% Saliola, Thiéry. In particular they coorganized and participated to a
% stream of events that included (international) developer meetings,
% training sessions, and development of educational material:
% \begin{itemize}
% \item \href{http://wiki.sagemath.org/GroupeUtilisateursParis}{Groupe
%     d'utilisateurs Sage en région parisienne}: a monthly meeting;
%   %Labbé, Hivert, Thiéry, Valibouze, et al.
% \item \href{http://www.ragaad.org/bobo2012/}{\emph{CIMPA discrete math research school}},
%     % : «Mathématiques discrètes: aspects combinatoires, dynamiques et algorithmiques»}},
%     Bobo-Dioulasso (Burkina Faso), Fall 2012;
%     %Lelièvre, Delecroix, Monteil, Thiéry et al.
% %\item \href{http://smai.emath.fr/smai2011/programme_slides.php}{\emph{Congrès SMAI}}, Mini-symposium \sage, Guidel May 2011;
% %  Mezzarobba, Thiéry et al.
% \item Sage course, \emph{\'Ecole Jeunes Chercheurs Informatique
%     Math\'ematique}, %April
%   Perpignan, 2013;
%   % : Monteil, Cohen,  Delecroix, Lelièvre.
% \item Three week-long developer meetings in Cernay, Orsay (2009, 2010, 2012);
%   % Borie, Cohen, Boussicault, Delecroix, Hivert, Labbé, Monteil, Saliola, Thiéry et al.
% \item \href{http://wiki.sagemath.org/combinat/FPSAC13}{Sage Days 49},  Orsay, %June
%   2013;
%   % , Cohen, Hivert, Mezzarobba, Monteil, Thiéry.
% %\item
%   \href{http://wiki.sagemath.org/days38}{Sage Days 38}, Montréal, %May
%   2012;
%   % , Brlek, Hivert, Labbé, Saliola, Thiéry et al.
% %\item
%   \href{http://wiki.sagemath.org/days28}{Sage Days 28}, %: dynamics, geometry, combinatorics},
%   Orsay, %January
%   2011;
%   %, Borie, Cohen, Delecroix, Hivert, Labbé, Lelièvre, Monteil, Thiéry et al.
% \item \href{https://www.lirmm.fr/arith/wiki/MathInfo2010/SageDays}{\emph{Sage Days 20}},
%   during the \emph{Math-Info 2010} thematic month at CIRM, Luminy;%, February 2010
%   % : a one week workshop during the \emph{Math-Info 2010} thematic month at
%   %CIRM, Luminy, France, February 2010; Delecroix, Hivert, Labbé,
%   %Monteil, Saliola, Thiéry et al.
% \item Introductory book «\emph{Calcul Mathématique avec Sage}»,
%   % Cohen, Thiéry, Mezzarobba contributed a third of the chapters.
% %\item
%  Stream of Sage thematic tutorials.
%   %: Cohen, Labbé, Lelièvre, Hivert, Saliola, Thiéry et al.
% \end{itemize}


%\bibliographystyle{alpha}
%\bibliography{project-description}
\end{document}
